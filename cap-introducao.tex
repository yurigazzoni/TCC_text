\chapter{Introdução}
\label{Chap:intro}

O desenvolvimento de satélites de pequeno porte, como CubeSats e nanosatélites, trouxe novas oportunidades e desafios para a indústria espacial, permitindo que uma ampla gama de missões científicas, comerciais e educacionais fosse realizada com custos reduzidos e prazos de desenvolvimento mais curtos (CUBESAT Design Specification, 2022). No entanto, a miniaturização e a operação em ambientes espaciais impõem requisitos à robustez, confiabilidade e versatilidade dos sistemas embarcados, especialmente para os módulos OBDHs (\textit{On-Board Data Handling}). Nesse contexto, a arquitetura de um computador de bordo eficiente e robusto é essencial para o gerenciamento seguro das operações e para garantir a integridade das missões.

O sistema desenvolvido para o presente trabalho foca na implementação de uma arquitetura de processamento e memória capaz de atender às demandas de um satélite de pequeno porte. Esse sistema deve operar de maneira confiável em ambientes de alta radiação e alta variação de temperatura, em conjunto com a otimização do uso de energia.

Além disso, a versatilidade do computador de bordo é essencial para adaptar o sistema a diferentes tipos de missões, desde operações de imagem e telemetria até experimentos científicos em órbita. Para isso, é necessário que o sistema ofereça uma arquitetura versátil, baseada em uma FPGA (\textit{Field-Programmable Gate Array}), com capacidade de expansão e adaptação a novos sensores e módulos de comunicação.

\section{Objetivo geral}

O presente trabalho tem como objetivo o projetar e implementação de uma arquitetura de hardware robusta e versátil para um computador de bordo de satélite de pequeno porte, integrando diferentes tipos de memórias e periféricos para assegurar a operação confiável em ambientes espaciais adversos, garantindo a integridade dos dados e a eficiência no gerenciamento dos mesmos.

\section{Objetivos Específicos}

\begin{itemize}
    \item Analisar os requisitos de robustez em condições espaciais, com foco em resistência a radiação, tolerância a falhas e estabilidade térmica, a fim de assegurar o funcionamento contínuo do computador de bordo em órbita.
    \item Especificar uma arquitetura de hardware que permita a adaptação a diferentes tipos de missões, integrando diferentes tipos de componentes comerciais com um SoC (\textit{System-on-a-chip}).
    \item  Documentar as decisões de projeto para consolidar um guia técnico com recomendações de design para sistemas de robustos e versáteis aplicáveis a satélites de pequeno porte, contribuindo para futuras otimizações e adaptações em missões espaciais.
\end{itemize}
