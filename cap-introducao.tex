\chapter{Introdução}
\label{Chap:intro}

CubeSats são pequenos satélites que atendem a estritas formas de cubos padronizados de 10 cm de aresta, além de pesarem menos de 300 kg. Cada um desses cubos padronizados recebe a denominação de 1U, e os tamanhos subsequentes de 1,5U, 2U, 3U, e assim por diante (CUBESAT Design Specification, 2022). Devido a essa padronização e ao uso de componentes comerciais, os CubeSats podem ser produzidos em massa, o que diminui substancialmente os custos de lançamento e desenvolvimento (CubeSat 101, 2017).

O desenvolvimento de satélites de pequeno porte, como CubeSats e nanosatélites, trouxe novas oportunidades e desafios para a indústria espacial, permitindo que uma ampla gama de missões científicas, comerciais e educacionais fosse realizada com custos reduzidos e prazos de desenvolvimento mais curtos (CUBESAT Design Specification, 2022). No entanto, a miniaturização e a operação em ambientes espaciais impõem requisitos à robustez, confiabilidade e versatilidade dos sistemas embarcados, especialmente para os módulos OBDHs (\textit{On-Board Data Handling}). Nesse contexto, a arquitetura de um computador de bordo eficiente e robusto é essencial para o gerenciamento seguro das operações e para garantir a integridade das missões.

Essa segurança e integridade são pontos chave no desenvolvimento de CubeSats no SpaceLab, laboratório da UFSC especializado em desenvolvimento de sistemas espaciais para a comunidade científica e para a indústria. Um dos objetivos primários do SpaceLab é o desenvolvimento de uma plataforma \textit{open-source}, tanto para \textit{software} quanto \textit{hardware}, o que já foi feito nos desenvolvimentos do FloripaSat-1 (MARCELINO et al., 2020) e FloripaSat-2 (MARCELINO et al., 2024). Com esse paradigma, a oportunidade de se ter um computador de bordo mais robusto (com mais memória e capacidade de processamento) e versátil surgiu, como uma consequência direta dos desenvolvimentos das gerações anteriores de \textit{hardware} do laboratório.

O sistema desenvolvido para o presente trabalho foca na implementação de uma arquitetura de processamento e memória capaz de atender às demandas de um satélite de pequeno porte. Esse sistema deve operar de maneira confiável em ambientes suscetíveis à radiação e alta variação de temperatura, em conjunto com a otimização do uso de energia. Além disso, a versatilidade do computador de bordo é essencial para adaptar o sistema a diferentes tipos de missões, desde operações de imagem e telemetria até experimentos científicos em órbita. Para isso, é necessário que o sistema ofereça uma arquitetura versátil, baseada em uma FPGA (\textit{Field-Programmable Gate Array}), com capacidade de expansão e adaptação a novos sensores e módulos de comunicação.

Inicialmente, será feita uma revisão bibliográfica, explorando as características de computadores de bordo comerciais e de trabalhos acadêmicos, além de entender como a radiação em órbita baixa afeta os sistemas eletrônicos. Depois disso, será definida uma arquitetura, respeitando os requisitos impostos, em conjunto com uma estimativa de consumo de potência. Com a arquitetura, será desenvolvido um esquemático, usando o software Altium Designer (versão 24.4.1). Por fim, serão apresentados os resultados obtidos e as considerações finais e conclusões para esse projeto.

\section{Objetivo geral}

O presente trabalho tem como objetivo o projetar e implementação de uma arquitetura de hardware robusta e versátil para um computador de bordo de satélite de pequeno porte, integrando diferentes tipos de memórias e periféricos para assegurar a operação confiável em ambientes espaciais adversos, garantindo a integridade dos dados e a eficiência no gerenciamento dos mesmos.

\section{Objetivos Específicos}

\begin{itemize}
    \item Analisar os requisitos de robustez em condições espaciais, com foco em resistência a radiação, tolerância a falhas e estabilidade térmica, a fim de assegurar o funcionamento contínuo do computador de bordo em órbita.
    \item Especificar uma arquitetura de hardware que permita a adaptação a diferentes tipos de missões, integrando diferentes tipos de componentes comerciais com um SoC (\textit{System-on-a-chip}).
    \item  Documentar as decisões de projeto para consolidar um guia técnico com recomendações de design para sistemas de robustos e versáteis aplicáveis a satélites de pequeno porte, contribuindo para futuras otimizações e adaptações em missões espaciais.
\end{itemize}
