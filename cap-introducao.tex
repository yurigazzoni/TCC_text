\chapter{Introdução}
\label{Chap:intro}

A Introdução é um texto sucinto e direto, no qual deve constar, organizado em parágrafos, com textualidade (coesão e coerência) e sem vícios linguísticos:

\begin{enumerate}[label=\alph*)]
    \item Tema;
    \item Problema;
    \item Justificativa;
    \item Informação das bases/linhas teóricas e/ou autores de referência que serão utilizados no trabalho;
    \item Objetivo geral e, se for o caso, alguns específicos;
    \item Metodologia a ser seguida;
    \item Resultados gerais.
\end{enumerate}

Não crie subtítulos para motivação ou metodologia.
Motivação é justificativa e deve compor o texto.
Para qualquer dúvida, consulte a Norma, não siga exemplos de formatação observados em outros textos, pois podem estar desatualizados ou equivocados.

\section{Objetivos}

Para resolver a problemática X, propõem-se os seguintes objetivos.

\subsection{Objetivo geral}

O objetivo geral deve ser claro, sucinto, direto e coerente com o que foi anunciado no título do trabalho.

\subsection{Objetivos Específicos}

\begin{itemize}
    \item Utilize a lista de verbos indicada para composição de objetivos específicos, conforme  \href{https://www.youtube.com/watch?v=Ycl5a-5gR4w}{disponível no material da disciplina de PTCC};
    \item Os objetivos específicos atingem metas em fases de começo, meio e fim, da pesquisa;
    \item Observe para não colocar a tarefa, mas sim o objetivo que deseja atingir com a mesma.
\end{itemize}
