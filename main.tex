\documentclass[
12pt,				% tamanho da fonte
%openright,		% capítulos começam em pág ímpar (insere página vazia caso preciso)
oneside,			% para impressão no anverso. Oposto a twoside
a4paper,			% tamanho do papel. 
chapter=TITLE,		% títulos de capítulos convertidos em letras maiúsculas
section=TITLE,		% títulos de seções convertidos em letras maiúsculas
%subsection=TITLE,	% títulos de subseções convertidos em letras maiúsculas
%subsubsection=TITLE,% títulos de subsubseções convertidos em letras maiúsculas
% -- opções do pacote babel --
english,			% idioma adicional para hifenização
brazil				% o último idioma é o principal do documento
hyperref=hidelinks]{abntex2}

\usepackage[semdicas]{tccctj} % carrega o estilo CTJ
\bibliographystyle{abntex2-alf-ufsc} % Arquivo bst com patch para correção de citação de proceedings. Produz italico em "In", conforme descrito em: https://github.com/abntex/abntex2/issues/226

% Useful packages
\usepackage{amsmath}
\usepackage{graphicx}
% Pacotes adicionais
\usepackage{multicol}
\usepackage{multirow}
\usepackage{tabularx}
\usepackage{quoting}
\quotingsetup{indentfirst={false},font={footnotesize},leftmargin=4cm,rightmargin=0cm} 
\hypersetup{hidelinks}
\usepackage{enumitem}
\setlist{noitemsep}
%\usepackage{hyperref}
%\usepackage{array,epsfig}
%\usepackage{amsfonts}
%\usepackage{amssymb}
%\usepackage{amsxtra}
%\usepackage{amsthm}
%\usepackage{mathrsfs}
%\usepackage{color}
%\usepackage{indentfirst}
\usepackage{float}
\usepackage{booktabs}
\usepackage{array}
\usepackage{longtable}
\usepackage[nobiblatex]{xurl}
\usepackage{textcomp}

% ----------------------------------------------------

% ----------------------------------------------------
% Informações do trabalho
\autor{Yuri Gazzoni Rezende}
\titulo{Projeto de um computador de bordo versátil para uso em pequenos satélites}
% \subtitulo{Subtítulo} % Preenchimento opcional para quando houver sub-título
\orientador{MSc. Gabriel Mariano Marcelino}
%\coorientador{Nome do coorientador} % Preenchimento opcional para quando houver co-orientador
\curso{Departamento de Engenharia Elétrica e Eletrônica}
% \titulacao{Bacharel em x}
\datadadefesa{18}{novembro}{2024}
\local{Florianópolis}

\begin{document}

% ----------------------------------------------------
% 1. Capa do trabalho
\imprimircapa

% 2. Folha de rosto
\imprimirfolhaderosto

\newpage
\null
\vfill
\begin{figure}[H]
    \centering
    \includegraphics[width=0.8\linewidth]{images/ficha-catalografica.png}
\end{figure}


% 3. Folha de aprovação
\begin{folhadeaprovacao}
    \membrodabanca[Orientador]{Msc. Gabriel Mariano Marcelino}{}
    \membrodabanca{Prof. Dr. Eduardo Bezerra}{UFSC}
    \membrodabanca{Eng. André Martins Pio de Mattos}{Université de Montpellier}
    \membrodabanca{Eng. João Cláudio Elsen Barcellos}{UFSC}
\end{folhadeaprovacao}

% 4. Dedicatória
\begin{dedicatoria}
    Este trabalho é dedicado aos meus pais, que sempre me apoiaram em todas escolhas difíceis.
\end{dedicatoria}

% 5. Agradecimentos 
\begin{agradecimentos}
	Primeiramente, gostaria de agradecer minha família, em especial meus pais, Carlos e Edina, por sempre me darem o suporte necessário e me impulsionarem, quaisquer que fossem minhas escolhas.

	Também gostaria de agradecer à minha namorada, Nathalia, por estar ao meu lado durante todo esse trajeto, compartilhando os momentos importantes que fizeram parte do meu crescimento pessoal e profissional.

	Agradeço também aos meus amigos e colegas, em especial Bruno, Caio, João, Luciane, Matheus, Sabrina, Vinicius e Vitória, por deixarem momentos difíceis mais leves.

	Por fim, agradeço ao meu orientador, Gabriel, por todo o suporte e orientação, tornando possível a realização desse trabalho de conclusão de curso.
	
\end{agradecimentos}

% 7. Resumo em português
\begin{resumo}
Este trabalho apresenta o projeto e desenvolvimento de uma arquitetura de hardware para um computador de bordo adequado para pequenos satélites, como CubeSats, que operam em órbitas baixas (LEO). O objetivo principal foi desenvolver uma solução que atendesse aos requisitos críticos do ambiente espacial, integrando componentes comerciais (COTS) selecionados conforme diretrizes de herança de voo e normas da ESA e NASA. Para garantir versatilidade, a solução foi baseada em um SoC da família Zynq, que integra um microprocessador e um FPGA, que permite a mudança da funcionalidade de cada pino de entrada e saída. O projeto adota métodos de escolha de componentes e estimativas de potência. Como resultado, obtém-se um esquemático eletrônico baseado em uma arquitetura com interfaces genéricas, apresentando circuitos que possibilitam futuros desenvolvimentos para missões ou estudos específicos em órbita. Conclui-se que o computador de bordo projetado cumpre as exigências de flexibilidade para missões espaciais em pequenos satélites.

\palavrachave{Computador de Bordo}
\palavrachave{Satélites}
\palavrachave{Sistemas Embarcados}
\palavrachave{COTS}
\end{resumo}

\begin{resumo}[Abstract]
\begin{otherlanguage*}{english}
This project presents the design and development of a hardware architecture for an on-board computer suitable for small satellites, such as CubeSats, operating in low Earth orbit (LEO). The main objective was to develop a solution that meets the critical requirements of the space environment, integrating commercial components (COTS) selected according to flight heritage guidelines and ESA and NASA standards. To ensure versatility, the solution was based on a SoC from the Zynq family, which integrates a microprocessor and an FPGA, allowing the functionality of each input and output pin to be changed. The project adopts methods for choosing components and estimating power. The result is a schematic capture based on an architecture with generic interfaces, featuring circuits that enable future developments for specific missions or studies in orbit. It is concluded that the proposed on-board computer meets the requirements of flexibility for space missions on small satellites.
\vspace{\onelineskip}
\noindent 

\textbf{Keywords}: On-board Computer. Satellites. Embedded Systems. COTS.
 \end{otherlanguage*}
\end{resumo}

% 9. Lista de figuras
\imprimirlistafiguras

% 10. Lista de quadros
%\imprimirlistaquadros

% 11. Lista de tabelas
\imprimirlistatabelas

% 12. Lista de abreviaturas e siglas
\begin{siglas}
	\item[ADC] \textit{Analog to Digital Converter}
	\item[ADCS] \textit{Attitude Determination and Control System}
	\item[CAN] \textit{Controller Area Network}
	\item[COTS] \textit{Commercial-off-the-shelf}
	\item[DDR] \textit{Double Data Rate}
	\item[ECSS] \textit{European Cooperation for Space Standardization}
	\item[EMI] \textit{Electromagnetic Interference}
	\item[ESA] \textit{European Space Agency}
	\item[ESR] \textit{Equivalent Series Resistor}
	\item[FPGA] \textit{Field-Programmable Gate Array}
	\item[HP] \textit{High Performance} 
	\item[HR] \textit{High Range}
	\item[I2C] \textit{Inter-Integrated Circuit}
	\item[JTAG] \textit{Joint Test Action Group}
	\item[LEO] \textit{Low Earth Orbit}
	\item[MEO] \textit{Medium Earth Orbit}
	\item[MIO] \textit{Multiplexed In-Out}
	\item[NASA] \textit{National Aeronautics and Space Administration}
	\item[NPSL] \textit{NASA Part Selection List}
	\item[OBDH] \textit{On-board Data Handling} 
	\item[PCB] \textit{Printed Circuit Board}
	\item[PL] \textit{Programmable Logic}
	\item[PS] \textit{Processing System}
	\item[QSPI] \textit{Quad Serial Peripheral Interface}
	\item[RTC] \textit{Real Time Clock}
	\item[RTOS] \textit{Real Time Operational System}
	\item[SDR] \textit{Single Data Rate}
	\item[SEE] \textit{Single Event Effects}
	\item[SoC] \textit{System-on-a-Chip}
	\item[SPI] \textit{Serial Peripheral Interface}
	\item[TID] \textit{Total Ionizing Dose}
	\item[UART] \textit{Universal Asynchronous Receiver-Transmitter}
	\item[WDT] \textit{Watchdog Timer}	
	\item[XADC] \textit{Xilinx Analog to Digital Converter}
\end{siglas}

% 13. Lista de símbolos 

% 14. Sumário
\imprimirsumario

% ----------------------------------------------------

% Elementos textuais
\textual


\chapter{Introdução}
\label{Chap:intro}

CubeSats são pequenos satélites que atendem a estritas formas de cubos padronizados de 10 cm de aresta, além de pesarem menos de 300 kg. Cada um desses cubos padronizados recebe a denominação de 1U, e os tamanhos subsequentes de 1,5U, 2U, 3U, e assim por diante (CUBESAT Design Specification, 2022). Devido a essa padronização e ao uso de componentes comerciais, os CubeSats podem ser produzidos em massa, o que diminui substancialmente os custos de lançamento e desenvolvimento (CubeSat 101, 2017).

O desenvolvimento de satélites de pequeno porte, como CubeSats e nanosatélites, trouxe novas oportunidades e desafios para a indústria espacial, permitindo que uma ampla gama de missões científicas, comerciais e educacionais fosse realizada com custos reduzidos e prazos de desenvolvimento mais curtos (CUBESAT Design Specification, 2022). No entanto, a miniaturização e a operação em ambientes espaciais impõem requisitos à robustez, confiabilidade e versatilidade dos sistemas embarcados, especialmente para os módulos OBDHs (\textit{On-Board Data Handling}). Nesse contexto, a arquitetura de um computador de bordo eficiente e robusto é essencial para o gerenciamento seguro das operações e para garantir a integridade das missões.

Essa segurança e integridade são pontos chave no desenvolvimento de CubeSats no SpaceLab, laboratório da UFSC especializado em desenvolvimento de sistemas espaciais para a comunidade científica e para a indústria. Um dos objetivos primários do SpaceLab é o desenvolvimento de uma plataforma \textit{open-source}, tanto para \textit{software} quanto \textit{hardware}, o que já foi feito nos desenvolvimentos do FloripaSat-1 (MARCELINO et al., 2020) e FloripaSat-2 (MARCELINO et al., 2024). Com esse paradigma, a oportunidade de se ter um computador de bordo mais robusto (com mais memória e capacidade de processamento) e versátil surgiu, como uma consequência direta dos desenvolvimentos das gerações anteriores de \textit{hardware} do laboratório.

O sistema desenvolvido para o presente trabalho foca na implementação de uma arquitetura de processamento e memória capaz de atender às demandas de um satélite de pequeno porte. Esse sistema deve operar de maneira confiável em ambientes suscetíveis à radiação e alta variação de temperatura, em conjunto com a otimização do uso de energia. Além disso, a versatilidade do computador de bordo é essencial para adaptar o sistema a diferentes tipos de missões, desde operações de imagem e telemetria até experimentos científicos em órbita. Para isso, é necessário que o sistema ofereça uma arquitetura versátil, baseada em uma FPGA (\textit{Field-Programmable Gate Array}), com capacidade de expansão e adaptação a novos sensores e módulos de comunicação.

Inicialmente, será feita uma revisão bibliográfica, explorando as características de computadores de bordo comerciais e de trabalhos acadêmicos, além de entender como a radiação em órbita baixa afeta os sistemas eletrônicos. Depois disso, será definida uma arquitetura, respeitando os requisitos impostos, em conjunto com uma estimativa de consumo de potência. Com a arquitetura, será desenvolvido um esquemático, usando o software Altium Designer (versão 24.4.1). Por fim, serão apresentados os resultados obtidos e as considerações finais e conclusões para esse projeto.

\section{Objetivo geral}

O presente trabalho tem como objetivo o projetar e implementação de uma arquitetura de hardware robusta e versátil para um computador de bordo de satélite de pequeno porte, integrando diferentes tipos de memórias e periféricos para assegurar a operação confiável em ambientes espaciais adversos, garantindo a integridade dos dados e a eficiência no gerenciamento dos mesmos.

\section{Objetivos Específicos}

\begin{itemize}
    \item Analisar os requisitos de robustez em condições espaciais, com foco em resistência a radiação, tolerância a falhas e estabilidade térmica, a fim de assegurar o funcionamento contínuo do computador de bordo em órbita.
    \item Especificar uma arquitetura de hardware que permita a adaptação a diferentes tipos de missões, integrando diferentes tipos de componentes comerciais com um SoC (\textit{System-on-a-chip}).
    \item  Documentar as decisões de projeto para consolidar um guia técnico com recomendações de design para sistemas de robustos e versáteis aplicáveis a satélites de pequeno porte, contribuindo para futuras otimizações e adaptações em missões espaciais.
\end{itemize}

\include{cap-revbib.tex}
% ----------------------------------------------------------
\chapter{Arquitetura}
% ----------------------------------------------------------

Após estudar os desdobramentos dos efeitos de órbita baixa e entender o que é necessário para se realizar um projeto confiável de computador de bordo de um nanossatélite através de projetos anteriores, foi necessária a compreensão dos requisitos de projeto. Com isso, foram escolhidos os componentes principais da placa, propondo-se uma arquitetura para o sistema para um \textit{hardware} confiável, robusto e versátil.  

\section{Requisitos de Projeto}

Como dito, foi preciso entender os requisitos impostos para o OBDH da terceira geração do SpaceLab. Com base nas necessidades levantadas para o laboratório nos próximos projetos e na revisão do estado da arte, são apresentados na Tabela \ref{tab:Tab_Req} os requisitos gerais do projeto, em conjunto com a \textit{rationale} e com o método de verificação (NASA Product Verification, 2024). 

\begin{longtable}{@{}>{\centering}p{1.5cm}p{4cm}p{4cm}p{4.7cm}@{}}
    \centering
	\ABNTEXfontereduzida
	\label{tab:Tab_Req}\tabularnewline
	\caption{Requisitos do projeto.}\tabularnewline
	%\begin{tabular}{@{}p{2cm}p{2cm}p{2cm}p{2cm}p{2cm}p{2cm}p{3cm}@{}}
	\hline
	\multicolumn{1}{c}{\textbf{Índice}} & \multicolumn{1}{c}{\textbf{Descrição}} & \multicolumn{1}{c}{\textbf{Rationale}} & \multicolumn{1}{c}{\textbf{Método de Verificação}} \tabularnewline
        \hline
        1 & O módulo OBDH deve ser compatível com o padrão CubeSat & Assegura compatibilidade com outros satélites desenvolvidos no SpaceLab & Inspeção \tabularnewline
        
       \hline
        2 & O módulo OBDH deve operar corretamente entre -40°C e 85°C & Para operar com segurança em um ambiente LEO & Teste e Análise \tabularnewline

       \hline
       3 & O módulo OBDH deve possuir um processador capaz de usar um sistema Linux & Para gerenciar e coordenar operações dentro e fora do módulo, sendo capaz de realizar tarefas complexas definidas pela equipe  & Inspeção \tabularnewline

       \hline
        4 & O módulo OBDH deve possuir uma memória DDR com capacidade de 512Mb (preferencialmente com ECC)  & Memória suficiente para operações do OBDH e armazenamento de dados  & Inspeção\tabularnewline

        \hline
       5 & O módulo OBDH deve possuir uma memória FRAM para armazenar parâmetros de configuração & Provê memória não-volátil e duradoura, menos sucetível à radiação & Inspeção \tabularnewline 

        \hline
        6 & O módulo OBDH deve possuir uma memória Flash para armazenar dados do satélite (preferencialmente com ECC) & Para armazenar dados & Inspeção \tabularnewline 

        \hline
       7 & O módulo OBDH deve possuir um WDT (\textit{Watchdog Timer}) para reiniciar o processador em caso de falha de \textit{software} & Reinicia automaticamente o processador caso haja a falha  & Teste \tabularnewline

        \hline
       8 & O módulo OBDH deve possuir sensores de medição de tensão e corrente em seus barramentos de alimentação & Para monitoramento de potência consumida & Inspeção \tabularnewline

        \hline
      9 &  O módulo OBDH deve possuir proteção de sobrecorrente (20\% acima do valor nominal) & Para proteção contra \textit{latch-up}  & Análise \tabularnewline

        \hline
        10 & O módulo OBDH deve possuir um giroscópio para medição de velocidade angular & Para permitir controle de atitude ativo do satélite  & Inspeção \tabularnewline 

       \hline
       11 & O módulo OBDH deve possuir um magnetômetro & Para permitir controle de atitude ativo do satélite  & Inspeção \tabularnewline

        \hline
        12 & O módulo OBDH deve possuir uma interface RS-422 para transmissão de mensagens de \textit{debug/log} e receber parâmetros de configuração & Comunicação de longa distância com maior imunidade ao ruído e maior taxa de dados  & Teste \tabularnewline

       \hline
       13 & O módulo OBDH deve possuir uma interface CAN para receber e transmitir comandos e dados & Para comunicação robusta e com suporte a múltiplos subsistemas do CubeSat  & Inspeção \tabularnewline

        \hline
        14 & O módulo OBDH deve possuir uma interface acessível externamente para programação do microcontrolador & Para o módulo ser facilmente programado pelo time  & Inspeção \tabularnewline

        \hline
        15 & O módulo OBDH deve possuir uma interface para um barramento de expansão & Para prover suporte a outras interfaces e periféricos  & Inspeção \tabularnewline

        \hline
       16 & O módulo OBDH deve possuir um sensor de temperatura com precisão menor ou igual a 1°C & Para monitorar a temperatura de operação & Inspeção\tabularnewline

        \hline
       17 & O módulo OBDH deve possuir uma interface RS-485 para receber e transmitir comandos e dados  & Para comunicação robusta e com suporte a múltiplos subsistemas do CubeSat & Inspeção \tabularnewline
       \hline
\end{longtable}
{\fonte{Elaboração própria.}}

Com as definições apresentadas na Tabela \ref{tab:Tab_Req}, foi então necessária a definição da arquitetura do hardware, ou seja, os componentes e sua interconexões, bem como as interfaces de comunicação e saídas necessárias.

\section{Arquitetura Proposta}

A partir dos requisitos, o primeiro passo foi definir de forma geral como seria o funcionamento do \textit{hardware} do projeto. Na Figura \ref{fig:arq_geral}, pode-se verificar um esquema inicial de proposta de arquitetura, usando os pontos descritos anteriormente.

\begin{figure}[H]
    \centering
    \includegraphics[scale=1.2]{images/arquitetura geral.png}
    \caption{Esquema geral de arquitetura.}
    \label{fig:arq_geral}
    \fonte{Elaboração própria.}
\end{figure}

Como podemos verificar, o processador será crucial e deverá ter pinos suficientes para interface com todas as memórias, sensores e para se comunicar com os outros módulos do CubeSat. Além disso, a parte dos circuitos dedicados às tensões utilizadas deverá ser cuidadosamente estudada, para que seja corretamente dimensionado de acordo com o consumo de potência estimado. A escolha de cada componente será descrita nas seções a seguir, respeitando sempre os seguintes critérios:

\begin{itemize}
	\item O componente deve funcionar corretamente nas temperaturas entre -40°C e 85°C;
	\item Circuitos integrados devem possuir herança de voo sempre que possível;
	\item Caso o circuito integrado necessite de um circuito específico, o mesmo deve conter itens preferencialmente dispostos na ECSS-Q-ST-60C, na NPSL ou similar aos mesmos, especialmente componentes discretos (capacitores, resistores, indutores, diodos, transistores, entre outros);
\end{itemize}

\subsection{Microcontrolador}

Como visto na Tabela \ref{tab:Tab_Missoes}, a fabricante com maior herança de voo estudada é a Xilinx, em especial os chips da família Zynq 7000, que são SoCs. Após um estudo próprio, o SoC Zynq 7030 se mostrou mais adequado pelas seguintes características:

\begin{itemize}
	\item Foi usado em missões extensivas em pequenos satélites (GomSpace, 2024), ou seja, possui herança de voo em missões similares em LEO e em CubeSats;
	\item Possui um envelopamento com 484 pinos, suficiente para prover as conexões necessárias para todas as interfaces requeridas (UG865, 2021);
	\item Capaz de rodar um sistema Linux (KADI et al.,2013);
	\item Por ser um SoC, possui alta adaptabilidade e flexibilidade, disponibilizando no mesmo chip uma FPGA e um microprocessador, denominados respectivamente de PL (\textit{Programmable Logic}) e PS (\textit{Processing System});
\end{itemize}

\subsection{Memórias}

As memórias serão necessárias para realizar operações, armazenar dados externos e internos e armazenar parâmetros de configuração do OBDH e de outros subsistemas do CubeSat. Para cada uma dessas funções uma memória diferente é necessária, seguindo suas características principais, sendo elas: tempo de acesso, tamanho do armazenamento e volatilidade.

\subsubsection{Memórias voláteis}

Partindo dos requisitos de projeto, bem como do esforço de se obter um \textit{hardware} capaz de rodar um sistema Linux, a principal opção se tornou as memórias do tipo DDR (\textit{Double Data Rate}), que utilizam ambas a borda de subida e de descida para transferência de dados para atingir o dobro de largura de banda de uma memória com SDR (\textit{Single Data Rate}) para uma mesma frequência de relógio (JEDEC, 2008). Essa relação pode ser ilustrada pela Figura \ref{fig:sdrvsddr}, onde pode-se verificar a transferência de dados do sinal DQ em relação ao sinal de relógio (bCLK e CLK) para SDR e DDR.

\begin{figure}[H]
    \centering
    \includegraphics[scale=1]{images/ddrsdr.png}
    \caption{Comparação entre DDR e SDR.}
    \label{fig:sdrvsddr}
    \fonte{KLEHN E BROX, 2003.}
\end{figure}
 
Por essa razão, foi escolhida uma memória do tipo DDR3, com capacidade de 2Gb e frequência de operação de 800 MHz.

\subsubsection{Memórias não voláteis}

No caso das memórias não voláteis, é necessária uma atenção especial ao tipo de dado que será armazenado em cada uma delas. Para o caso de dados críticos, é preciso de uma memória que possua alta resistência aos efeitos da radiação, mantendo-se um compromisso com os tempos de escrita e leitura. Por sua vez, para dados de inicialização são mais críticos os tempos de leitura, enquanto para uma memória de dados mais gerais, o importante é o armazenamento total. Por meio desses critérios, foi possível avaliar, por meio da Tabela \ref{tab:memnvol}, o tipo de memória ideal para cada caso, considerando o número máximo de ciclos de leitura e escrita de cada tipo de memória.

\begin{table}[H]
	\ABNTEXfontereduzida
	\caption{\label{tab:memnvol}Tabela comparativa de memórias não voláteis.}
	%\begin{tabular}{@{}p{2cm}p{2cm}p{2cm}p{2cm}p{2cm}p{2cm}p{3cm}@{}}
    \centering
    \begin{tabular}{@{} >{\centering}p{2cm} >{\centering}p{3cm} >{\centering}p{3cm} >{\centering}p{3cm}>{\centering}p{3cm} @{}}
    
		\toprule
		\textbf{Memória} & \textbf{Tempo de leitura/escrita} & \textbf{Tolerância à radiação} & \textbf{Armazenamento máximo} & \textbf{Ciclos de escrita / apagamento} \tabularnewline 
        \midrule
        Flash NOR & \textasciitilde{} 1$\mu$s & Ruim & \textasciitilde{} 1 Gb & \textbf{$10^5$} \tabularnewline
        
        \midrule
        Flash NAND & \textasciitilde{} 100 $\mu$s & Ruim &\textasciitilde{} 1 Tb & \textbf{$10^5$} \tabularnewline 

        \midrule
        FRAM & \textasciitilde{} 50 ns & Boa & \textasciitilde{} 1 Mb & \textbf{$10^{15}$}  \tabularnewline 
        
        \bottomrule
	\end{tabular}
	\fonte{Elaboração própria com base em (GERARDIN E PACCAGNELLA, 2010) e (BOUKHOBZA E OLIVIER, 2017).}
\end{table}

Com isso, foi então escolhida uma FRAM para armazenar dados críticos, uma Flash NAND para armazenamento de dados gerais e uma Flash NOR para armazenar o boot do sistema operacional no SoC.

\subsection{Conversores DC-DC}
Nos sistemas CubeSat do SpaceLab da UFSC, o módulo responsável pelo fornecimento de potência é o chamado EPS (MARCELINO, 2024). A partir disso, partindo do pressuposto que haverá uma tensão fornecida de 3,3 V, pode-se inferir a cascata dos barramentos de alimentação a partir do mesmo. Para o caso do Zynq e da memória DDR3, circuitos integrados são necessários para gerar as seguintes tensões: 

\begin{itemize}
	\item Zynq: 1 V e 1,8 V; 
	\item DDR3L: 1,35 V e 0,675 V.
\end{itemize}

Todos os demais periféricos devem aceitar uma tensão de alimentação de 3,3 V. Outro ponto importante são os circuitos de proteção contra \textit{latch-up}, um efeito similar a um curto-circuito na trilha de alimentação de circuitos CMOS (AN-600, 1989). Essa proteção é essencial, pois ao ocorrer, gera um consumo elevado de corrente e consequentemente tem potencial de gerar efeitos e falhas catastróficas (ECSS, 2018). No caso desse projeto, será utilizado o LTC4361, anteriormente usado em outras PCBs do SpaceLab, como o Payload HARSH (MATTOS, 2020). 

Na Figura \ref{fig:diapower}, está esquematizado o sistema de potência proposto.

\begin{figure}[H]
    \centering
    \includegraphics[scale=1]{images/diapower.png}
    \caption{Sistema de potência proposto.}
    \label{fig:diapower}
    \fonte{Elaboração própria.}
\end{figure}

\subsection{Sensores e Periféricos}
Como dito nos requisitos, alguns sensores precisam estar presentes no OBDH. Entre eles:

\begin{itemize}
	\item Um monitor de tensão, para todas os barramentos de alimentação importantes do sistema;
	\item Um sensor de corrente para o barramento de alimentação principal do módulo;
	\item Um giroscópio para medir a velocidade angular em órbita;
	\item Um magnetômetro para medição do campo magnético da Terra em órbita;
	\item Um WDT para reiniciar o sistema em caso de falha de software;
	\item Um sensor de temperatura para o SoC e um para a região das memórias na PCB;
\end{itemize}

\section{Visualização da Arquitetura Proposta}

Depois das decisões tomadas, foi possível montar um diagrama, apresentado na Figura \ref{fig:arq}, que mostra cada circuito do computador de bordo. Aqui, por simplicidade, foram suprimidos os circuitos da parte de potência do módulo.

\begin{figure}[H]
    \centering
    \includegraphics[scale=1]{images/arquitetura final.png}
    \caption{Arquitetura proposta para o OBDH.}
    \label{fig:arq}
    \fonte{Elaboração própria.}
\end{figure}

Também foram levadas em consideração as interfaces disponibilizadas pelo SoC, os componentes escolhidos e os conectores necessários. Os componentes escolhidos se encontram na Tabela \ref{tab:componentes}, conjuntamente com as interfaces requeridas para cada um, suas tensões de alimentação e suas correntes máximas no terminal de alimentação, no pior caso especificado pelo fabricante.

\begin{table}[H]
	\ABNTEXfontereduzida
	\caption{\label{tab:componentes}Informações sobre os componentes escolhidos.}
	%\begin{tabular}{@{}p{2cm}p{2cm}p{2cm}p{2cm}p{2cm}p{2cm}p{3cm}@{}}
    \centering
    \begin{tabular}{@{} >{\centering}p{2cm} >{\centering}p{4cm} >{\centering}p{2cm} >{\centering}p{3cm}>{\centering}p{3cm} @{}}
    
		\toprule
		\textbf{Componente} & \textbf{Número do Fabricante} & \textbf{Interface} & \textbf{Tensão de Alimentação} & \textbf{Corrente máxima} \tabularnewline 
        \midrule
        FRAM & CY15B104QN-50SXI & SPI & 3,3 V & 3,7 mA \tabularnewline
        
        \midrule
        Flash NOR & MT25QL128ABB1ESE-0AUT & QSPI & 3,3 V & 55 mA \tabularnewline 

        \midrule
        Flash NAND & MT29F1G01ABAFDSF-AAT:F & SPI & 3,3 V & 35 mA \tabularnewline 

        \midrule
        DDR3L & MT41K256M8DA-125:K & Paralela & 1,35 V & 182 mA \tabularnewline 

        \midrule
        WDT & TPS3823-33QDBVRQ1 & - & 3,3 V & 10 mA \tabularnewline 

        \midrule
        Monitor de Temperatura e Tensão & LTC2991IMS\#TRPBF & I2C & 3,3 V &  1,5 mA \tabularnewline 

        \midrule
        Sensor de Corrente & INA180A2IDBVR & - & 3,3 V & 1 mA \tabularnewline 

        \midrule
        Giroscópio & A3G4250D & I2C & 3,3 V & 7 mA \tabularnewline 

        \midrule
        Magnetômetro & MMC5983MA & I2C & 3,3 V & 0,45 mA \tabularnewline 

        \midrule
        Buffer I2C & TCA4311ADR & I2C & 3,3 V & 7 mA \tabularnewline 

        \midrule
        Transceptor CAN & A3G4250D & CAN & 3,3 V & 60 mA \tabularnewline 

        \midrule
        Transceptor RS-485 & THVD1451DR & Serial & 3,3 V & 3 mA \tabularnewline 

        \midrule
        Conversor DC-DC & TPS82085SILR & - & 3,3 V & - \tabularnewline 

        \midrule
        Conversor DC-DC para DDR3 & TPS51200DRCR & - & 3,3 V & 1 mA \tabularnewline 

        \midrule
        \textit{Load Switch} & TPS22920YZPR & - & 3,3 V & 0,2 mA \tabularnewline 
        
        \midrule
        Proteção contra \textit{Latch-up} & LTC4361 & - & - & - \tabularnewline 

        \bottomrule
	\end{tabular}
	\fonte{Elaboração própria com base nos Datasheets de cada componente.}
\end{table}

\subsection{Estimativa de Potência Consumida}

A fim de garantir o funcionamento correto dos conversores DC-DC e seus respectivos periféricos, foi necessária uma estimativa da potência total consumida por todas as tensões disponíveis no módulo. Para isso, foi utilizada a Tabela \ref{tab:componentes}, bem como o datasheet de cada componente. No caso do SoC, sua fabricante disponibiliza uma planilha (XPE, 2019) para estimativas de potência em cada tensão de alimentação. 

Com isso, foram obtidos os valores da Tabela \ref{tab:estpow}, considerando uma eficiência de conversão de 85\% (TPS82085, 2019), já incluindo as estimativas de potência e os piores casos descritos anteriormente.

\begin{table}[H]
	\ABNTEXfontereduzida
	\caption{\label{tab:estpow}Estimativas de potência consumida.}
	%\begin{tabular}{@{}p{2cm}p{2cm}p{2cm}p{2cm}p{2cm}p{2cm}p{3cm}@{}}
    \centering
    \begin{tabular}{@{} >{\centering}p{2cm} >{\centering}p{4cm} >{\centering}p{4cm} >{\centering}p{4cm}@{}}
    
		\toprule
		\textbf{Tensão [V]} & \textbf{Potência Dissipada [W]} & \textbf{Potência dissipada na tensão de 3,3 V [W]} & \textbf{Corrente máxima da trilha [A]} \tabularnewline 
        \midrule
         1,00 & 2,20 & 2,59 & 2,20 \tabularnewline
        
        \midrule
        1,35 & 0,25 & 0,29 & 0,18 \tabularnewline 

        \midrule
        1,80 & 0,63 & 0,74 & 0,35 \tabularnewline

        \midrule
        3,3 & 7,5 & - & 2,27  \tabularnewline        

        \bottomrule
	\end{tabular}
	\fonte{Elaboração própria.}
\end{table}

Através dessas estimativas, pode-se confirmar que o sistema de potência proposto suporta os componentes escolhidos e suas tensões e variações, mesmo quando se considera o pior caso.
\include{cap-projeto.tex}
% ----------------------------------------------------------
\chapter{Resultados}
% ----------------------------------------------------------
O desenvolvimento deste trabalho teve como principal objetivo a criação de uma arquitetura de hardware robusta e versátil para um computador de bordo destinado a pequenos satélites, especificamente CubeSats. Este computador de bordo foi projetado para operar em ambientes espaciais adversos, assegurando a integridade e a confiabilidade no tratamento de dados, além de possibilitar a adaptação a diferentes tipos de missões e experimentos científicos em órbita.

Dentre os objetivos específicos, estavam a análise dos requisitos de versatilidade para ambientes espaciais, a especificação de uma arquitetura adaptável e a documentação detalhada de todas as decisões de projeto. Para atender esses objetivos, inicialmente, a robustez do sistema foi trabalhada com a seleção cuidadosa de componentes eletrônicos. Tais componentes foram escolhidos de acordo com diretrizes de herança de voo e normas estabelecidas pela ESA e NASA, de forma a garantir maior confiabilidade e segurança operacional.

No aspecto da versatilidade, o sistema foi arquitetado de modo a integrar memórias, sensores e periféricos variados, de modo a atender a diferentes tipos de missões. Esse objetivo foi cumprido por meio do uso de um SoC da família Zynq, que incorpora um microprocessador e um FPGA. As interfaces genéricas disponibilizadas para os conectores (SPI, I2C, UART, CAN e diversos pinos de entrada e saída genéricos), também corroboraram para essa versatilidade, permitindo a interconexão e adaptação a módulos e subsistemas diversos.

A abordagem de modularidade do projeto também reforça sua robustez e flexibilidade. Com o uso de memórias não voláteis (Flash NOR, Flash NAND e FRAM) para o armazenamento de dados críticos e a inicialização segura do sistema, o computador de bordo projetado consegue resistir às adversidades do ambiente espacial. Além disso, cada componente foi avaliado quanto ao consumo energético e às exigências de funcionamento entre -40 e 85 °C.

A arquitetura foi definida conforme a Figura \ref{fig:arq}, respeitando os requisitos levantados e projetada como mostra a Figura \ref{fig:inter}. Para fins de comparação, a Tabela \ref{tab:results} mostra uma comparação entre o OBDH projetado e alguns dos OBDHs revisados, mostrando que as principais características estão de acordo com o estado da arte investigado.

\begin{table}
\tiny
	\caption{\label{tab:results}Comparação entre o OBDH proposto e o estado da arte.}
	%\begin{tabular}{@{}p{2cm}p{2cm}p{2cm}p{2cm}p{2cm}p{2cm}p{3cm}@{}}
    \centering
    \begin{tabular}{@{} >{\centering}p{2cm}>{\centering}p{3cm}>{\centering}p{3cm}>{\centering}p{2cm}>{\centering}p{2cm}>{\centering}p{2cm}@{}}
    
		\toprule
		\textbf{Nome} & \textbf{Memória RAM} & \textbf{Memória não volátil} & \textbf{Sensores} & \textbf{Dimensões} & \textbf{Massa} \tabularnewline 
        \midrule
         OBDH proposto & DDR3L (256 Gbit) & FRAM (8 Mbit), Flash NOR (128 Mbit), Flash NAND (1 Gbit). & Tensão, Corrente, Temperatura, Giroscópio, Magnetômetro & TBD & TBD   \tabularnewline
        \midrule
         OBDH2 (MARCELINO et al., 2024) & SRAM (64 kB) & FRAM (2 Mbit), Flash NOR (128 MB). & Tensão, Corrente, Temperatura & 89,15 x 92,13 x 15 mm & 53 g\tabularnewline
        \midrule
         (PUTRA, 2018) & SRAM (128 kB) & Flash NOR (512 kB) & Tensão, Acelerômetro, Giroscópio, Magnetômetro, Temperatura, Umidade & Não informado & Não informado\tabularnewline
        \midrule
         GomSpace Nanomind Z7000 & DDR3 (1 GB) & Flash NAND (32 GB), Flash NOR (64 MB). & Tensão, Corrente. & 65 x 40 x 6,5 mm & 105,5 g  \tabularnewline
        \bottomrule
	\end{tabular}
	\fonte{Elaboração própria.}
\end{table}

Através disso, é possível verificar que a estrutura proposta é condizente com os OBDHs estudados, demonstrando uma evolução clara quando comparado ao OBDH 1.0 e OBDH 2.0, principalmente no que tange as memórias, com a inclusão da DDR. Por fim, o último resultado é o esquemático desenvolvido, apresentado no Apêndice A. Nele estão dispostas todas conexões necessárias entre os circuitos apresentados no Capítulo 4, respeitando as decisões de projeto tomadas na definição da arquitetura. 


% ----------------------------------------------------------
\chapter{Conclusão}
% ----------------------------------------------------------

Conclui-se que o projeto realizado cumpre os objetivos propostos, apresentando uma solução robusta, segura e adaptável para a operação em CubeSats, respeitando os requisitos. A flexibilidade conferida pela arquitetura do SoC (com FPGA integrada), associada à robustez das interfaces de comunicação e ao uso de memórias para fuções específicas permite que o sistema seja facilmente adaptável a diferentes missões. A partir das diretrizes iniciais, os resultados planejados foram atingidos, validando a arquitetura desenvolvida e sua viabilidade para diferentes aplicações em missões usando o padrão CubeSat. 

Outro ponto importante é que o OBDH desenvolvido apresenta um potencial significativo para aplicações futuras tanto no SpaceLab quanto em outras instituições. O módulo ter sido projetado como adaptável o permite ser utilizado em missões variadas, com diferentes requisitos e subsistemas. Além disso, sendo um projeto aberto, outras instituições poderão adotar essa solução conforme suas necessidades específicas, o que amplia o escopo de uso e promove um ambiente colaborativo de desenvolvimento de tecnologia espacial. No entanto, esse trabalho é apenas o início, com aprimoramentos e implementações a serem realizados em trabalhos futuros.

\section{Trabalhos Futuros}

Para dar continuidade ao desenvolvimento deste OBDH, ainda restam etapas importantes. A primeira delas é o desenvolvimento do layout da placa de circuito impresso, seguindo as diretrizes da ESA (ECSS, 2014). Depois disso, devem ser conduzidos os testes dos circuitos, visando validar o funcionamento e observar possíveis pontos de melhoria. Outra expectativa é o desenvolvimento do \textit{firmware} do sistema, para que o SoC atue no controle das operações e na comunicação com periféricos, sensores e subsistemas do CubeSat, usando um sistema Linux. Além disso, deve-se desenvolver a parte mecânica requerida, com a conexão entre os submódulos e a estrutura metálica.

Por fim, a proteção contra TID também é uma continuação importante e interessante. Um estudo sobre o uso de \textit{shields} de proteção, a fim de atenuar esse efeito sobre os componentes e sobre a placa como um todo seria de grande valia, visto que isso melhoraria a durabilidade do OBDH e aumentaria a resistência do sistema aos efeitos da radiação em LEO.

\begin{flushleft}
\begin{thebibliography}{00}
\bibitem{b1} AAC Clyde Space. Datasheet: Kryten-M3. Disponível em $<$https://www.aac-clyde.space/wp-content/uploads/2021/10/AAC\_DataSheet\_Kryten.pdf$>$.  Acesso em: 07 de junho, 2024.

\bibitem{b2} CAPPELLETTI, C.; BATTISTINI, S.; MALPHRUS, B. Cubesat Handbook: From Mission Design to Operations. Editora Elsevier, 2021.

\bibitem{b3} GEORGE, A. D.; WILSON, C. M. Onboard processing with hybrid and reconfigurable computing on small satellites. Proceedings of the IEEE. Institute of Electrical and Electronics Engineers, 2018.

\bibitem{b4} GomSpace. Datasheet: NanoMind A3200. Disponível em $<$https://gomspace.com/UserFiles/Subsystems/datasheet/gs-ds-nanomind-a3200\_1006901-117.pdf$>$. Acesso em: 07 de junho, 2024.

\bibitem{b5} GomSpace. Datasheet: NanoMind HP MK3. Disponível em $<$https://gomspace.com/UserFiles/Subsystems/datasheet/gs-ds-NanoMind\_HP\_MK3.pdf$>$. Acesso em: 07 de junho, 2024.

\bibitem{b6} ISIS Space On Board Computer. Disponível em $<$https://www.isispace.nl/product/on-board-computer/$>$. Acesso em: 07 de junho, 2024.

\bibitem{b7} Nano Avionics. CubeSat On-Board Computer – Main Bus Unit SatBus 3C2. Disponível em $<$https://nanoavionics.com/cubesat-components/cubesat-on-board-computer-main-bus-unit-satbus-3c2/$>$. Acesso em: 07 de junho, 2024.

\bibitem{b8} MARCELINO, G. M. et al. A critical embedded system challenge: The FloripaSat-1 mission. IEEE Latin America Transactions, 2020.

\bibitem{b9} MARCELINO, G. M. et al. FloripaSat-2: An Open-Source Platform for CubeSats. IEEE embedded systems letters, 2024.
\end{thebibliography}
\end{flushleft}


\postextual
%\begin{flushleft}
\begin{thebibliography}{00}
\bibitem{b1} AAC Clyde Space. Datasheet: Kryten-M3. Disponível em $<$https://www.aac-clyde.space/wp-content/uploads/2021/10/AAC\_DataSheet\_Kryten.pdf$>$.  Acesso em: 07 de junho, 2024.

\bibitem{b2} CAPPELLETTI, C.; BATTISTINI, S.; MALPHRUS, B. Cubesat Handbook: From Mission Design to Operations. Editora Elsevier, 2021.

\bibitem{b3} GEORGE, A. D.; WILSON, C. M. Onboard processing with hybrid and reconfigurable computing on small satellites. Proceedings of the IEEE. Institute of Electrical and Electronics Engineers, 2018.

\bibitem{b4} GomSpace. Datasheet: NanoMind A3200. Disponível em $<$https://gomspace.com/UserFiles/Subsystems/datasheet/gs-ds-nanomind-a3200\_1006901-117.pdf$>$. Acesso em: 07 de junho, 2024.

\bibitem{b5} GomSpace. Datasheet: NanoMind HP MK3. Disponível em $<$https://gomspace.com/UserFiles/Subsystems/datasheet/gs-ds-NanoMind\_HP\_MK3.pdf$>$. Acesso em: 07 de junho, 2024.

\bibitem{b6} ISIS Space On Board Computer. Disponível em $<$https://www.isispace.nl/product/on-board-computer/$>$. Acesso em: 07 de junho, 2024.

\bibitem{b7} Nano Avionics. CubeSat On-Board Computer – Main Bus Unit SatBus 3C2. Disponível em $<$https://nanoavionics.com/cubesat-components/cubesat-on-board-computer-main-bus-unit-satbus-3c2/$>$. Acesso em: 07 de junho, 2024.

\bibitem{b8} MARCELINO, G. M. et al. A critical embedded system challenge: The FloripaSat-1 mission. IEEE Latin America Transactions, 2020.

\bibitem{b9} MARCELINO, G. M. et al. FloripaSat-2: An Open-Source Platform for CubeSats. IEEE embedded systems letters, 2024.
\end{thebibliography}
\end{flushleft}


%\include{ref.bib}

\begin{apendicesenv}

% ----------------------------------------------------------
\chapter{Esquemático completo}
% ----------------------------------------------------------



\end{apendicesenv}
% ---
\end{document}