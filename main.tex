\documentclass[
12pt,				% tamanho da fonte
%openright,		% capítulos começam em pág ímpar (insere página vazia caso preciso)
oneside,			% para impressão no anverso. Oposto a twoside
a4paper,			% tamanho do papel. 
chapter=TITLE,		% títulos de capítulos convertidos em letras maiúsculas
section=TITLE,		% títulos de seções convertidos em letras maiúsculas
%subsection=TITLE,	% títulos de subseções convertidos em letras maiúsculas
%subsubsection=TITLE,% títulos de subsubseções convertidos em letras maiúsculas
% -- opções do pacote babel --
english,			% idioma adicional para hifenização
brazil				% o último idioma é o principal do documento
hyperref=hidelinks]{abntex2}

\usepackage{tccctj} % carrega o estilo CTJ
\bibliographystyle{abntex2-alf-ufsc} % Arquivo bst com patch para correção de citação de proceedings. Produz italico em "In", conforme descrito em: https://github.com/abntex/abntex2/issues/226

% Useful packages
\usepackage{amsmath}
\usepackage{graphicx}
% Pacotes adicionais
\usepackage{multicol}
\usepackage{multirow}
\usepackage{tabularx}
\usepackage{quoting}
\quotingsetup{indentfirst={false},font={footnotesize},leftmargin=4cm,rightmargin=0cm} 
\hypersetup{hidelinks}
\usepackage{enumitem}
\setlist{noitemsep}
%\usepackage{hyperref}
%\usepackage{array,epsfig}
%\usepackage{amsfonts}
%\usepackage{amssymb}
%\usepackage{amsxtra}
%\usepackage{amsthm}
%\usepackage{mathrsfs}
%\usepackage{color}
%\usepackage{indentfirst}
\usepackage{float}
\usepackage{booktabs}
\usepackage{array}
\usepackage{longtable}
\usepackage[nobiblatex]{xurl}

% ----------------------------------------------------

% ----------------------------------------------------
% Informações do trabalho
\autor{Yuri Gazzoni Rezende}
\titulo{OBDH Robusto Para Uso em LEO}
% \subtitulo{Subtítulo} % Preenchimento opcional para quando houver sub-título
\orientador{Gabriel Marcelino}
%\coorientador{Nome do coorientador} % Preenchimento opcional para quando houver co-orientador
\curso{Departamento de Engenharia Elétrica e Eletrônica}
% \titulacao{Bacharel em x}
\datadadefesa{15}{novembro}{2024}
\local{Florianópolis}

\begin{document}

% ----------------------------------------------------
% 1. Capa do trabalho
\imprimircapa

% 2. Folha de rosto
\imprimirfolhaderosto

% 3. Folha de aprovação
\begin{folhadeaprovacao}
    \membrodabanca[Orientador/Presidente]{Prof. Dr. Nome do Orientador/Presidente}{}
    \membrodabanca{Prof. Dr. Membro da banca 1}{UFSC}
    \membrodabanca{Prof. Dr. Membro da banca 2}{UFSC}
    \membrodabanca{Prof. Dr. Membro da banca 3}{UFSC\,-\,Florianópolis}
\end{folhadeaprovacao}

% 4. Dedicatória
\begin{dedicatoria}
    Este trabalho é dedicado aos meus colegas de classe e aos meus queridos pais.
\end{dedicatoria}

% 5. Agradecimentos 
\begin{agradecimentos}
    Inserir os agradecimentos aos colaboradores à execução do trabalho.
\end{agradecimentos}

% 6. Epígrafe
%\begin{epigrafe}
%    \aspas
%    A natureza é um enorme jogo de xadrez disputado por deuses, e que temos o privilégio de observar.
%    As regras do jogo são o que chamamos de física fundamental, e compreender essas regras é a nossa meta.
%    \aspas
%    \autor{Richard Phillips Feynman} % autor da epígrafe
%\end{epigrafe}

% 7. Resumo em português
\begin{resumo}
    O texto do resumo deve ser digitado em um único bloco, sem espaço de parágrafo. Deve ser composto por uma sequência de frases concisas, afirmativas e não de uma enumeração de tópicos. Não deve conter citações. Manter o tempo verbal do texto do trabalho (impessoal) e por vezes usar a voz ativa (Ex.: este trabalho apresenta). O Resumo deve conter: tema, problema, justificativa, objetivos, método e resultados (de forma geral). Abaixo do resumo, informar as palavras-chave (palavras ou expressões significativas retiradas do texto) e preferencialmente não repetir termos do título (para aumentar os indexadores). No mínimo três e no máximo cinco. Separadas por ponto. Observe que o espaçamento aqui, entre linhas, é simples (1,0).

    \palavrachave{Palavra-chave 1}
    \palavrachave{Palavra-chave 2}
    \palavrachave{Palavra-chave 3}
\end{resumo}

% 8a. Resumo em inglês
\begin{resumo}[Abstract]
    \begin{otherlanguage*}{english}
        Resumo traduzido para outros idiomas, neste caso, inglês. Segue o formato do resumo feito na língua vernácula. As palavras-chave traduzidas, versão em língua estrangeira, são colocadas abaixo do texto precedidas pela expressão \emph{Keywords}, separadas por ponto. Observe que o espaçamento aqui, entre linhas, é simples (1,0).
    \end{otherlanguage*}

    \palavrachave{First keyword}
    \palavrachave{Second keyword}
    \palavrachave{Third keyword}
\end{resumo}
% 9. Lista de figuras
\imprimirlistafiguras

% 10. Lista de quadros
%\imprimirlistaquadros

% 11. Lista de tabelas
\imprimirlistatabelas

% 12. Lista de abreviaturas e siglas
\begin{siglas}
	\item[OBDH] \textit{On-board Data Handling} 
\end{siglas}

% 13. Lista de símbolos 

% 14. Sumário
\imprimirsumario

% ----------------------------------------------------

% Elementos textuais
\textual


\chapter{Introdução}
\label{Chap:intro}

CubeSats são pequenos satélites que atendem a estritas formas de cubos padronizados de 10 cm de aresta, além de pesarem menos de 300 kg. Cada um desses cubos padronizados recebe a denominação de 1U, e os tamanhos subsequentes de 1,5U, 2U, 3U, e assim por diante (CUBESAT Design Specification, 2022). Devido a essa padronização e ao uso de componentes comerciais, os CubeSats podem ser produzidos em massa, o que diminui substancialmente os custos de lançamento e desenvolvimento (CubeSat 101, 2017).

O desenvolvimento de satélites de pequeno porte, como CubeSats e nanosatélites, trouxe novas oportunidades e desafios para a indústria espacial, permitindo que uma ampla gama de missões científicas, comerciais e educacionais fosse realizada com custos reduzidos e prazos de desenvolvimento mais curtos (CUBESAT Design Specification, 2022). No entanto, a miniaturização e a operação em ambientes espaciais impõem requisitos à robustez, confiabilidade e versatilidade dos sistemas embarcados, especialmente para os módulos OBDHs (\textit{On-Board Data Handling}). Nesse contexto, a arquitetura de um computador de bordo eficiente e robusto é essencial para o gerenciamento seguro das operações e para garantir a integridade das missões.

Essa segurança e integridade são pontos chave no desenvolvimento de CubeSats no SpaceLab, laboratório da UFSC especializado em desenvolvimento de sistemas espaciais para a comunidade científica e para a indústria. Um dos objetivos primários do SpaceLab é o desenvolvimento de uma plataforma \textit{open-source}, tanto para \textit{software} quanto \textit{hardware}, o que já foi feito nos desenvolvimentos do FloripaSat-1 (MARCELINO et al., 2020) e FloripaSat-2 (MARCELINO et al., 2024). Com esse paradigma, a oportunidade de se ter um computador de bordo mais robusto (com mais memória e capacidade de processamento) e versátil surgiu, como uma consequência direta dos desenvolvimentos das gerações anteriores de \textit{hardware} do laboratório.

O sistema desenvolvido para o presente trabalho foca na implementação de uma arquitetura de processamento e memória capaz de atender às demandas de um satélite de pequeno porte. Esse sistema deve operar de maneira confiável em ambientes suscetíveis à radiação e alta variação de temperatura, em conjunto com a otimização do uso de energia. Além disso, a versatilidade do computador de bordo é essencial para adaptar o sistema a diferentes tipos de missões, desde operações de imagem e telemetria até experimentos científicos em órbita. Para isso, é necessário que o sistema ofereça uma arquitetura versátil, baseada em uma FPGA (\textit{Field-Programmable Gate Array}), com capacidade de expansão e adaptação a novos sensores e módulos de comunicação.

Inicialmente, será feita uma revisão bibliográfica, explorando as características de computadores de bordo comerciais e de trabalhos acadêmicos, além de entender como a radiação em órbita baixa afeta os sistemas eletrônicos. Depois disso, será definida uma arquitetura, respeitando os requisitos impostos, em conjunto com uma estimativa de consumo de potência. Com a arquitetura, será desenvolvido um esquemático, usando o software Altium Designer (versão 24.4.1). Por fim, serão apresentados os resultados obtidos e as considerações finais e conclusões para esse projeto.

\section{Objetivo geral}

O presente trabalho tem como objetivo o projetar e implementação de uma arquitetura de hardware robusta e versátil para um computador de bordo de satélite de pequeno porte, integrando diferentes tipos de memórias e periféricos para assegurar a operação confiável em ambientes espaciais adversos, garantindo a integridade dos dados e a eficiência no gerenciamento dos mesmos.

\section{Objetivos Específicos}

\begin{itemize}
    \item Analisar os requisitos de robustez em condições espaciais, com foco em resistência a radiação, tolerância a falhas e estabilidade térmica, a fim de assegurar o funcionamento contínuo do computador de bordo em órbita.
    \item Especificar uma arquitetura de hardware que permita a adaptação a diferentes tipos de missões, integrando diferentes tipos de componentes comerciais com um SoC (\textit{System-on-a-chip}).
    \item  Documentar as decisões de projeto para consolidar um guia técnico com recomendações de design para sistemas de robustos e versáteis aplicáveis a satélites de pequeno porte, contribuindo para futuras otimizações e adaptações em missões espaciais.
\end{itemize}

% ----------------------------------------------------------
\chapter{Revisão Bibliográfica}
% ----------------------------------------------------------

Para atingir o objetivo de projetar o \textit{hardware} de um computador de bordo robusto, foi preciso buscar na literatura acadêmica o estado da arte que tange o projeto de OBDHs para satélites de pequeno porte, especialmente para CubeSats.
 
Primeiro, foi necessário um estudo sobre a radiação em LEO - \textit{Low Earth Orbit} (órbitas com raio menor que 1000 km, segundo ESA, 2024), para que a escolha dos componentes do projeto seja a melhor possível. Com esse estudo, buscaram-se formas de mitigar os efeitos mais conhecidos e verificar como instituições têm lidado com componentes do tipo \textit{commercial-off-the-shelf} (COTS).% Além disso, também foi dada a fundamentação dos conversores de potência, tipos de memória, microprocessadores e interfaces de comunicação.

Depois, foram analisadas as placas de OBDH dos projetos do FloripaSat-1 e FloripaSat-2, desenvolvidas pelo SpaceLab da UFSC. Por fim, outros projetos comerciais foram estudados para obtenção de noções sobre a arquitetura e componentes usados. Um panorama geral foi feito, verificando-se principalmente os componentes principais e mais críticos, ou seja, processadores, memórias voláteis e não-voláteis e outros periféricos.

\section{Radiação em LEO e Componentes COTS}

Estando em solo terrestre, os eletrônicos atuais estão bem protegidos contra a maior parte da radiação incidente do universo. No caso dos satélites orbitais, a proteção atmosférica é atenuada pela distância em relação ao solo, mesmo para aqueles que operam em LEO. Nesse caso, a radiação pode ser suficientemente significativa para causar a mudança do comportamento eletromagnético dos materiais, causando efeitos como falhas, aquisição ou execução errada de comandos e distorções dos sinais (MAYANBARI, 2011) (LABEL, 2004).  Esses danos são divididos em dois grupos (JUNQUEIRA, 2020): os acumulativos como o TID (\textit{Total Ionizing Dose}), e os SEE (\textit{Single Event Effects}), que indicam o acontecimento de eventos únicos. 

Ainda segundo Junqueira (2020), o TID se caracteriza principalmente pela formação de pares elétron-lacuna, onde o primeiro aumenta a condutividade do material e o segundo contribui para oxidação, mudando as características elétricas do componente com o tempo.  Já os SEE ocorrem quando um íon atravessa um componente crítico, gerando uma linha de ionização que pode ou não ser destrutiva. 

Por esse motivo, quando são escolhidos os componentes críticos para o \textit{hardware} de um \textit{CubeSat}, em sua maioria COTS, deve-se levar em consideração algumas diretrizes cruciais. Segundo Carmo et al. (2021), o componente escolhido precisa atender os requisitos operacionais, concomitante ao gerenciamento de riscos com mitigações e blindagens. 

Com isso, é possível ver três formas confiáveis de escolher cada componente: usando as diretrizes da ESA (\textit{European Space Agency}), as da NASA (\textit{National Aeronautics and Space Administration}) e também através da herança de voo, ou seja, escolhendo componentes que já estiveram em missões semelhantes ou mais críticas. Nos dois primeiros casos, a consulta é através da norma ECSS-Q-ST-60C para a ESA e da lista NPSL (\textit{NASA Part Selection List}) para a NASA.  No caso da herança de voo, outros projetos devem ser analisados e consultados, o que será feito na seção a seguir.

%\section{Conversores de potência}

%Também chamados de conversores DC-DC, são uma peça crucial em um projeto de PCB. Através de uma tensão de entrada, conseguem convertê-la em tensões maiores ou menores, conforme a necessidade do projetista. Existem diferentes tipos de conversores DC-DC, e tratar-se-á apenas dos conversores chaveados \textit{step-up} e \textit{step-down} nessa seção. 

%\subsection{Conversores \textit{step-up}}

%\section{Memória}

%\section{Microprocessadores}

%\section{Interfaces de comunicação}

\section{Projetos Anteriores}

\subsection{FloripaSat-1}
% https://ieeexplore.ieee.org/abstract/document/9085277
O FloripaSat-1 (MARCELINO et al., 2020) é uma plataforma \textit{open-source} para nanossatélites, além de ser também o nome do primeiro satélite lançado pelo SpaceLab. O satélite FloripaSat-1 é um CubeSat 1U,  composto de três módulos: um módulo de fornecimento de potência (EPS), um computador de bordo (OBDH) e um módulo de telemetria e comunicação (TTC). Além disso, possuía duas cargas úteis que consistiam de placas com FPGAs. A missão tinha como objetivos a validação do satélite em órbita, tanto dos módulos desenvolvidos na UFSC quanto do módulo de comunicação desenvolvido no INPE e do módulo da FPGA tolerante a radiação.

Seu OBDH foi feito para realização da interface e comunicação entre os módulos e \textit{payloads}. Aqui, destacam-se os sensores presentes: uma \textit{Inertial Measurement Unit} (com giroscópio, magnetômetro e acelerômetro), a interface com os sensores dos painéis solares e as medições de tensão e corrente de entrada do próprio módulo.

Além disso, contava com um microprocessador de 16 bits, memórias flash (IS25LP128) e suporte para cartão microSD para armazenamento.

\subsection{FloripaSat-2}
% https://ieeexplore.ieee.org/abstract/document/10078027
O FloripaSat-2 é a segunda geração da plataforma \textit{open-source} desenvolvida pelo SpaceLab, baseando-se no projeto FloripaSat-1 e trazendo melhorias para os três módulos principais (MARCELINO et al., 2024). O diagrama de blocos do CubeSat proposto está disposto na Figura \ref{fig:floripasat2}, onde pode-se verificar as interfaces do OBDH com o restante do módulo.

\begin{figure}[H]
    \centering
    \includegraphics[scale=0.8]{images/floripasat2.png}
    \caption{Diagrama de blocos da plataforma FloripaSat-2.}
    \label{fig:floripasat2}
    \fonte{MARCELINO et al., 2024.}
\end{figure}

Especificamente para o OBDH, foram introduzidas uma memória FRAM (\textit{Ferroelectric Random-Access Memory}) e uma Flash NOR de maior capacidade de armazenamento, o que mostra uma melhoria clara de capacidade e confiabilidade. Outras duas melhorias importantes foram, primeiramente, a adição de um conector para eventualmente conectar uma \textit{daughter board} à placa, e, segundamente, a adição de \textit{buffers} às trilhas de I2C (\textit{Inter-Integrated Circuit}) entre os módulos. Isso acrescenta flexibilidade e confiabilidade ao OBDH da segunda geração.

\subsection{Projetos Comerciais}

Abaixo se encontram sintetizados os projetos comerciais estudados, para obtenção de noções sobre a arquitetura e componentes usados. Foram verificados principalmente os processadores, as memórias voláteis e não-voláteis, as interfaces de comunicação e outros periféricos (ADCs, RTC, etc.) utilizados. A Tabela \ref{tab:Tab_Rev} mostra a pesquisa realizada sobre o estado da arte, em conjunto com os dados de George e Wilson (2018), sintetizados na Tabela \ref{tab:Tab_Missoes}.

\begin{table}[htb]
    \centering
	\ABNTEXfontereduzida
	\caption{\label{tab:Tab_Rev}Comparação entre os principais modelos comerciais de OBDH disponíveis atualmente no mercado.}
	%\begin{tabular}{@{}p{2cm}p{2cm}p{2cm}p{2cm}p{2cm}p{2cm}p{3cm}@{}}
    \begin{tabular}{@{}p{2cm}p{2.6cm}p{2cm}p{2cm}p{2.2cm}p{2.6cm}@{}}
		\toprule
		\textbf{Fabricante} & \textbf{Nome do Produto} & \textbf{Processador} & \textbf{Memórias} & \textbf{Periféricos} & \textbf{Interfaces de comunicação} \\ 
        \midrule
        GomSpace & NanoMind A3200 & AT32UC3C & Flash, SDRAM, FRAM & Giroscópio, Magnetômetro, Transceivers, Sensores de temperatura & CAN, I2C, SPI, JTAG, USART \\%https://gomspace.com/UserFiles/Subsystems/datasheet/gs-ds-nanomind-a3200_1006901-117.pdf
        
        \midrule
        GomSpace & NanoMind HPMK3 & Zynq 7030 & Flash, eMMC, DDR3 & Watchdog, Sensores de temperatura, VCO, Sensores de tensão e corrente & CAN, USART, USB, I2C, JTAG, LVDS, SpaceWire \\ %https://gomspace.com/UserFiles/Subsystems/datasheet/gs-ds-NanoMind_HP_MK3.pdf

        \midrule
        ISIS Space & ISIS On Board Computer & Atmel & Flash, SDRAM, FRAM, Cartões SD & Sensores de temperatura, Sensores de tensão e corrente, RTC, ADC & USART, USB, I2C, JTAG, PWM \\ %https://www.isispace.nl/product/on-board-computer/

        \midrule
        Nano Avionics & SatBus 3C2 & Não informado & Flash, FRAM, Cartões SD & Giroscópio, Magnetômetro, Rádio UHF, ADC & CAN, SPI, I2C, USART, PWM, USB \\ %https://nanoavionics.com/cubesat-components/cubesat-on-board-computer-main-bus-unit-satbus-3c2/

        \midrule
        AAC Clyde Space & Kryten-M3 & Smart Fusion 2 SoC & MRAM, eNVM & RTC, Sensores de tensão e corrente & CAN, SPI, I2C, USART, RS422, LVDS \\ %https://www.aac-clyde.space/what-we-do/space-products-components/command-data-handling/kryten-m3       


		
        \\ \bottomrule
	\end{tabular}
	\fonte{Elaboração própria.}
\end{table}

\begin{table}[H]
	\ABNTEXfontereduzida
	\caption{\label{tab:Tab_Missoes}Síntese da tabela apresentada por George e Wilson (2018).}
	%\begin{tabular}{@{}p{2cm}p{2cm}p{2cm}p{2cm}p{2cm}p{2cm}p{3cm}@{}}
    \centering
    \begin{tabular}{@{} >{\centering}p{3.5cm} >{\centering}p{3.5cm} >{\centering}p{3.5cm} @{}}
    
		\toprule
		\textbf{Fabricante} & \textbf{Processadores} & \textbf{Missões por Fabricante} \tabularnewline 
        \midrule
        Xilinx & Zynq 7020, Zynq 7030, Zynq 7045, Ultrascale+, etc. & 24 \tabularnewline
        
        \midrule
        Atmel + Microchip & ATmega329P, AT91SAM9G20, PIC24F, etc. & 22 \tabularnewline 

        \midrule
        Texas Instruments & MSP430, OMAP3530, Sitara AM3703, etc. & 15 \tabularnewline 

        \midrule
        Cobham Gaisler & GR712RC, UT699, LEON3FT & 8 \tabularnewline
        
        \bottomrule
	\end{tabular}
	\fonte{Elaboração própria com base em George e Wilson, 2018, página 463.}
\end{table}

Comparando ambas tabelas, é possível verificar que a maioria dos processadores apresentados, no contexto explorado por (GEORGE E WILSON,  2018), são de duas fabricantes: Xilinx (especialmente \textit{chips} da família Zynq 7000) e Microchip (incluindo Atmel). Além disso, a maior parte dos projetos comerciais vistos apresentam memórias FRAM, que possuem um número máximo de ciclos de leitura e escrita muito elevada, além de memórias Flash. Outro destaque foi a presença de sensores de tensão e corrente, bem como magnetômetros e giroscópios.

Além disso, projetos como o OBDH Nanomind Z7000 (Gomspace Nanomind Z7000 Datasheet, 2019) demonstraram sua efetividade em diversas missões, como FSSCAT (CAMPS et al., 2018), ORCA (BARLES et al., 2022) e CubeMAP (WEIDMANN et al., 2020), o que mostra a confiabilidade e herança de voo de \textit{hardwares} contendo SoCs (\textit{System-on-a-Chip}) da família Zynq 7000. Na Figura \ref{fig:nanomind}, podemos verificar o diagrama de blocos do anteriormente citado Nanomind Z7000.

\begin{figure}[H]
    \centering
    \includegraphics[scale=0.8]{images/nanomind z7000.png}
    \caption{Diagrama de blocos do OBDH Nanomind Z7000.}
    \label{fig:nanomind}
    \fonte{GomSpace Nanomind Z7000 Datasheet, 2019.}
\end{figure}

\subsection{Projetos Acadêmicos}

Outro ponto são os OBDHs propostos em publicações acadêmicas. Serão estudados quatro casos de design de OBDH, ainda no contexto de nanossatélites. 

No primeiro caso, o OBDH foi feito para ser compacto e reconfigurável, como o projeto proposto nesse trabalho. O sistema foi pensado para conter um processador, SDRAMs, uma Flash NOR, uma Flash NAND, uma FPGA e algumas interfaces externas (ZHOU et al., 2018). O diagrama de blocos do OBDH proposto pelos autores está disposto na Figura \ref{fig:zhou}.

\begin{figure}[H]
    \centering
    \includegraphics[scale=0.8]{images/zhou.png}
    \caption{Diagrama de blocos do OBDH proposto por ZHOU et al., 2018.}
    \label{fig:zhou}
    \fonte{ZHOU et al., 2018.}
\end{figure}

Na segunda publicação estudada, o OBDH é parte de um sistema que implementa um sistema operacional em tempo real (RTOS), outro objetivo desse trabalho. Nesse caso, o OBDH é capaz de verificar telecomandos, sincronizar sistemas, reportar eventos e monitorar parâmetros (PUTRA, 2021). Seu diagrama de blocos do \textit{hardware} está disposto na Figura \ref{fig:putra}.

\begin{figure}[H]
    \centering
    \includegraphics[scale=0.8]{images/putra.png}
    \caption{Diagrama de blocos do OBDH proposto por PUTRA, 2021.}
    \label{fig:putra}
    \fonte{PUTRA, 2021.}
\end{figure}

No terceiro caso, a missão incluía a pesquisa e observação em órbita média (MEO), ou seja, em condições mais críticas do que o propósito do OBDH projetado nesse trabalho. Mesmo assim, as noções da arquitetura proposta são muito parecidas com o estado da arte para LEO, usando inclusive um SoC da família Zynq 7000 (LOFFLER, 2021). O diagrama de blocos do OBDH proposto nesse trabalho está disposto na Figura \ref{fig:loffler}.

\begin{figure}[H]
    \centering
    \includegraphics[scale=0.8]{images/loffler.png}
    \caption{Diagrama de blocos do OBDH proposto por LOFFLER et al., 2021.}
    \label{fig:loffler}
    \fonte{LOFFLER et al., 2021.}
\end{figure}

Nos três casos existem semelhanças na arquitetura, incluindo memórias usadas e interfaces de comunicação. Com isso, juntamente com o estudo dos projetos FloripaSat-1 e FloripaSat-2 e projetos comerciais, é possível começar a projetar o \textit{hardware} do OBDH, utilizando as diretrizes citadas e as heranças de voo, tomando como base os projetos citados, escolhendo os componentes e respeitando os requisitos impostos.







% Citação: 
% https://sci-hub.se/10.1109/jproc.2018.2802438
 


% ----------------------------------------------------------
\chapter{Desenvolvimento do Projeto}
% ----------------------------------------------------------

Após estudar os desdobramentos dos efeitos de órbita baixa e entender o que é necessário para se realizar um projeto confiável de computador de bordo de um nanossatélite, foi necessária a compreensão dos pré-requisitos de projeto. Com isso, foram escolhidos os componentes principais da placa, e a partir deles, um esquemático elétrico foi construído, propondo enfim um hardware confiável, robusto e versátil.  

\section{Pré-Requisitos de Projeto}

Como dito, foi preciso entender os pré-requisitos impostos para o OBDH da terceira geração do SpaceLab. Abaixo, na Tabela \ref{tab:Tab_Req}, se encontram os requisitos gerais do projeto, em conjunto com o pretexto e com o nível de prioridade.

\begin{longtable}{@{}p{5cm}p{5cm}p{3.5cm}@{}}
    \centering
	\ABNTEXfontereduzida
	\label{tab:Tab_Req}\tabularnewline
	\caption{Requisitos do projeto.}\tabularnewline
	%\begin{tabular}{@{}p{2cm}p{2cm}p{2cm}p{2cm}p{2cm}p{2cm}p{3cm}@{}}
	\hline
	\textbf{\centering{Descrição}} & \textbf{\centering{Pretexto}} & \textbf{\centering{Prioridade}} \tabularnewline
        \hline
        O módulo OBDH deve ser compatível com o padrão CubeSat & Assegura compatibilidade com outros satélites desenvolvidos no SpaceLab & Alta \tabularnewline
        
       \hline
        O módulo OBDH deve operar corretamente entre -40°C e 85°C & Para operar com segurança em um ambiente LEO & Alta \tabularnewline

       \hline
        O módulo OBDH deve possuir um microcontrolador capaz de usar um sistema Linux & Para gerenciar e coordenar operações dentro e fora do módulo, sendo capaz de realizar tarefas complexas  & Alta \tabularnewline

       \hline
        O módulo OBDH deve possuir uma memória DDR com capacidade de 512Mb (preferencialmente com ECC)  & Memória suficiente para operações do OBDH e armazenamento de dados  & Alta\tabularnewline

        \hline
        O módulo OBDH deve possuir uma memória FRAM para armazenar parâmetros de configuração & Provê memória não-volátil e duradoura, menos sucetível à radiação & Alta \tabularnewline 

        \hline
        O módulo OBDH deve possuir uma memória Flash para armazenar pacotes (preferencialmente com ECC) & Para armazenar dados e pacotes recebidos & Alta\tabularnewline 

        \hline
        O módulo OBDH deve possuir um WDT para reiniciar o microcontrolador em caso de falha de \textit{software} & Reinicia automaticamente o microcontrolador caso haja a falha  & Alta \tabularnewline

        \hline
        O módulo OBDH deve possuir sensores de medição de tensão e corrente em suas tensões & Para monitoramento de potência consumida & Alta\tabularnewline

        \hline
        O módulo OBDH deve possuir proteção de sobrecorrente (20\% acima do valor nominal) & Para proteção contra \textit{latch-up}  & Alta \tabularnewline

        \hline
        O módulo OBDH deve possuir um giroscópio para medição de velocidade angular & Para permitir controle ativo do satélite  & Alta \tabularnewline 

       \hline
        O módulo OBDH deve possuir um magnetômetro & Para permitir controle ativo do satélite  & Alta \tabularnewline

        \hline
        O módulo OBDH deve possuir uma interface RS-422 para transmissão de mensagens de \textit{debug/log} e receber parâmetros de configuração & Comunicação de longa distância com maior imunidade ao ruído e maior taxa de dados (comparando com UART)  & Alta \tabularnewline

       \hline
        O módulo OBDH deve possuir uma interface CAN para receber e transmitir comandos e dados & Comunicação robusta e com suporte a múltiplos sistemas do CubeSat  & Alta\tabularnewline

        \hline
        O módulo OBDH deve possuir uma interface acessível externamente para programação do microcontrolador & Para o módulo ser facilmente programado pelo time  & Alta \tabularnewline

        \hline
        O módulo OBDH deve possuir uma interface para uma \textit{daughter board} & Para prover suporte a outras interfaces e periféricos  & Baixa \tabularnewline

        \hline
        O módulo OBDH deve possuir um sensor de temperatura com precisão menor ou igual a 1°C & Para prevenir danos de temperaturas extremas & Baixa\tabularnewline

        \hline
        O módulo OBDH deve possuir uma interface RS-485 para receber e transmitir comandos e dados  & Para transmissão robusta de dados com módulos externos & Baixa \tabularnewline
       \hline
	\centering{\fonte{Elaboração própria.}}
\end{longtable}
\begin{flushleft}
\begin{thebibliography}{00}
\bibitem{b1} AAC Clyde Space. Datasheet: Kryten-M3. Disponível em $<$https://www.aac-clyde.space/wp-content/uploads/2021/10/AAC\_DataSheet\_Kryten.pdf$>$.  Acesso em: 07 de junho, 2024.

\bibitem{b2} CAPPELLETTI, C.; BATTISTINI, S.; MALPHRUS, B. Cubesat Handbook: From Mission Design to Operations. Editora Elsevier, 2021.

\bibitem{b3} GEORGE, A. D.; WILSON, C. M. Onboard processing with hybrid and reconfigurable computing on small satellites. Proceedings of the IEEE. Institute of Electrical and Electronics Engineers, 2018.

\bibitem{b4} GomSpace. Datasheet: NanoMind A3200. Disponível em $<$https://gomspace.com/UserFiles/Subsystems/datasheet/gs-ds-nanomind-a3200\_1006901-117.pdf$>$. Acesso em: 07 de junho, 2024.

\bibitem{b5} GomSpace. Datasheet: NanoMind HP MK3. Disponível em $<$https://gomspace.com/UserFiles/Subsystems/datasheet/gs-ds-NanoMind\_HP\_MK3.pdf$>$. Acesso em: 07 de junho, 2024.

\bibitem{b6} ISIS Space On Board Computer. Disponível em $<$https://www.isispace.nl/product/on-board-computer/$>$. Acesso em: 07 de junho, 2024.

\bibitem{b7} Nano Avionics. CubeSat On-Board Computer – Main Bus Unit SatBus 3C2. Disponível em $<$https://nanoavionics.com/cubesat-components/cubesat-on-board-computer-main-bus-unit-satbus-3c2/$>$. Acesso em: 07 de junho, 2024.

\bibitem{b8} MARCELINO, G. M. et al. A critical embedded system challenge: The FloripaSat-1 mission. IEEE Latin America Transactions, 2020.

\bibitem{b9} MARCELINO, G. M. et al. FloripaSat-2: An Open-Source Platform for CubeSats. IEEE embedded systems letters, 2024.
\end{thebibliography}
\end{flushleft}


\postextual
%\begin{flushleft}
\begin{thebibliography}{00}
\bibitem{b1} AAC Clyde Space. Datasheet: Kryten-M3. Disponível em $<$https://www.aac-clyde.space/wp-content/uploads/2021/10/AAC\_DataSheet\_Kryten.pdf$>$.  Acesso em: 07 de junho, 2024.

\bibitem{b2} CAPPELLETTI, C.; BATTISTINI, S.; MALPHRUS, B. Cubesat Handbook: From Mission Design to Operations. Editora Elsevier, 2021.

\bibitem{b3} GEORGE, A. D.; WILSON, C. M. Onboard processing with hybrid and reconfigurable computing on small satellites. Proceedings of the IEEE. Institute of Electrical and Electronics Engineers, 2018.

\bibitem{b4} GomSpace. Datasheet: NanoMind A3200. Disponível em $<$https://gomspace.com/UserFiles/Subsystems/datasheet/gs-ds-nanomind-a3200\_1006901-117.pdf$>$. Acesso em: 07 de junho, 2024.

\bibitem{b5} GomSpace. Datasheet: NanoMind HP MK3. Disponível em $<$https://gomspace.com/UserFiles/Subsystems/datasheet/gs-ds-NanoMind\_HP\_MK3.pdf$>$. Acesso em: 07 de junho, 2024.

\bibitem{b6} ISIS Space On Board Computer. Disponível em $<$https://www.isispace.nl/product/on-board-computer/$>$. Acesso em: 07 de junho, 2024.

\bibitem{b7} Nano Avionics. CubeSat On-Board Computer – Main Bus Unit SatBus 3C2. Disponível em $<$https://nanoavionics.com/cubesat-components/cubesat-on-board-computer-main-bus-unit-satbus-3c2/$>$. Acesso em: 07 de junho, 2024.

\bibitem{b8} MARCELINO, G. M. et al. A critical embedded system challenge: The FloripaSat-1 mission. IEEE Latin America Transactions, 2020.

\bibitem{b9} MARCELINO, G. M. et al. FloripaSat-2: An Open-Source Platform for CubeSats. IEEE embedded systems letters, 2024.
\end{thebibliography}
\end{flushleft}


%\include{ref.bib}

\end{document}