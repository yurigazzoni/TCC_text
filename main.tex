\documentclass[
12pt,				% tamanho da fonte
%openright,		% capítulos começam em pág ímpar (insere página vazia caso preciso)
oneside,			% para impressão no anverso. Oposto a twoside
a4paper,			% tamanho do papel. 
chapter=TITLE,		% títulos de capítulos convertidos em letras maiúsculas
section=TITLE,		% títulos de seções convertidos em letras maiúsculas
%subsection=TITLE,	% títulos de subseções convertidos em letras maiúsculas
%subsubsection=TITLE,% títulos de subsubseções convertidos em letras maiúsculas
% -- opções do pacote babel --
english,			% idioma adicional para hifenização
brazil				% o último idioma é o principal do documento
hyperref=hidelinks]{abntex2}

\usepackage[semdicas]{tccctj} % carrega o estilo CTJ
\bibliographystyle{abntex2-alf-ufsc} % Arquivo bst com patch para correção de citação de proceedings. Produz italico em "In", conforme descrito em: https://github.com/abntex/abntex2/issues/226

% Useful packages
\usepackage{amsmath}
\usepackage{graphicx}
% Pacotes adicionais
\usepackage{multicol}
\usepackage{multirow}
\usepackage{tabularx}
\usepackage{quoting}
\quotingsetup{indentfirst={false},font={footnotesize},leftmargin=4cm,rightmargin=0cm} 
\hypersetup{hidelinks}
\usepackage{enumitem}
\setlist{noitemsep}
%\usepackage{hyperref}
%\usepackage{array,epsfig}
%\usepackage{amsfonts}
%\usepackage{amssymb}
%\usepackage{amsxtra}
%\usepackage{amsthm}
%\usepackage{mathrsfs}
%\usepackage{color}
%\usepackage{indentfirst}
\usepackage{float}
\usepackage{booktabs}
\usepackage{array}
\usepackage{longtable}
\usepackage[nobiblatex]{xurl}
\usepackage{textcomp}

% ----------------------------------------------------

% ----------------------------------------------------
% Informações do trabalho
\autor{Yuri Gazzoni Rezende}
\titulo{Projeto de um Computador de Bordo Robusto e Versátil Para Uso em pequenos satélites}
% \subtitulo{Subtítulo} % Preenchimento opcional para quando houver sub-título
\orientador{Msc. Gabriel Mariano Marcelino}
%\coorientador{Nome do coorientador} % Preenchimento opcional para quando houver co-orientador
\curso{Departamento de Engenharia Elétrica e Eletrônica}
% \titulacao{Bacharel em x}
\datadadefesa{15}{novembro}{2024}
\local{Florianópolis}

\begin{document}

% ----------------------------------------------------
% 1. Capa do trabalho
\imprimircapa

% 2. Folha de rosto
\imprimirfolhaderosto

% 3. Folha de aprovação
\begin{folhadeaprovacao}
    \membrodabanca[Orientador]{Msc. Gabriel Mariano Marcelino}{}
    \membrodabanca{Prof. Dr. Eduardo Bezerra}{UFSC}
    \membrodabanca{Membro da banca 2}{UFSC}
    \membrodabanca{Membro da banca 3}{UFSC}
\end{folhadeaprovacao}

% 4. Dedicatória
%\begin{dedicatoria}
%    Este trabalho é dedicado aos meus colegas de classe e aos meus queridos pais.
%\end{dedicatoria}

% 5. Agradecimentos 
%\begin{agradecimentos}
%   Inserir os agradecimentos aos colaboradores à execução do trabalho.
%\end{agradecimentos}

% 6. Epígrafe
%\begin{epigrafe}
%    \aspas
%    A natureza é um enorme jogo de xadrez disputado por deuses, e que temos o privilégio de observar.
%    As regras do jogo são o que chamamos de física fundamental, e compreender essas regras é a nossa meta.
%    \aspas
%    \autor{Richard Phillips Feynman} % autor da epígrafe
%\end{epigrafe}

% 7. Resumo em português
\begin{resumo}
Este trabalho apresenta o projeto e desenvolvimento de uma arquitetura de hardware para um computador de bordo robusto e versátil, adequado para pequenos satélites, como CubeSats, que operam em órbitas baixas (LEO). O objetivo principal foi desenvolver uma solução que atendesse aos requisitos críticos do ambiente espacial, integrando componentes comerciais (COTS) selecionados conforme diretrizes de herança de voo e normas da ESA e NASA, assegurando confiabilidade e estabilidade. Para garantir versatilidade, a solução foi baseada em um SoC da família Zynq, que integra um microprocessador e uma FPGA, permitindo a execução de um sistema RTOS. O projeto adota métodos de escolha de componentes e estimativas de potência. Como resultado, obtém-se uma arquitetura com interfaces genéricas, o que permite a adaptação a diferentes missões e estudos. Conclui-se que o computador de bordo projetado cumpre as exigências de robustez e flexibilidade para missões espaciais em pequenos satélites.

\palavrachave{CubeSats}
\palavrachave{Computador de Bordo}
\palavrachave{Versatilidade}
\end{resumo}

% 9. Lista de figuras
\imprimirlistafiguras

% 10. Lista de quadros
%\imprimirlistaquadros

% 11. Lista de tabelas
\imprimirlistatabelas

% 12. Lista de abreviaturas e siglas
\begin{siglas}
	\item[ADC] \textit{Analog to Digital Converter}
	\item[ADCS] \textit{Attitude Determination and Control System}
	\item[CAN] \textit{Controller Area Network}
	\item[COTS] \textit{Commercial-off-the-shelf}
	\item[DDR] \textit{Double Data Rate}
	\item[ECSS] \textit{European Cooperation for Space Standardization}
	\item[EMI] \textit{Electromagnetic Interference}
	\item[ESA] \textit{European Space Agency}
	\item[ESR] \textit{Equivalent Series Resistor}
	\item[FPGA] \textit{Field-Programmable Gate Array}
	\item[HP] \textit{High Performance} 
	\item[HR] \textit{High Range}
	\item[I2C] \textit{Inter-Integrated Circuit}
	\item[JTAG] \textit{Joint Test Action Group}
	\item[LEO] \textit{Low Earth Orbit}
	\item[MEO] \textit{Medium Earth Orbit}
	\item[MIO] \textit{Multiplexed In-Out}
	\item[NASA] \textit{National Aeronautics and Space Administration}
	\item[NPSL] \textit{NASA Part Selection List}
	\item[OBDH] \textit{On-board Data Handling} 
	\item[PCB] \textit{Printed Circuit Board}
	\item[PL] \textit{Programmable Logic}
	\item[PS] \textit{Processing System}
	\item[QSPI] \textit{Quad Serial Peripheral Interface}
	\item[RTC] \textit{Real Time Clock}
	\item[RTOS] \textit{Real Time Operational System}
	\item[SDR] \textit{Single Data Rate}
	\item[SEE] \textit{Single Event Effects}
	\item[SoC] \textit{System-on-a-Chip}
	\item[SPI] \textit{Serial Peripheral Interface}
	\item[TID] \textit{Total Ionizing Dose}
	\item[UART] \textit{Universal Asynchronous Receiver-Transmitter}
	\item[WDT] \textit{Watchdog Timer}	
	\item[XADC] \textit{Xilinx Analog to Digital Converter}
\end{siglas}

% 13. Lista de símbolos 

% 14. Sumário
\imprimirsumario

% ----------------------------------------------------

% Elementos textuais
\textual


\chapter{Introdução}
\label{Chap:intro}

CubeSats são pequenos satélites que atendem a estritas formas de cubos padronizados de 10 cm de aresta, além de pesarem menos de 300 kg. Cada um desses cubos padronizados recebe a denominação de 1U, e os tamanhos subsequentes de 1,5U, 2U, 3U, e assim por diante (CUBESAT Design Specification, 2022). Devido a essa padronização e ao uso de componentes comerciais, os CubeSats podem ser produzidos em massa, o que diminui substancialmente os custos de lançamento e desenvolvimento (CubeSat 101, 2017).

O desenvolvimento de satélites de pequeno porte, como CubeSats e nanosatélites, trouxe novas oportunidades e desafios para a indústria espacial, permitindo que uma ampla gama de missões científicas, comerciais e educacionais fosse realizada com custos reduzidos e prazos de desenvolvimento mais curtos (CUBESAT Design Specification, 2022). No entanto, a miniaturização e a operação em ambientes espaciais impõem requisitos à robustez, confiabilidade e versatilidade dos sistemas embarcados, especialmente para os módulos OBDHs (\textit{On-Board Data Handling}). Nesse contexto, a arquitetura de um computador de bordo eficiente e robusto é essencial para o gerenciamento seguro das operações e para garantir a integridade das missões.

Essa segurança e integridade são pontos chave no desenvolvimento de CubeSats no SpaceLab, laboratório da UFSC especializado em desenvolvimento de sistemas espaciais para a comunidade científica e para a indústria. Um dos objetivos primários do SpaceLab é o desenvolvimento de uma plataforma \textit{open-source}, tanto para \textit{software} quanto \textit{hardware}, o que já foi feito nos desenvolvimentos do FloripaSat-1 (MARCELINO et al., 2020) e FloripaSat-2 (MARCELINO et al., 2024). Com esse paradigma, a oportunidade de se ter um computador de bordo mais robusto (com mais memória e capacidade de processamento) e versátil surgiu, como uma consequência direta dos desenvolvimentos das gerações anteriores de \textit{hardware} do laboratório.

O sistema desenvolvido para o presente trabalho foca na implementação de uma arquitetura de processamento e memória capaz de atender às demandas de um satélite de pequeno porte. Esse sistema deve operar de maneira confiável em ambientes suscetíveis à radiação e alta variação de temperatura, em conjunto com a otimização do uso de energia. Além disso, a versatilidade do computador de bordo é essencial para adaptar o sistema a diferentes tipos de missões, desde operações de imagem e telemetria até experimentos científicos em órbita. Para isso, é necessário que o sistema ofereça uma arquitetura versátil, baseada em uma FPGA (\textit{Field-Programmable Gate Array}), com capacidade de expansão e adaptação a novos sensores e módulos de comunicação.

Inicialmente, será feita uma revisão bibliográfica, explorando as características de computadores de bordo comerciais e de trabalhos acadêmicos, além de entender como a radiação em órbita baixa afeta os sistemas eletrônicos. Depois disso, será definida uma arquitetura, respeitando os requisitos impostos, em conjunto com uma estimativa de consumo de potência. Com a arquitetura, será desenvolvido um esquemático, usando o software Altium Designer (versão 24.4.1). Por fim, serão apresentados os resultados obtidos e as considerações finais e conclusões para esse projeto.

\section{Objetivo geral}

O presente trabalho tem como objetivo o projetar e implementação de uma arquitetura de hardware robusta e versátil para um computador de bordo de satélite de pequeno porte, integrando diferentes tipos de memórias e periféricos para assegurar a operação confiável em ambientes espaciais adversos, garantindo a integridade dos dados e a eficiência no gerenciamento dos mesmos.

\section{Objetivos Específicos}

\begin{itemize}
    \item Analisar os requisitos de robustez em condições espaciais, com foco em resistência a radiação, tolerância a falhas e estabilidade térmica, a fim de assegurar o funcionamento contínuo do computador de bordo em órbita.
    \item Especificar uma arquitetura de hardware que permita a adaptação a diferentes tipos de missões, integrando diferentes tipos de componentes comerciais com um SoC (\textit{System-on-a-chip}).
    \item  Documentar as decisões de projeto para consolidar um guia técnico com recomendações de design para sistemas de robustos e versáteis aplicáveis a satélites de pequeno porte, contribuindo para futuras otimizações e adaptações em missões espaciais.
\end{itemize}

% ----------------------------------------------------------
\chapter{Revisão Bibliográfica}
% ----------------------------------------------------------

Para atingir o objetivo de projetar o \textit{hardware} de um computador de bordo robusto, foi preciso buscar na literatura acadêmica o estado da arte que tange o projeto de OBDHs para satélites de pequeno porte, especialmente para CubeSats.
 
Primeiro, foi necessário um estudo sobre a radiação em LEO - \textit{Low Earth Orbit} (órbitas com raio menor que 1000 km, segundo ESA, 2024), para que a escolha dos componentes do projeto seja a melhor possível. Com esse estudo, buscaram-se formas de mitigar os efeitos mais conhecidos e verificar como instituições têm lidado com componentes do tipo \textit{commercial-off-the-shelf} (COTS).% Além disso, também foi dada a fundamentação dos conversores de potência, tipos de memória, microprocessadores e interfaces de comunicação.

Depois, foram analisadas as placas de OBDH dos projetos do FloripaSat-1 e FloripaSat-2, desenvolvidas pelo SpaceLab da UFSC. Por fim, outros projetos comerciais foram estudados para obtenção de noções sobre a arquitetura e componentes usados. Um panorama geral foi feito, verificando-se principalmente os componentes principais e mais críticos, ou seja, processadores, memórias voláteis e não-voláteis e outros periféricos.

\section{Radiação em LEO e Componentes COTS}

Estando em solo terrestre, os eletrônicos atuais estão bem protegidos contra a maior parte da radiação incidente do universo. No caso dos satélites orbitais, a proteção atmosférica é atenuada pela distância em relação ao solo, mesmo para aqueles que operam em LEO. Nesse caso, a radiação pode ser suficientemente significativa para causar a mudança do comportamento eletromagnético dos materiais, causando efeitos como falhas, aquisição ou execução errada de comandos e distorções dos sinais (MAYANBARI, 2011) (LABEL, 2004).  Esses danos são divididos em dois grupos (JUNQUEIRA, 2020): os acumulativos como o TID (\textit{Total Ionizing Dose}), e os SEE (\textit{Single Event Effects}), que indicam o acontecimento de eventos únicos. 

Ainda segundo Junqueira (2020), o TID se caracteriza principalmente pela formação de pares elétron-lacuna, onde o primeiro aumenta a condutividade do material e o segundo contribui para oxidação, mudando as características elétricas do componente com o tempo.  Já os SEE ocorrem quando um íon atravessa um componente crítico, gerando uma linha de ionização que pode ou não ser destrutiva. 

Por esse motivo, quando são escolhidos os componentes críticos para o \textit{hardware} de um \textit{CubeSat}, em sua maioria COTS, deve-se levar em consideração algumas diretrizes cruciais. Segundo Carmo et al. (2021), o componente escolhido precisa atender os requisitos operacionais, concomitante ao gerenciamento de riscos com mitigações e blindagens. 

Com isso, é possível ver três formas confiáveis de escolher cada componente: usando as diretrizes da ESA (\textit{European Space Agency}), as da NASA (\textit{National Aeronautics and Space Administration}) e também através da herança de voo, ou seja, escolhendo componentes que já estiveram em missões semelhantes ou mais críticas. Nos dois primeiros casos, a consulta é através da norma ECSS-Q-ST-60C para a ESA e da lista NPSL (\textit{NASA Part Selection List}) para a NASA.  No caso da herança de voo, outros projetos devem ser analisados e consultados, o que será feito na seção a seguir.

%\section{Conversores de potência}

%Também chamados de conversores DC-DC, são uma peça crucial em um projeto de PCB. Através de uma tensão de entrada, conseguem convertê-la em tensões maiores ou menores, conforme a necessidade do projetista. Existem diferentes tipos de conversores DC-DC, e tratar-se-á apenas dos conversores chaveados \textit{step-up} e \textit{step-down} nessa seção. 

%\subsection{Conversores \textit{step-up}}

%\section{Memória}

%\section{Microprocessadores}

%\section{Interfaces de comunicação}

\section{Projetos Anteriores}

\subsection{FloripaSat-1}
% https://ieeexplore.ieee.org/abstract/document/9085277
O FloripaSat-1 (MARCELINO et al., 2020) é uma plataforma \textit{open-source} para nanossatélites, além de ser também o nome do primeiro satélite lançado pelo SpaceLab. O satélite FloripaSat-1 é um CubeSat 1U,  composto de três módulos: um módulo de fornecimento de potência (EPS), um computador de bordo (OBDH) e um módulo de telemetria e comunicação (TTC). Além disso, possuía duas cargas úteis que consistiam de placas com FPGAs. A missão tinha como objetivos a validação do satélite em órbita, tanto dos módulos desenvolvidos na UFSC quanto do módulo de comunicação desenvolvido no INPE e do módulo da FPGA tolerante a radiação.

Seu OBDH foi feito para realização da interface e comunicação entre os módulos e \textit{payloads}. Aqui, destacam-se os sensores presentes: uma \textit{Inertial Measurement Unit} (com giroscópio, magnetômetro e acelerômetro), a interface com os sensores dos painéis solares e as medições de tensão e corrente de entrada do próprio módulo.

Além disso, contava com um microprocessador de 16 bits, memórias flash (IS25LP128) e suporte para cartão microSD para armazenamento.

\subsection{FloripaSat-2}
% https://ieeexplore.ieee.org/abstract/document/10078027
O FloripaSat-2 é a segunda geração da plataforma \textit{open-source} desenvolvida pelo SpaceLab, baseando-se no projeto FloripaSat-1 e trazendo melhorias para os três módulos principais (MARCELINO et al., 2024). O diagrama de blocos do CubeSat proposto está disposto na Figura \ref{fig:floripasat2}, onde pode-se verificar as interfaces do OBDH com o restante do módulo.

\begin{figure}[H]
    \centering
    \includegraphics[scale=0.8]{images/floripasat2.png}
    \caption{Diagrama de blocos da plataforma FloripaSat-2.}
    \label{fig:floripasat2}
    \fonte{MARCELINO et al., 2024.}
\end{figure}

Especificamente para o OBDH, foram introduzidas uma memória FRAM (\textit{Ferroelectric Random-Access Memory}) e uma Flash NOR de maior capacidade de armazenamento, o que mostra uma melhoria clara de capacidade e confiabilidade. Outras duas melhorias importantes foram, primeiramente, a adição de um conector para eventualmente conectar uma \textit{daughter board} à placa, e, segundamente, a adição de \textit{buffers} às trilhas de I2C (\textit{Inter-Integrated Circuit}) entre os módulos. Isso acrescenta flexibilidade e confiabilidade ao OBDH da segunda geração.

\subsection{Projetos Comerciais}

Abaixo se encontram sintetizados os projetos comerciais estudados, para obtenção de noções sobre a arquitetura e componentes usados. Foram verificados principalmente os processadores, as memórias voláteis e não-voláteis, as interfaces de comunicação e outros periféricos (ADCs, RTC, etc.) utilizados. A Tabela \ref{tab:Tab_Rev} mostra a pesquisa realizada sobre o estado da arte, em conjunto com os dados de George e Wilson (2018), sintetizados na Tabela \ref{tab:Tab_Missoes}.

\begin{table}[htb]
    \centering
	\ABNTEXfontereduzida
	\caption{\label{tab:Tab_Rev}Comparação entre os principais modelos comerciais de OBDH disponíveis atualmente no mercado.}
	%\begin{tabular}{@{}p{2cm}p{2cm}p{2cm}p{2cm}p{2cm}p{2cm}p{3cm}@{}}
    \begin{tabular}{@{}p{2cm}p{2.6cm}p{2cm}p{2cm}p{2.2cm}p{2.6cm}@{}}
		\toprule
		\textbf{Fabricante} & \textbf{Nome do Produto} & \textbf{Processador} & \textbf{Memórias} & \textbf{Periféricos} & \textbf{Interfaces de comunicação} \\ 
        \midrule
        GomSpace & NanoMind A3200 & AT32UC3C & Flash, SDRAM, FRAM & Giroscópio, Magnetômetro, Transceivers, Sensores de temperatura & CAN, I2C, SPI, JTAG, USART \\%https://gomspace.com/UserFiles/Subsystems/datasheet/gs-ds-nanomind-a3200_1006901-117.pdf
        
        \midrule
        GomSpace & NanoMind HPMK3 & Zynq 7030 & Flash, eMMC, DDR3 & Watchdog, Sensores de temperatura, VCO, Sensores de tensão e corrente & CAN, USART, USB, I2C, JTAG, LVDS, SpaceWire \\ %https://gomspace.com/UserFiles/Subsystems/datasheet/gs-ds-NanoMind_HP_MK3.pdf

        \midrule
        ISIS Space & ISIS On Board Computer & Atmel & Flash, SDRAM, FRAM, Cartões SD & Sensores de temperatura, Sensores de tensão e corrente, RTC, ADC & USART, USB, I2C, JTAG, PWM \\ %https://www.isispace.nl/product/on-board-computer/

        \midrule
        Nano Avionics & SatBus 3C2 & Não informado & Flash, FRAM, Cartões SD & Giroscópio, Magnetômetro, Rádio UHF, ADC & CAN, SPI, I2C, USART, PWM, USB \\ %https://nanoavionics.com/cubesat-components/cubesat-on-board-computer-main-bus-unit-satbus-3c2/

        \midrule
        AAC Clyde Space & Kryten-M3 & Smart Fusion 2 SoC & MRAM, eNVM & RTC, Sensores de tensão e corrente & CAN, SPI, I2C, USART, RS422, LVDS \\ %https://www.aac-clyde.space/what-we-do/space-products-components/command-data-handling/kryten-m3       


		
        \\ \bottomrule
	\end{tabular}
	\fonte{Elaboração própria.}
\end{table}

\begin{table}[H]
	\ABNTEXfontereduzida
	\caption{\label{tab:Tab_Missoes}Síntese da tabela apresentada por George e Wilson (2018).}
	%\begin{tabular}{@{}p{2cm}p{2cm}p{2cm}p{2cm}p{2cm}p{2cm}p{3cm}@{}}
    \centering
    \begin{tabular}{@{} >{\centering}p{3.5cm} >{\centering}p{3.5cm} >{\centering}p{3.5cm} @{}}
    
		\toprule
		\textbf{Fabricante} & \textbf{Processadores} & \textbf{Missões por Fabricante} \tabularnewline 
        \midrule
        Xilinx & Zynq 7020, Zynq 7030, Zynq 7045, Ultrascale+, etc. & 24 \tabularnewline
        
        \midrule
        Atmel + Microchip & ATmega329P, AT91SAM9G20, PIC24F, etc. & 22 \tabularnewline 

        \midrule
        Texas Instruments & MSP430, OMAP3530, Sitara AM3703, etc. & 15 \tabularnewline 

        \midrule
        Cobham Gaisler & GR712RC, UT699, LEON3FT & 8 \tabularnewline
        
        \bottomrule
	\end{tabular}
	\fonte{Elaboração própria com base em George e Wilson, 2018, página 463.}
\end{table}

Comparando ambas tabelas, é possível verificar que a maioria dos processadores apresentados, no contexto explorado por (GEORGE E WILSON,  2018), são de duas fabricantes: Xilinx (especialmente \textit{chips} da família Zynq 7000) e Microchip (incluindo Atmel). Além disso, a maior parte dos projetos comerciais vistos apresentam memórias FRAM, que possuem um número máximo de ciclos de leitura e escrita muito elevada, além de memórias Flash. Outro destaque foi a presença de sensores de tensão e corrente, bem como magnetômetros e giroscópios.

Além disso, projetos como o OBDH Nanomind Z7000 (Gomspace Nanomind Z7000 Datasheet, 2019) demonstraram sua efetividade em diversas missões, como FSSCAT (CAMPS et al., 2018), ORCA (BARLES et al., 2022) e CubeMAP (WEIDMANN et al., 2020), o que mostra a confiabilidade e herança de voo de \textit{hardwares} contendo SoCs (\textit{System-on-a-Chip}) da família Zynq 7000. Na Figura \ref{fig:nanomind}, podemos verificar o diagrama de blocos do anteriormente citado Nanomind Z7000.

\begin{figure}[H]
    \centering
    \includegraphics[scale=0.8]{images/nanomind z7000.png}
    \caption{Diagrama de blocos do OBDH Nanomind Z7000.}
    \label{fig:nanomind}
    \fonte{GomSpace Nanomind Z7000 Datasheet, 2019.}
\end{figure}

\subsection{Projetos Acadêmicos}

Outro ponto são os OBDHs propostos em publicações acadêmicas. Serão estudados quatro casos de design de OBDH, ainda no contexto de nanossatélites. 

No primeiro caso, o OBDH foi feito para ser compacto e reconfigurável, como o projeto proposto nesse trabalho. O sistema foi pensado para conter um processador, SDRAMs, uma Flash NOR, uma Flash NAND, uma FPGA e algumas interfaces externas (ZHOU et al., 2018). O diagrama de blocos do OBDH proposto pelos autores está disposto na Figura \ref{fig:zhou}.

\begin{figure}[H]
    \centering
    \includegraphics[scale=0.8]{images/zhou.png}
    \caption{Diagrama de blocos do OBDH proposto por ZHOU et al., 2018.}
    \label{fig:zhou}
    \fonte{ZHOU et al., 2018.}
\end{figure}

Na segunda publicação estudada, o OBDH é parte de um sistema que implementa um sistema operacional em tempo real (RTOS), outro objetivo desse trabalho. Nesse caso, o OBDH é capaz de verificar telecomandos, sincronizar sistemas, reportar eventos e monitorar parâmetros (PUTRA, 2021). Seu diagrama de blocos do \textit{hardware} está disposto na Figura \ref{fig:putra}.

\begin{figure}[H]
    \centering
    \includegraphics[scale=0.8]{images/putra.png}
    \caption{Diagrama de blocos do OBDH proposto por PUTRA, 2021.}
    \label{fig:putra}
    \fonte{PUTRA, 2021.}
\end{figure}

No terceiro caso, a missão incluía a pesquisa e observação em órbita média (MEO), ou seja, em condições mais críticas do que o propósito do OBDH projetado nesse trabalho. Mesmo assim, as noções da arquitetura proposta são muito parecidas com o estado da arte para LEO, usando inclusive um SoC da família Zynq 7000 (LOFFLER, 2021). O diagrama de blocos do OBDH proposto nesse trabalho está disposto na Figura \ref{fig:loffler}.

\begin{figure}[H]
    \centering
    \includegraphics[scale=0.8]{images/loffler.png}
    \caption{Diagrama de blocos do OBDH proposto por LOFFLER et al., 2021.}
    \label{fig:loffler}
    \fonte{LOFFLER et al., 2021.}
\end{figure}

Nos três casos existem semelhanças na arquitetura, incluindo memórias usadas e interfaces de comunicação. Com isso, juntamente com o estudo dos projetos FloripaSat-1 e FloripaSat-2 e projetos comerciais, é possível começar a projetar o \textit{hardware} do OBDH, utilizando as diretrizes citadas e as heranças de voo, tomando como base os projetos citados, escolhendo os componentes e respeitando os requisitos impostos.







% Citação: 
% https://sci-hub.se/10.1109/jproc.2018.2802438
 


% ----------------------------------------------------------
\chapter{Arquitetura}
% ----------------------------------------------------------

Após estudar os desdobramentos dos efeitos de órbita baixa e entender o que é necessário para se realizar um projeto confiável de computador de bordo de um nanossatélite, foi necessária a compreensão dos pré-requisitos de projeto. Com isso, foram escolhidos os componentes principais da placa, propondo-se uma arquitetura para o sistema, propondo um hardware confiável, robusto e versátil.  

\section{Pré-Requisitos de Projeto}

Como dito, foi preciso entender os pré-requisitos impostos para o OBDH da terceira geração do SpaceLab. Abaixo, na Tabela \ref{tab:Tab_Req}, se encontram os requisitos gerais do projeto, em conjunto com o pretexto e com o nível de prioridade.

\begin{longtable}{@{}p{5cm}p{5cm}p{3.5cm}@{}}
    \centering
	\ABNTEXfontereduzida
	\label{tab:Tab_Req}\tabularnewline
	\caption{Requisitos do projeto.}\tabularnewline
	%\begin{tabular}{@{}p{2cm}p{2cm}p{2cm}p{2cm}p{2cm}p{2cm}p{3cm}@{}}
	\hline
	\textbf{\centering{Descrição}} & \textbf{\centering{Pretexto}} & \textbf{\centering{Prioridade}} \tabularnewline
        \hline
        O módulo OBDH deve ser compatível com o padrão CubeSat & Assegura compatibilidade com outros satélites desenvolvidos no SpaceLab & Alta \tabularnewline
        
       \hline
        O módulo OBDH deve operar corretamente entre -40°C e 85°C & Para operar com segurança em um ambiente LEO & Alta \tabularnewline

       \hline
        O módulo OBDH deve possuir um microcontrolador capaz de usar um sistema Linux & Para gerenciar e coordenar operações dentro e fora do módulo, sendo capaz de realizar tarefas complexas  & Alta \tabularnewline

       \hline
        O módulo OBDH deve possuir uma memória DDR com capacidade de 512Mb (preferencialmente com ECC)  & Memória suficiente para operações do OBDH e armazenamento de dados  & Alta\tabularnewline

        \hline
        O módulo OBDH deve possuir uma memória FRAM para armazenar parâmetros de configuração & Provê memória não-volátil e duradoura, menos sucetível à radiação & Alta \tabularnewline 

        \hline
        O módulo OBDH deve possuir uma memória Flash para armazenar pacotes (preferencialmente com ECC) & Para armazenar dados e pacotes recebidos & Alta\tabularnewline 

        \hline
        O módulo OBDH deve possuir um WDT (\textit{Watchdog Timer}) para reiniciar o microcontrolador em caso de falha de \textit{software} & Reinicia automaticamente o microcontrolador caso haja a falha  & Alta \tabularnewline

        \hline
        O módulo OBDH deve possuir sensores de medição de tensão e corrente em suas tensões & Para monitoramento de potência consumida & Alta\tabularnewline

        \hline
        O módulo OBDH deve possuir proteção de sobrecorrente (20\% acima do valor nominal) & Para proteção contra \textit{latch-up}  & Alta \tabularnewline

        \hline
        O módulo OBDH deve possuir um giroscópio para medição de velocidade angular & Para permitir controle ativo do satélite  & Alta \tabularnewline 

       \hline
        O módulo OBDH deve possuir um magnetômetro & Para permitir controle ativo do satélite  & Alta \tabularnewline

        \hline
        O módulo OBDH deve possuir uma interface RS-422 para transmissão de mensagens de \textit{debug/log} e receber parâmetros de configuração & Comunicação de longa distância com maior imunidade ao ruído e maior taxa de dados (comparando com UART)  & Alta \tabularnewline

       \hline
        O módulo OBDH deve possuir uma interface CAN para receber e transmitir comandos e dados & Comunicação robusta e com suporte a múltiplos sistemas do CubeSat  & Alta\tabularnewline

        \hline
        O módulo OBDH deve possuir uma interface acessível externamente para programação do microcontrolador & Para o módulo ser facilmente programado pelo time  & Alta \tabularnewline

        \hline
        O módulo OBDH deve possuir uma interface para uma \textit{daughter board} & Para prover suporte a outras interfaces e periféricos  & Baixa \tabularnewline

        \hline
        O módulo OBDH deve possuir um sensor de temperatura com precisão menor ou igual a 1°C & Para prevenir danos de temperaturas extremas & Baixa\tabularnewline

        \hline
        O módulo OBDH deve possuir uma interface RS-485 para receber e transmitir comandos e dados  & Para transmissão robusta de dados com módulos externos & Baixa \tabularnewline
       \hline
	\centering{\fonte{Elaboração própria.}}
\end{longtable}

Com as definições apresentadas na Tabela \ref{tab:Tab_Req}, foi então necessária a definição da arquitetura do hardware, ou seja, os componentes e sua interconexões, bem como as interfaces de comunicação e saídas necessárias.

\section{Arquitetura}

A partir dos requisitos, o primeiro passo foi definir de forma geral como seria o funcionamento do \textit{hardware} do projeto. Na Figura \ref{fig:arq_geral}, pode-se verificar um esquema inicial de proposta de arquitetura, usando os pontos descritos anteriormente.

\begin{figure}[H]
    \centering
    \includegraphics[scale=0.8]{images/arquitetura geral.png}
    \caption{Esquema geral de arquitetura.}
    \label{fig:arq_geral}
    \fonte{Elaboração própria.}
\end{figure}

Como podemos verificar, o microprocessador será crucial e deverá ter pinos suficientes para interface com todas as memórias, sensores e para se comunicar com os outros módulos do CubeSat. Além disso, a parte dedicada às tensões usadas deverá ser cuidadosamente feita, para suportar a potência dissipada por todos os componentes da placa de circuito impresso. A escolha de cada componente será descrita nas seções a seguir, respeitando sempre os seguintes critérios:

\begin{itemize}
	\item O componente deve funcionar corretamente nas temperaturas entre -40°C e 85°C;
	\item Circuitos integrados devem possuir herança de voo sempre que possível;
	\item Caso o circuito integrado necessite de um circuito específico, o mesmo deve conter itens preferencialmente dispostos na ECSS-Q-ST-60C, na NPSL ou similar aos mesmos, especialmente componentes discretos (capacitores, resistores, indutores, diodos, transistores, entre outros);
\end{itemize}

\subsection{Microcontrolador}

Como visto na Tabela \ref{tab:Tab_Missoes}, a fabricante com maior herança de voo estudada é a Xilinx, em especial os chips da família Zynq 7000, que são SoCs (\textit{System on a Chip}). Após um estudo próprio, o SoC Zynq 7030 se mostrou mais adequado pelas seguintes características:

\begin{itemize}
	\item Foi usado em missões extensivas em pequenos satélites (Gomspace, 2024), ou seja, possui herança de voo em missões similares em LEO e em CubeSats;
	\item Possui um envelopamento com 484 pinos, suficiente para prover as conexões necessárias para todas as interfaces requeridas (UG865, 2021);
	\item Capaz de rodar um sistema Linux (KADI et al.,2013);
	\item Por ser um SoC, possui alta adaptabilidade e flexibilidade, disponibilizando no mesmo chip uma FPGA (\textit{Field-Programmable Gate Array}) e um microprocessador, denominados respectivamente de PL e PS;
\end{itemize}

\subsection{Memórias}

As memórias serão necessárias para realizar operações, armazenar dados externos e internos e armazenar parâmetros de configuração do OBDH e de outros subsistemas do CubeSat. Para cada uma dessas funções uma memória diferente é necessária, seguindo suas características principais, sendo elas: tempo de acesso, tamanho do armazenamento e volatilidade.

\subsubsection{Memórias voláteis}

Partindo do princípio que a robustez e versatilidade estão alinhadas com a velocidade da memória, bem como sua capacidade de armazenamento máximo, a principal opção se tornou as memórias do tipo DDR (\textit{Double Data Rate}), que utilizam ambas a borda de subida e de descida para transferência de dados, atingindo o dobro de largura de banda de uma memória com SDR (\textit{Single Data Rate}) para uma mesma frequência de relógio (JEDEC, 2008). Essa relação pode ser ilustrada pela Figura \ref{fig:sdrvsddr}, onde pode-se verificar a transferência de dados do sinal DQ em relação ao sinal de relógio (bCLK e CLK) para SDR e DDR.

\begin{figure}[H]
    \centering
    \includegraphics[scale=1]{images/ddrsdr.png}
    \caption{Comparação entre DDR e SDR.}
    \label{fig:sdrvsddr}
    \fonte{KLEHN E BROX, 2003.}
\end{figure}
 
Por essa razão, foi escolhida uma memória do tipo DDR3, com capacidade de 2Gb e frequência de operação de 800 MHz.

\subsubsection{Memórias não voláteis}

No caso das memórias não voláteis, é necessária uma atenção especial ao tipo de dado que será armazenado em cada uma delas. Para o caso de dados críticos, é preciso de uma memória que possua alta resistência aos efeitos da radiação, mantendo-se um compromisso com os tempos de escrita e leitura. Por sua vez, para dados de inicialização são mais críticos os tempos de leitura, enquanto para uma memória de dados mais gerais, o importante é o armazenamento total. Por meio desses critérios, foi possível avaliar, por meio da Tabela \ref{tab:memnvol}, o tipo de memória ideal para cada caso.

\begin{table}[H]
	\ABNTEXfontereduzida
	\caption{\label{tab:memnvol}Tabela comparativa de memórias não voláteis.}
	%\begin{tabular}{@{}p{2cm}p{2cm}p{2cm}p{2cm}p{2cm}p{2cm}p{3cm}@{}}
    \centering
    \begin{tabular}{@{} >{\centering}p{2cm} >{\centering}p{3cm} >{\centering}p{3cm} >{\centering}p{3cm}>{\centering}p{3cm} @{}}
    
		\toprule
		\textbf{Memória} & \textbf{Tempo de leitura} & \textbf{Tempo de escrita} & \textbf{Tolerância à radiação} & \textbf{Armazenamento máximo} \tabularnewline 
        \midrule
        Flash NOR & Rápido & Lento & Ruim & Regular\tabularnewline
        
        \midrule
        Flash NAND & Rápido & Lento & Ruim & Bom \tabularnewline 

        \midrule
        FRAM & Rápido & Rápido & Bom & Ruim \tabularnewline 
        
        \bottomrule
	\end{tabular}
	\fonte{Adaptado de GERARDIN E PACCAGNELLA, 2010.}
\end{table}

Com isso, foi então escolhida uma FRAM (\textit{Ferroelectric Random-Access Memory}) para armazenar dados críticos, uma Flash NAND para armazenamento de dados gerais e uma Flash NOR para armazenar o boot do sistema Linux no SoC.

\subsection{Conversores DC-DC}
Nos sistemas CubeSat do SpaceLab da UFSC, o módulo responsável pelo fornecimento de potência é o chamado EPS (MARCELINO, 2024). A partir disso, partindo do pressuposto que haverá uma tensão fornecida de 3,3 V, pode-se inferir a cascata de potência a partir do mesmo. Para o caso do Zynq e da memória DDR3, circuitos integrados são necessários para gerar as seguintes tensões: 

\begin{itemize}
	\item Zynq: 1 V e 1,8 V; 
	\item DDR3L: 1,35 V e 0,675 V.
\end{itemize}

Todos os demais periféricos devem aceitar uma tensão de alimentação de 3,3 V. Outro ponto importante são os circuitos de proteção contra \textit{latch-up}, um efeito similar a um curto-circuito na trilha de alimentação de circuitos CMOS (AN-600, 1989). 

\subsection{Sensores e Periféricos}
Como dito nos pré-requisitos, alguns sensores precisam estar presentes no OBDH. Entre eles:

\begin{itemize}
	\item Um monitor de tensão, para todas as tensões importantes do sistema;
	\item Um sensor de corrente para a tensão de entrada do módulo;
	\item Um giroscópio para medir a velocidade angular em órbita;
	\item Um magnetômetro para medição do campo magnético da Terra em órbita;
	\item Um WDT para reiniciar o sistema em caso de falha de software.
\end{itemize}

\section{Visualização da Arquitetura Proposta}

Depois das decisões tomadas, foi possível montar um diagrama, apresentado na Figura \ref{fig:arq}, que mostra cada circuito do computador de bordo. Aqui, por simplicidade, foram suprimidos os transceptores do protocolo CAN (\textit{Controller Area Network}) e a parte de potência do módulo. 

\begin{figure}[H]
    \centering
    \includegraphics[scale=0.8]{images/arquitetura final.png}
    \caption{Arquitetura proposta para o OBDH.}
    \label{fig:arq}
    \fonte{Elaboração própria.}
\end{figure}

Também foram levadas em consideração as interfaces disponibilizadas pelo SoC, os componentes escolhidos e os conectores necessários. Os componentes escolhidos se encontram na Tabela \ref{tab:componentes}, conjuntamente com as interfaces requeridas para cada um, suas tensões de alimentação e suas correntes máximas no terminal de alimentação, no pior caso especificado pelo fabricante.

\begin{table}[H]
	\ABNTEXfontereduzida
	\caption{\label{tab:componentes}Informações sobre os componentes escolhidos.}
	%\begin{tabular}{@{}p{2cm}p{2cm}p{2cm}p{2cm}p{2cm}p{2cm}p{3cm}@{}}
    \centering
    \begin{tabular}{@{} >{\centering}p{2cm} >{\centering}p{4cm} >{\centering}p{2cm} >{\centering}p{3cm}>{\centering}p{3cm} @{}}
    
		\toprule
		\textbf{Componente} & \textbf{Número do Fabricante} & \textbf{Interface} & \textbf{Tensão de Alimentação} & \textbf{Corrente máxima} \tabularnewline 
        \midrule
        FRAM & CY15B104QN-50SXI & SPI & 3,3 V & 3,7 mA \tabularnewline
        
        \midrule
        Flash NOR & MT25QL128ABB1ESE-0AUT & QSPI & 3,3 V & 55 mA \tabularnewline 

        \midrule
        Flash NAND & MT29F1G01ABAFDSF-AAT:F & SPI & 3,3 V & 35 mA \tabularnewline 

        \midrule
        DDR3L & MT41K256M8DA-125:K & Paralela & 1,35 V & 182 mA \tabularnewline 

        \midrule
        WDT & TPS3823-33QDBVRQ1 & - & 3,3 V & 10 mA \tabularnewline 

        \midrule
        Monitor de Temperatura e Tensão & LTC2991IMS\#TRPBF & I2C & 3,3 V &  1,5 mA \tabularnewline 

        \midrule
        Sensor de Corrente & INA180A2IDBVR & - & 3,3 V & 1 mA \tabularnewline 

        \midrule
        Giroscópio & A3G4250D & I2C & 3,3 V & 7 mA \tabularnewline 

        \midrule
        Magnetômetro & MMC5983MA & I2C & 3,3 V & 0,45 mA \tabularnewline 

        \midrule
        Buffer I2C & TCA4311ADR & I2C & 3,3 V & 7 mA \tabularnewline 

        \midrule
        Transceptor CAN & A3G4250D & CAN & 3,3 V & 60 mA \tabularnewline 

        \midrule
        Transceptor RS-485 & THVD1451DR & Serial & 3,3 V & 3 mA \tabularnewline 

        \midrule
        Conversor DC-DC & TPS82085SILR & - & 3,3 V & - \tabularnewline 

        \midrule
        Conversor DC-DC para DDR3 & TPS51200DRCR & - & 3,3 V & 1 mA \tabularnewline 

        \midrule
        \textit{Load Switch} & TPS22920YZPR & - & 3,3 V & 0,2 mA \tabularnewline 
        
        \bottomrule
	\end{tabular}
	\fonte{Elaboração própria com base nos Datasheets de cada componente.}
\end{table}

\subsection{Estimativa de Potência Consumida}

A fim de garantir o funcionamento correto dos conversores DC-DC e seus respectivos periféricos, foi necessária uma estimativa da potência total consumida por todas as tensões disponíveis no módulo. Para isso, foi utilizada a Tabela \ref{tab:componentes}, bem como o datasheet de cada componente. No caso do SoC, sua fabricante disponibiliza uma planilha (XPE, 2019) para estimativas de potência em cada tensão de alimentação. 

Com isso, foram obtidos os valores da Tabela \ref{tab:estpow}, considerando uma eficiência de conversão de 85\% (TPS82085, 2019), já incluindo as estimativas de potência e os piores casos descritos anteriormente.

\begin{table}[H]
	\ABNTEXfontereduzida
	\caption{\label{tab:estpow}Estimativas de potência consumida.}
	%\begin{tabular}{@{}p{2cm}p{2cm}p{2cm}p{2cm}p{2cm}p{2cm}p{3cm}@{}}
    \centering
    \begin{tabular}{@{} >{\centering}p{2cm} >{\centering}p{4cm} >{\centering}p{4cm} >{\centering}p{4cm}@{}}
    
		\toprule
		\textbf{Tensão [V]} & \textbf{Potência Dissipada na Tensão [W]} & \textbf{Potência dissipada na tensão de 3,3 V [W]} & \textbf{Corrente máxima da trilha [A]} \tabularnewline 
        \midrule
         1,00 & 2,20 & 2,59 & 2,20 \tabularnewline
        
        \midrule
        1,35 & 0,25 & 0,29 & 0,18 \tabularnewline 

        \midrule
        1,80 & 0,63 & 0,74 & 0,35 \tabularnewline

        \midrule
        3,3 & 7,5 & - & 2,27  \tabularnewline        

        \bottomrule
	\end{tabular}
	\fonte{Elaboração própria.}
\end{table}

Através dessas estimativas, pode-se confirmar que o sistema de potência proposto suporta os componentes escolhidos e suas tensões e variações, mesmo quando se considera o pior caso.
% ----------------------------------------------------------
\chapter{Desenvolvimento do Projeto}
% ----------------------------------------------------------

Depois das definições apresentadas e da escolha de componentes apresentada na seção anterior, foi possível construir um esquemático elétrico, que esquematiza a PCB do OBDH. Nesse capítulo, discutir-se-á circuitos específicos mais relevantes do projeto, usando o esquemático pronto, que se encontra no Anexo I.

\section{Conversores de Potência}

Partindo do princípio que o módulo EPS da terceira geração de módulos do SpaceLab será capaz de fornecer 3,3 V para o OBDH, foi proposta uma cascata de potência descrita na Figura \ref{fig:power}. Nela, são suprimidos os circuitos de proteção que serão descritos posteriormente.

\begin{figure}[H]
    \centering
    \includegraphics[scale=0.6]{images/Power_system.png}
    \caption{Cascata de potência proposta.}
    \label{fig:power}
    \fonte{Elaboração própria.}
\end{figure}

\subsection{Filtro de Entrada}

Costumeiramente, a entrada de tensão de uma placa robusta deve ser filtrada, principalmente devido às flutuações do ruído conduzido de outros subsistemas do satélite, caracterizando o fenômeno de Interferência Eletromagnética (EMI), esquematizado na Figura \ref{fig:emi}.

\begin{figure}[H]
    \centering
    \includegraphics[scale=1]{images/EMI noise.png}
    \caption{Interferência com ruído conduzido.}
    \label{fig:emi}
    \fonte{SOH et al., 2010.}
\end{figure}

Além disso, também foi necessária a inclusão de um diodo Zener em paralelo à entrada, servindo como um elemento extra de proteção contra perturbações e transientes (CADENCE, 2023). Outra característica explorada foi a colocação de capacitores em paralelo, a fim de reduzir sua resistência em série equivalente (ESR) e sua indutância série (SARJEANT, 1990).  O filtro proposto está disposto na Figura \ref{fig:FILTRO}.  Além disso, sua magnitude e fase simuladas estão dispostas na Figura \ref{fig:filtrof}.

\begin{figure}[H]
    \centering
    \includegraphics[scale=1]{images/FILTRO.png}
    \caption{Filtro proposto.}
    \label{fig:FILTRO}
    \fonte{Elaboração própria.}
\end{figure}

\begin{figure}[H]
    \centering
    \includegraphics[scale=1]{images/filtrof.png}
    \caption{Simulação de magnitude e fase em função da frequência para o filtro proposto.}
    \label{fig:filtrof}
    \fonte{Elaboração própria.}
\end{figure}

\subsection{Cascata de potência}

Devido à escolha do SoC e da memória DDR3, foi necessária a definição de uma cascata de potência, levando-se em consideração os requisitos de (UG585, 2023), que descreve o sequenciamento das tensões para o menor consumo de potência e para garantir a integridade do fusível interno do SoC. Dessa forma, como pode-se ver na Figura \ref{fig:power}, são usados os denominados \textit{load switches}, a fim de garantir o sequenciamento descrito e garantir uma proteção efetiva contra sobrecorrente (MAK, 2018). 

O primeiro regulador, que gera a tensão de 1 V, apresentado na Figura \ref{fig:1vsupp}, é o primeiro da cascata. Seu divisor de tensão de saída foi calculado conforme (TPS82085, 2019):

\begin{equation}
	V_{out} = 0,8 * (1 + R_1/R_2) = 0,8 * (1+ 37,4k/150k) = 0,999 V
\end{equation} 

\begin{figure}[H]
    \centering
    \includegraphics[scale=0.8]{images/1vsupp.png}
    \caption{Regulador de tensão de 1 V.}
    \label{fig:1vsupp}
    \fonte{Elaboração própria com base no circuito apresentado pelo fabricante.}
\end{figure}

Também foi possível montar seu circuito de proteção de sobrecorrente, disposto na Figura \ref{fig:1vocp}. Seu resistor de entrada, que escolhe o limiar de corrente permitido, foi caculado conforme (LTC4361, 2018), considerando uma corrente 20\% superior à máxima calculada (na Tabela \ref{tab:estpow}):

\begin{equation}
	R_{sense} = 50 mV / I_{max} =50 / 2,63 = 19,01 m\Omega
\end{equation} 

\begin{figure}[H]
    \centering
    \includegraphics[scale=1]{images/1vocp.png}
    \caption{Proteção contra \textit{latch-up} para a tensão de 1 V.}
    \label{fig:1vocp}
    \fonte{Elaboração própria com base no circuito apresentado pelo fabricante.}
\end{figure}

Depois disso, para seguir com o sequenciamento requerido pelo SoC, precisa-se de um circuito de chaveamento de carga, apresentado na Figura \ref{fig:sw1}. Seu circuito é baseado no sugerido por (TPS22920, 2016), com seu ligamento sendo feito pela própria tensão de 3,3 V.

\begin{figure}[H]
    \centering
    \includegraphics[scale=1]{images/sw1.png}
    \caption{Circuito de \textit{Load switch} para a tensão de 1,8 V.}
    \label{fig:sw1}
    \fonte{Elaboração própria com base no circuito apresentado pelo fabricante.}
\end{figure}

Analogamente, para a tensão de 1,8 V, são necessários ambos um conversor e um circuito de proteção. Estes estão dispostos respectivamente nas Figuras \ref{fig:1v8supp} e \ref{fig:1v8ocp} a seguir, conjuntamente com suas equações (3) e (4) para obtenção das resistências requeridas, usando a mesma margem de 20\% de corrente máxima. 

\begin{equation}
	V_{out} = 0,8 * (1 + R_1/R_2) = 0,8 * (1+ 110k/88,7k) = 1,792 V
\end{equation} 

\begin{figure}[H]
    \centering
    \includegraphics[scale=0.8]{images/1v8supp.png}
    \caption{Regulador de tensão de 1,8 V.}
    \label{fig:1v8supp}
    \fonte{Elaboração própria com base no circuito apresentado pelo fabricante.}
\end{figure}

\begin{equation}
	R_{sense} = 50 mV / I_{max} =50 / 0,5 = 100 m\Omega
\end{equation} 

\begin{figure}[H]
    \centering
    \includegraphics[scale=1]{images/1v8ocp.png}
    \caption{Proteção contra \textit{latch-up} para a tensão de 1,8 V.}
    \label{fig:1v8ocp}
    \fonte{Elaboração própria com base no circuito apresentado pelo fabricante.}
\end{figure}

Por fim, para ligar a tensão de 3,3 V fornecida para o SoC, é necessário um último circuito de chaveamento, dessa vez com seu ligamento feito pela tensão de 1,8 V, como mostra a Figura \ref{fig:sw2}.

\begin{figure}[H]
    \centering
    \includegraphics[scale=1]{images/sw2.png}
    \caption{Circuito de \textit{Load switch} para a tensão de 3,3 V do SoC.}
    \label{fig:sw2}
    \fonte{Elaboração própria com base no circuito apresentado pelo fabricante.}
\end{figure}

Paralelamente, para a memória DDR3L, são necessários um conversor para a alimentação, de 1,35 V, e um conversor para a tensão de referência e de terminação. Esses circuitos estão dispostos respectivamente nas Figuras \ref{fig:1v35supp} e \ref{fig:1v35ref}.

\begin{equation}
	V_{out} = 0,8 * (1 + R_1/R_2) = 0,8 * (1+ 47k/68k) = 1,353 V
\end{equation} 

\begin{figure}[H]
    \centering
    \includegraphics[scale=0.8]{images/1v35supp.png}
    \caption{Regulador de tensão de 1,35 V.}
    \label{fig:1v35supp}
    \fonte{Elaboração própria com base no circuito apresentado pelo fabricante.}
\end{figure}

\begin{figure}[H]
    \centering
    \includegraphics[scale=0.8]{images/refsupp.png}
    \caption{Regulador de tensão de referência e terminação para a memória DDR3L.}
    \label{fig:1v35ref}
    \fonte{Elaboração própria com base no circuito apresentado pelo fabricante.}
\end{figure}

\section{SoC}

No caso do SoC Zynq 7030, temos no total seis blocos operacionais, que incluem o funcionamento do PL e do PS, bem como as configurações e o bloco dedicado ao controlador da memória DDR (UG585, 2023). A seguir, estão dispostas as descrições funcionais e circuitos necessários para o funcionamento correto desse SoC, separados por cada um dos blocos citados.

\subsection{Bloco de Configuração}

O banco zero do SoC é o responsável por algumas opções e sinais de configuração. Abaixo, na Tabela \ref{tab:config}, se encontra a descrição funcional de cada pino desse banco, esquematizado na Figura \ref{fig:config}. Esse esquemático, bem como seus resistores de \textit{pull-up} (Figura \ref{fig:pullupconfig}), foram baseados na documentação técnica fornecida pela Xilinx (UG865, 2023) (UG470, 2023) (UG933, 2019) (DS191, 2018).


\begin{table}[H]
	\ABNTEXfontereduzida
	\caption{\label{tab:config}Descrição funcional dos pinos de configuração.}
	%\begin{tabular}{@{}p{2cm}p{2cm}p{2cm}p{2cm}p{2cm}p{2cm}p{3cm}@{}}
    \centering
    \begin{tabular}{@{} >{\centering}p{4cm} >{\centering}p{8cm} @{}}
    
		\toprule
		\textbf{Nome} & \textbf{Função} \tabularnewline 
        \midrule
         DXN e DXP & Terminais do diodo interno para medição de temperatura. \tabularnewline
         \midrule

         VREFP e VREFN & Tensões de referência do conversor analógico digital (XADC) do SoC. \tabularnewline

       \midrule
        VP e VN & Entrada extra do XADC. \tabularnewline

       \midrule
        VCCBAT & Não utilizada. Fonte da bateria. \tabularnewline

       \midrule
        TCK, TMS, TDI e TDO & Sinais da interface JTAG.  \tabularnewline

       \midrule
        INIT\_B & Indica inicialização da memória interna de configuração. \tabularnewline

       \midrule
       PROGRAM\_B & Reset assíncrono da lógica de configuração. \tabularnewline

       \midrule
        CFGBVS & Pino que seleciona o tipo de IO do banco 0. \tabularnewline

       \midrule
        DONE & Indica que a configuração foi terminada e feita corretamente. \tabularnewline

       \midrule
        VCCADC e GNDADC & Alimentação do XADC. \tabularnewline

       \midrule
        RSVDVCC e RSVDGND & Pinos de alimentação reservados. \tabularnewline

        \bottomrule
	\end{tabular}
	\fonte{Elaboração própria com base na documentação técnica do fabricante.}
\end{table}

\begin{figure}[H]
    \centering
    \includegraphics[scale=0.8]{images/zynqconfig.png}
    \caption{Banco de configuração do SoC.}
    \label{fig:config}
    \fonte{Elaboração própria com base no circuito apresentado pelo fabricante.}
\end{figure}

\begin{figure}[H]
    \centering
    \includegraphics[scale=0.8]{images/pullupconfig.png}
    \caption{Resistores de \textit{pull-up} necessários.}
    \label{fig:pullupconfig}
    \fonte{Elaboração própria com base no circuito apresentado pelo fabricante.}
\end{figure}

Além disso, para a alimentação do XADC, foi necessário um circuito de filtragem, disposto na Figura \ref{fig:xadcfilter}, como requerido por (UG480, 2022).

\begin{figure}[H]
    \centering
    \includegraphics[scale=0.8]{images/xadcfilter.png}
    \caption{Filtro da alimentação analógica do SoC.}
    \label{fig:xadcfilter}
    \fonte{Elaboração própria com base no circuito apresentado pelo fabricante.}
\end{figure}


\subsection{Blocos do PS}

No caso do sistema de processamento (PS), temos dois bancos principais. O primeiro, denominado MIO (\textit{Multiplexed In-Out}), é onde se encontram os controladores das interfaces de comunicação, bem como a entrada de relógio e a escolha do \textit{boot}. No caso desse projeto, foi-se decidido que o SoC poderá inicializar de duas formas, sendo a primeira pela interface JTAG e a segunda pela memória Flash NOR (QSPI), escolhidos pelos resistores R48 e R51. O banco MIO e seus modos de inicialização estão dispostos nas Figuras \ref{fig:psmio} e \ref{fig:boot}. O outro banco do PS (112) não é utilizado nesse projeto.

\begin{figure}[H]
    \centering
    \includegraphics[scale=0.8]{images/psmio.png}
    \caption{Banco MIO do SoC com suas respectivas entradas e saídas.}
    \label{fig:psmio}
    \fonte{Elaboração própria com base na Tabela MIO-at-a-glance (UG585, 2023).}
\end{figure}

\begin{figure}[H]
    \centering
    \includegraphics[scale=0.8]{images/bootmode.png}
    \caption{Modos de inicialização do SoC.}
    \label{fig:boot}
    \fonte{Elaboração própria com base em (UG585, 2023).}
\end{figure}

Por fim, Na Tabela \ref{tab:interfaces}, pode-se verificar qual a função de cada barramento de comunicação, em conformidade com a Figura \ref{fig:arq}. 

\begin{table}[H]
	\ABNTEXfontereduzida
	\caption{\label{tab:interfaces}Descrição das interfaces disponibilizadas.}
	%\begin{tabular}{@{}p{2cm}p{2cm}p{2cm}p{2cm}p{2cm}p{2cm}p{3cm}@{}}
    \centering
    \begin{tabular}{@{} >{\centering}p{4cm} >{\centering}p{8cm} @{}}
    
		\toprule
		\textbf{Interface} & \textbf{Função} \tabularnewline 
        \midrule
         SPI0 & Interface SPI para circuitos internos ao módulo OBDH. \tabularnewline
        \midrule
         SPI1 & Interface SPI para circuitos externos ao módulo OBDH.  \tabularnewline
        \midrule
         QSPI0 & Interface Quad-SPI para memória de inicialização. \tabularnewline
        \midrule
        I2C0  & Interface I2C para circuitos internos ao módulo OBDH. \tabularnewline
        \midrule
        I2C1  & Interface I2C para circuitos externos ao módulo OBDH.  \tabularnewline
        \midrule
        CAN0 & Interface CAN para circuitos externos ao módulo OBDH. \tabularnewline
        \midrule
        CAN1 & Interface CAN para o módulo \textit{daughter}. \tabularnewline
        \midrule
         UART0 & Conexão serial para \textit{debugging}. \tabularnewline
        \midrule
         UART1 & Conexão serial para o transceiver RS-485. \tabularnewline
        \midrule
         PS\_GPIO & Sinais de propósito geral de entrada e saída. \tabularnewline

        \bottomrule
	\end{tabular}
	\fonte{Elaboração própria com base na documentação técnica do fabricante.}
\end{table}

\subsection{Blocos do PL}

\subsection{Bloco Controlador da Memória DDR}

\subsection{Pinos de Potência}

\section{Conexões entre blocos}
\include{cap4-conclusoes.tex}
\begin{flushleft}
\begin{thebibliography}{00}
\bibitem{b1} AAC Clyde Space. Datasheet: Kryten-M3. Disponível em $<$https://www.aac-clyde.space/wp-content/uploads/2021/10/AAC\_DataSheet\_Kryten.pdf$>$.  Acesso em: 07 de junho, 2024.

\bibitem{b2} CAPPELLETTI, C.; BATTISTINI, S.; MALPHRUS, B. Cubesat Handbook: From Mission Design to Operations. Editora Elsevier, 2021.

\bibitem{b3} GEORGE, A. D.; WILSON, C. M. Onboard processing with hybrid and reconfigurable computing on small satellites. Proceedings of the IEEE. Institute of Electrical and Electronics Engineers, 2018.

\bibitem{b4} GomSpace. Datasheet: NanoMind A3200. Disponível em $<$https://gomspace.com/UserFiles/Subsystems/datasheet/gs-ds-nanomind-a3200\_1006901-117.pdf$>$. Acesso em: 07 de junho, 2024.

\bibitem{b5} GomSpace. Datasheet: NanoMind HP MK3. Disponível em $<$https://gomspace.com/UserFiles/Subsystems/datasheet/gs-ds-NanoMind\_HP\_MK3.pdf$>$. Acesso em: 07 de junho, 2024.

\bibitem{b6} ISIS Space On Board Computer. Disponível em $<$https://www.isispace.nl/product/on-board-computer/$>$. Acesso em: 07 de junho, 2024.

\bibitem{b7} Nano Avionics. CubeSat On-Board Computer – Main Bus Unit SatBus 3C2. Disponível em $<$https://nanoavionics.com/cubesat-components/cubesat-on-board-computer-main-bus-unit-satbus-3c2/$>$. Acesso em: 07 de junho, 2024.

\bibitem{b8} MARCELINO, G. M. et al. A critical embedded system challenge: The FloripaSat-1 mission. IEEE Latin America Transactions, 2020.

\bibitem{b9} MARCELINO, G. M. et al. FloripaSat-2: An Open-Source Platform for CubeSats. IEEE embedded systems letters, 2024.
\end{thebibliography}
\end{flushleft}


\postextual
%\begin{flushleft}
\begin{thebibliography}{00}
\bibitem{b1} AAC Clyde Space. Datasheet: Kryten-M3. Disponível em $<$https://www.aac-clyde.space/wp-content/uploads/2021/10/AAC\_DataSheet\_Kryten.pdf$>$.  Acesso em: 07 de junho, 2024.

\bibitem{b2} CAPPELLETTI, C.; BATTISTINI, S.; MALPHRUS, B. Cubesat Handbook: From Mission Design to Operations. Editora Elsevier, 2021.

\bibitem{b3} GEORGE, A. D.; WILSON, C. M. Onboard processing with hybrid and reconfigurable computing on small satellites. Proceedings of the IEEE. Institute of Electrical and Electronics Engineers, 2018.

\bibitem{b4} GomSpace. Datasheet: NanoMind A3200. Disponível em $<$https://gomspace.com/UserFiles/Subsystems/datasheet/gs-ds-nanomind-a3200\_1006901-117.pdf$>$. Acesso em: 07 de junho, 2024.

\bibitem{b5} GomSpace. Datasheet: NanoMind HP MK3. Disponível em $<$https://gomspace.com/UserFiles/Subsystems/datasheet/gs-ds-NanoMind\_HP\_MK3.pdf$>$. Acesso em: 07 de junho, 2024.

\bibitem{b6} ISIS Space On Board Computer. Disponível em $<$https://www.isispace.nl/product/on-board-computer/$>$. Acesso em: 07 de junho, 2024.

\bibitem{b7} Nano Avionics. CubeSat On-Board Computer – Main Bus Unit SatBus 3C2. Disponível em $<$https://nanoavionics.com/cubesat-components/cubesat-on-board-computer-main-bus-unit-satbus-3c2/$>$. Acesso em: 07 de junho, 2024.

\bibitem{b8} MARCELINO, G. M. et al. A critical embedded system challenge: The FloripaSat-1 mission. IEEE Latin America Transactions, 2020.

\bibitem{b9} MARCELINO, G. M. et al. FloripaSat-2: An Open-Source Platform for CubeSats. IEEE embedded systems letters, 2024.
\end{thebibliography}
\end{flushleft}


%\include{ref.bib}

\begin{apendicesenv}

% ----------------------------------------------------------
\chapter{Esquemático completo}
% ----------------------------------------------------------



\end{apendicesenv}
% ---
\end{document}