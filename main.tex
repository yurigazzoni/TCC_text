\documentclass[
12pt,				% tamanho da fonte
%openright,		% capítulos começam em pág ímpar (insere página vazia caso preciso)
oneside,			% para impressão no anverso. Oposto a twoside
a4paper,			% tamanho do papel. 
chapter=TITLE,		% títulos de capítulos convertidos em letras maiúsculas
section=TITLE,		% títulos de seções convertidos em letras maiúsculas
%subsection=TITLE,	% títulos de subseções convertidos em letras maiúsculas
%subsubsection=TITLE,% títulos de subsubseções convertidos em letras maiúsculas
% -- opções do pacote babel --
english,			% idioma adicional para hifenização
brazil				% o último idioma é o principal do documento
hyperref=hidelinks]{abntex2}

\usepackage{tccctj} % carrega o estilo CTJ
\bibliographystyle{abntex2-alf-ufsc} % Arquivo bst com patch para correção de citação de proceedings. Produz italico em "In", conforme descrito em: https://github.com/abntex/abntex2/issues/226

% Useful packages
\usepackage{amsmath}
\usepackage{graphicx}
% Pacotes adicionais
\usepackage{multicol}
\usepackage{multirow}
\usepackage{tabularx}
\usepackage{quoting}
\quotingsetup{indentfirst={false},font={footnotesize},leftmargin=4cm,rightmargin=0cm} 
\hypersetup{hidelinks}
\usepackage{enumitem}
\setlist{noitemsep}
%\usepackage{hyperref}
%\usepackage{array,epsfig}
%\usepackage{amsfonts}
%\usepackage{amssymb}
%\usepackage{amsxtra}
%\usepackage{amsthm}
%\usepackage{mathrsfs}
%\usepackage{color}
%\usepackage{indentfirst}
\usepackage{float}
\usepackage{booktabs}
\usepackage{array}
\usepackage{longtable}
\usepackage[nobiblatex]{xurl}

% ----------------------------------------------------

% ----------------------------------------------------
% Informações do trabalho
\autor{Yuri Gazzoni Rezende}
\titulo{OBDH Robusto Para Uso em LEO}
% \subtitulo{Subtítulo} % Preenchimento opcional para quando houver sub-título
\orientador{Gabriel Marcelino}
%\coorientador{Nome do coorientador} % Preenchimento opcional para quando houver co-orientador
\curso{Departamento de Engenharia Elétrica e Eletrônica}
% \titulacao{Bacharel em x}
\datadadefesa{15}{novembro}{2024}
\local{Florianópolis}

\begin{document}

% ----------------------------------------------------
% 1. Capa do trabalho
\imprimircapa

% 2. Folha de rosto
\imprimirfolhaderosto

% 3. Folha de aprovação
\begin{folhadeaprovacao}
    \membrodabanca[Orientador/Presidente]{Prof. Dr. Nome do Orientador/Presidente}{}
    \membrodabanca{Prof. Dr. Membro da banca 1}{UFSC}
    \membrodabanca{Prof. Dr. Membro da banca 2}{UFSC}
    \membrodabanca{Prof. Dr. Membro da banca 3}{UFSC\,-\,Florianópolis}
\end{folhadeaprovacao}

% 4. Dedicatória
\begin{dedicatoria}
    Este trabalho é dedicado aos meus colegas de classe e aos meus queridos pais.
\end{dedicatoria}

% 5. Agradecimentos 
\begin{agradecimentos}
    Inserir os agradecimentos aos colaboradores à execução do trabalho.
\end{agradecimentos}

% 6. Epígrafe
%\begin{epigrafe}
%    \aspas
%    A natureza é um enorme jogo de xadrez disputado por deuses, e que temos o privilégio de observar.
%    As regras do jogo são o que chamamos de física fundamental, e compreender essas regras é a nossa meta.
%    \aspas
%    \autor{Richard Phillips Feynman} % autor da epígrafe
%\end{epigrafe}

% 7. Resumo em português
\begin{resumo}
    O texto do resumo deve ser digitado em um único bloco, sem espaço de parágrafo. Deve ser composto por uma sequência de frases concisas, afirmativas e não de uma enumeração de tópicos. Não deve conter citações. Manter o tempo verbal do texto do trabalho (impessoal) e por vezes usar a voz ativa (Ex.: este trabalho apresenta). O Resumo deve conter: tema, problema, justificativa, objetivos, método e resultados (de forma geral). Abaixo do resumo, informar as palavras-chave (palavras ou expressões significativas retiradas do texto) e preferencialmente não repetir termos do título (para aumentar os indexadores). No mínimo três e no máximo cinco. Separadas por ponto. Observe que o espaçamento aqui, entre linhas, é simples (1,0).

    \palavrachave{Palavra-chave 1}
    \palavrachave{Palavra-chave 2}
    \palavrachave{Palavra-chave 3}
\end{resumo}

% 8a. Resumo em inglês
\begin{resumo}[Abstract]
    \begin{otherlanguage*}{english}
        Resumo traduzido para outros idiomas, neste caso, inglês. Segue o formato do resumo feito na língua vernácula. As palavras-chave traduzidas, versão em língua estrangeira, são colocadas abaixo do texto precedidas pela expressão \emph{Keywords}, separadas por ponto. Observe que o espaçamento aqui, entre linhas, é simples (1,0).
    \end{otherlanguage*}

    \palavrachave{First keyword}
    \palavrachave{Second keyword}
    \palavrachave{Third keyword}
\end{resumo}
% 9. Lista de figuras
\imprimirlistafiguras

% 10. Lista de quadros
%\imprimirlistaquadros

% 11. Lista de tabelas
\imprimirlistatabelas

% 12. Lista de abreviaturas e siglas
\begin{siglas}
	\item[OBDH] \textit{On-board Data Handling} 
\end{siglas}

% 13. Lista de símbolos 

% 14. Sumário
\imprimirsumario

% ----------------------------------------------------

% Elementos textuais
\textual


\chapter{Introdução}
\label{Chap:intro}

CubeSats são pequenos satélites que atendem a estritas formas de cubos padronizados de 10 cm de aresta, além de pesarem menos de 300 kg. Cada um desses cubos padronizados recebe a denominação de 1U, e os tamanhos subsequentes de 1,5U, 2U, 3U, e assim por diante (CUBESAT Design Specification, 2022). Devido a essa padronização e ao uso de componentes comerciais, os CubeSats podem ser produzidos em massa, o que diminui substancialmente os custos de lançamento e desenvolvimento (CubeSat 101, 2017).

O desenvolvimento de satélites de pequeno porte, como CubeSats e nanosatélites, trouxe novas oportunidades e desafios para a indústria espacial, permitindo que uma ampla gama de missões científicas, comerciais e educacionais fosse realizada com custos reduzidos e prazos de desenvolvimento mais curtos (CUBESAT Design Specification, 2022). No entanto, a miniaturização e a operação em ambientes espaciais impõem requisitos à robustez, confiabilidade e versatilidade dos sistemas embarcados, especialmente para os módulos OBDHs (\textit{On-Board Data Handling}). Nesse contexto, a arquitetura de um computador de bordo eficiente e robusto é essencial para o gerenciamento seguro das operações e para garantir a integridade das missões.

Essa segurança e integridade são pontos chave no desenvolvimento de CubeSats no SpaceLab, laboratório da UFSC especializado em desenvolvimento de sistemas espaciais para a comunidade científica e para a indústria. Um dos objetivos primários do SpaceLab é o desenvolvimento de uma plataforma \textit{open-source}, tanto para \textit{software} quanto \textit{hardware}, o que já foi feito nos desenvolvimentos do FloripaSat-1 (MARCELINO et al., 2020) e FloripaSat-2 (MARCELINO et al., 2024). Com esse paradigma, a oportunidade de se ter um computador de bordo mais robusto (com mais memória e capacidade de processamento) e versátil surgiu, como uma consequência direta dos desenvolvimentos das gerações anteriores de \textit{hardware} do laboratório.

O sistema desenvolvido para o presente trabalho foca na implementação de uma arquitetura de processamento e memória capaz de atender às demandas de um satélite de pequeno porte. Esse sistema deve operar de maneira confiável em ambientes suscetíveis à radiação e alta variação de temperatura, em conjunto com a otimização do uso de energia. Além disso, a versatilidade do computador de bordo é essencial para adaptar o sistema a diferentes tipos de missões, desde operações de imagem e telemetria até experimentos científicos em órbita. Para isso, é necessário que o sistema ofereça uma arquitetura versátil, baseada em uma FPGA (\textit{Field-Programmable Gate Array}), com capacidade de expansão e adaptação a novos sensores e módulos de comunicação.

Inicialmente, será feita uma revisão bibliográfica, explorando as características de computadores de bordo comerciais e de trabalhos acadêmicos, além de entender como a radiação em órbita baixa afeta os sistemas eletrônicos. Depois disso, será definida uma arquitetura, respeitando os requisitos impostos, em conjunto com uma estimativa de consumo de potência. Com a arquitetura, será desenvolvido um esquemático, usando o software Altium Designer (versão 24.4.1). Por fim, serão apresentados os resultados obtidos e as considerações finais e conclusões para esse projeto.

\section{Objetivo geral}

O presente trabalho tem como objetivo o projetar e implementação de uma arquitetura de hardware robusta e versátil para um computador de bordo de satélite de pequeno porte, integrando diferentes tipos de memórias e periféricos para assegurar a operação confiável em ambientes espaciais adversos, garantindo a integridade dos dados e a eficiência no gerenciamento dos mesmos.

\section{Objetivos Específicos}

\begin{itemize}
    \item Analisar os requisitos de robustez em condições espaciais, com foco em resistência a radiação, tolerância a falhas e estabilidade térmica, a fim de assegurar o funcionamento contínuo do computador de bordo em órbita.
    \item Especificar uma arquitetura de hardware que permita a adaptação a diferentes tipos de missões, integrando diferentes tipos de componentes comerciais com um SoC (\textit{System-on-a-chip}).
    \item  Documentar as decisões de projeto para consolidar um guia técnico com recomendações de design para sistemas de robustos e versáteis aplicáveis a satélites de pequeno porte, contribuindo para futuras otimizações e adaptações em missões espaciais.
\end{itemize}

% ----------------------------------------------------------
\chapter{Revisão Bibliográfica}
% ----------------------------------------------------------

Para atingir o objetivo de projetar o \textit{hardware} de uma PCB (\textit{Printed Circuit Board}) de um computador de bordo robusto, foi preciso buscar na literatura acadêmica o estado da arte que tange o projeto de OBDHs (\textit{On-Board Data Handling}) para satélites de pequeno porte, especialmente para CubeSats (satélites de baixo custo, dimensões padronizadas e que utilizam componentes comerciais) (CUBESATS ..., 2022).
 
Primeiro, foi necessário um estudo sobre a radiação em LEO - \textit{Low Earth Orbit} (órbitas com raio menor que 1000 km, segundo ESA, 2024), para que a escolha dos componentes do projeto seja a melhor possível. Com esse estudo, buscaram-se formas de mitigar os efeitos mais conhecidos e verificar como instituições têm lidado com componentes do tipo \textit{commercial-off-the-shelf} (COTS).% Além disso, também foi dada a fundamentação dos conversores de potência, tipos de memória, microprocessadores e interfaces de comunicação.

Depois, foram analisadas as placas de OBDH dos projetos do FloripaSat-1 e FloripaSat-2, desenvolvidas pelo SpaceLab da UFSC. Por fim, outros projetos comerciais foram estudados para obtenção de noções sobre a arquitetura e componentes usados. Um panorama geral foi feito, verificando-se principalmente os componentes principais e mais críticos, ou seja, processadores, memórias voláteis e não-voláteis e outros periféricos.

\section{Radiação em LEO e Componentes COTS}

Estando em solo terrestre, os eletrônicos atuais estão bem protegidos contra a maior parte da radiação incidente do universo. No caso dos satélites orbitais, a proteção atmosférica é atenuada pela distância em relação ao solo, mesmo para aqueles que operam em LEO. Nesse caso, a radiação pode ser suficientemente significativa para causar a mudança do comportamento eletromagnético dos materiais, causando efeitos como falhas, aquisição ou execução errada de comandos e distorção da corrente elétrica (MAYANBARI, 2011).  Esses danos são divididos em dois grupos (JUNQUEIRA, 2020): os acumulativos como o TID (\textit{Total Ionizing Dose}), e os SEE (\textit{Single Event Effects}), que indicam o acontecimento de eventos únicos. 

Ainda segundo Junqueira (2020), o TID se caracteriza principalmente pela formação de pares elétron-lacuna, onde o primeiro aumenta a condutividade do material e o segundo contribui para oxidação, mudando as características elétricas do componente com o tempo.  Já os SEE ocorrem quando um íon atravessa um componente crítico, gerando uma linha de ionização que pode ou não ser destrutiva. 

Por esse motivo, quando são escolhidos os componentes críticos para o \textit{hardware} de um \textit{CubeSat}, em sua maioria COTS, deve-se levar em consideração algumas diretrizes cruciais. Segundo Carmo et al. (2021), o componente escolhido precisa atender os requisitos operacionais, concomitante ao gerenciamento de riscos com mitigações e blindagens. 

Com isso, é possível ver três formas confiáveis de escolher cada componente: usando as diretrizes da ESA (\textit{European Space Agency}), as da NASA (\textit{National Aeronautics and Space Administration}) e também através da herança de voo, ou seja, escolhendo componentes que já estiveram em missões semelhantes ou mais críticas. Nos dois primeiros casos, a consulta é através da norma ECSS-Q-ST-60C para a ESA e da lista NPSL (\textit{NASA Part Selection List}) para a NASA.  No caso da herança de voo, outros projetos devem ser analisados e consultados, o que será feito na seção a seguir.

%\section{Conversores de potência}

%Também chamados de conversores DC-DC, são uma peça crucial em um projeto de PCB. Através de uma tensão de entrada, conseguem convertê-la em tensões maiores ou menores, conforme a necessidade do projetista. Existem diferentes tipos de conversores DC-DC, e tratar-se-á apenas dos conversores chaveados \textit{step-up} e \textit{step-down} nessa seção. 

%\subsection{Conversores \textit{step-up}}

%\section{Memória}

%\section{Microprocessadores}

%\section{Interfaces de comunicação}

\section{Projetos Anteriores}

\subsection{FloripaSat-1}
% https://ieeexplore.ieee.org/abstract/document/9085277
O FloripaSat-1 é uma plataforma \textit{open-source} para nanossatélites. Essa plataforma consiste em três módulos: um módulo de fornecimento de potência (EPS), um computador de bordo (OBDH) e um módulo de telemetria e comunicação (TTC).

Seu OBDH foi feito para realização da interface e comunicação entre os módulos e \textit{payloads}. Aqui, destacam-se os sensores presentes: uma \textit{Inertial Measurement Unit} (com giroscópio, magnetômetro e acelerômetro), a interface com os sensores dos painéis solares e as medições de tensão e corrente de entrada do próprio módulo.

Além disso, contava com um microprocessador de 16 bits, memórias flash (IS25LP128) e suporte para cartão microSD para armazenamento.

\subsection{FloripaSat-2}
% https://ieeexplore.ieee.org/abstract/document/10078027
O FloripaSat-2 é a segunda geração da plataforma \textit{open-source} desenvolvida pelo SpaceLab, baseando-se no projeto FloripaSat-1 e trazendo melhorias para os três módulos principais.

Especificamente para o OBDH, foi introduzida uma memória FRAM e uma NOR flash, ao invés das flashs e cartão microSD anteriores, o que mostra uma melhoria clara de capacidade e confiabilidade. 

Outras duas melhorias importantes foram, primeiramente, a adição de um conector para eventualmente conectar uma \textit{daughter board} à placa, e, segundamente, a adição de \textit{buffers} às trilhas de I2C entre os módulos. Isso acrescenta flexibilidade e confiabilidade ao OBDH da segunda geração.

\subsection{Projetos Comerciais}

Abaixo se encontram sintetizados os projetos comerciais estudados, para obtenção de noções sobre a arquitetura e componentes usados. Foram verificados principalmente os processadores, as memórias voláteis e não-voláteis, as interfaces de comunicação e outros periféricos (ADCs, RTC, etc.) utilizados. A Tabela \ref{tab:Tab_Rev} abaixo mostra a pesquisa realizada sobre o Estado da Arte, em conjunto com os dados de George e Wilson (2018), sintetizados na Tabela \ref{tab:Tab_Missoes}.

\begin{table}[htb]
    \centering
	\ABNTEXfontereduzida
	\caption{\label{tab:Tab_Rev}Principais componentes usados pelos fabricantes de OBDHs comerciais.}
	%\begin{tabular}{@{}p{2cm}p{2cm}p{2cm}p{2cm}p{2cm}p{2cm}p{3cm}@{}}
    \begin{tabular}{@{}p{2cm}p{2.6cm}p{2cm}p{2cm}p{2.2cm}p{2.6cm}@{}}
		\toprule
		\textbf{Fabricante} & \textbf{Nome do Produto} & \textbf{Processador} & \textbf{Memórias} & \textbf{Periféricos} & \textbf{Interfaces de comunicação} \\ 
        \midrule
        GomSpace & NanoMind A3200 & AT32UC3C & Flash, SDRAM, FRAM & Giroscópio, Magnetômetro, Transceivers, Sensores de temperatura & CAN, I2C, SPI, JTAG, USART \\%https://gomspace.com/UserFiles/Subsystems/datasheet/gs-ds-nanomind-a3200_1006901-117.pdf
        
        \midrule
        GomSpace & NanoMind HPMK3 & Zynq 7030 & Flash, eMMC, DDR3 & Watchdog, Sensores de temperatura, VCO, Sensores de tensão e corrente & CAN, USART, USB, I2C, JTAG, LVDS, SpaceWire \\ %https://gomspace.com/UserFiles/Subsystems/datasheet/gs-ds-NanoMind_HP_MK3.pdf

        \midrule
        ISIS Space & ISIS On Board Computer & Atmel & Flash, SDRAM, FRAM, Cartões SD & Sensores de temperatura, Sensores de tensão e corrente, RTC, ADC & USART, USB, I2C, JTAG, PWM \\ %https://www.isispace.nl/product/on-board-computer/

        \midrule
        Nano Avionics & SatBus 3C2 & Não informado & Flash, FRAM, Cartões SD & Giroscópio, Magnetômetro, Rádio UHF, ADC & CAN, SPI, I2C, USART, PWM, USB \\ %https://nanoavionics.com/cubesat-components/cubesat-on-board-computer-main-bus-unit-satbus-3c2/

        \midrule
        AAC Clyde Space & Kryten-M3 & Smart Fusion 2 SoC & MRAM, eNVM & RTC, Sensores de tensão e corrente & CAN, SPI, I2C, USART, RS422, LVDS \\ %https://www.aac-clyde.space/what-we-do/space-products-components/command-data-handling/kryten-m3       


		
        \\ \bottomrule
	\end{tabular}
	\fonte{Elaboração própria.}
\end{table}

\begin{table}[H]
	\ABNTEXfontereduzida
	\caption{\label{tab:Tab_Missoes}Síntese da tabela apresentada por George e Wilson (2018).}
	%\begin{tabular}{@{}p{2cm}p{2cm}p{2cm}p{2cm}p{2cm}p{2cm}p{3cm}@{}}
    \centering
    \begin{tabular}{@{} >{\centering}p{3.5cm} >{\centering}p{3.5cm} >{\centering}p{3.5cm} @{}}
    
		\toprule
		\textbf{Fabricante} & \textbf{Processadores} & \textbf{Missões por Fabricante} \tabularnewline 
        \midrule
        Xilinx & Zynq 7020, Zynq 7030, Zynq 7045, Ultrascale+, etc. & 24 \tabularnewline
        
        \midrule
        Atmel + Microchip & ATmega329P, AT91SAM9G20, PIC24F, etc. & 22 \tabularnewline 

        \midrule
        Texas Instruments & MSP430, OMAP3530, Sitara AM3703, etc. & 15 \tabularnewline 

        \midrule
        Cobham Gaisler & GR712RC, UT699, LEON3FT & 8 \tabularnewline
        
        \bottomrule
	\end{tabular}
	\fonte{George e Wilson, 2018, página 463.}
\end{table}

Comparando ambas tabelas, é possível verificar que a maioria dos processadores apresentados são de duas fabricantes: Xilinx (especialmente \textit{chips} da família Zynq 7000) e Microchip (incluindo Atmel). Além disso, a maior parte dos projetos comerciais vistos apresentam memórias FRAM, que possuem um número máximo de ciclos de leitura e escrita muito elevada, além de memórias Flash. Outro destaque foi a presença de sensores de tensão e corrente, bem como magnetômetros e giroscópios.

Com isso, é possível começar a projetar o \textit{hardware} do OBDH, utilizando as diretrizes citadas e as heranças de voo, escolhendo os componentes e respeitando os requisitos impostos para o projeto.







% Citação: 
% https://sci-hub.se/10.1109/jproc.2018.2802438
 


% ----------------------------------------------------------
\chapter{Desenvolvimento do Projeto}
% ----------------------------------------------------------

Após estudar os desdobramentos dos efeitos de órbita baixa e entender o que é necessário para se realizar um projeto confiável de computador de bordo de um nanossatélite, foi necessária a compreensão dos pré-requisitos de projeto. Com isso, foram escolhidos os componentes principais da placa, e a partir deles, um esquemático elétrico foi construído, propondo enfim um hardware confiável, robusto e versátil.  

\section{Pré-Requisitos de Projeto}

Como dito, foi preciso entender os pré-requisitos impostos para o OBDH da terceira geração do SpaceLab. Abaixo, na Tabela \ref{tab:Tab_Req}, se encontram os requisitos gerais do projeto, em conjunto com o pretexto e com o nível de prioridade.

\begin{longtable}{@{}p{5cm}p{5cm}p{3.5cm}@{}}
    \centering
	\ABNTEXfontereduzida
	\label{tab:Tab_Req}\tabularnewline
	\caption{Requisitos do projeto.}\tabularnewline
	%\begin{tabular}{@{}p{2cm}p{2cm}p{2cm}p{2cm}p{2cm}p{2cm}p{3cm}@{}}
	\hline
	\textbf{\centering{Descrição}} & \textbf{\centering{Pretexto}} & \textbf{\centering{Prioridade}} \tabularnewline
        \hline
        O módulo OBDH deve ser compatível com o padrão CubeSat & Assegura compatibilidade com outros satélites desenvolvidos no SpaceLab & Alta \tabularnewline
        
       \hline
        O módulo OBDH deve operar corretamente entre -40°C e 85°C & Para operar com segurança em um ambiente LEO & Alta \tabularnewline

       \hline
        O módulo OBDH deve possuir um microcontrolador capaz de usar um sistema Linux & Para gerenciar e coordenar operações dentro e fora do módulo, sendo capaz de realizar tarefas complexas  & Alta \tabularnewline

       \hline
        O módulo OBDH deve possuir uma memória DDR com capacidade de 512Mb (preferencialmente com ECC)  & Memória suficiente para operações do OBDH e armazenamento de dados  & Alta\tabularnewline

        \hline
        O módulo OBDH deve possuir uma memória FRAM para armazenar parâmetros de configuração & Provê memória não-volátil e duradoura, menos sucetível à radiação & Alta \tabularnewline 

        \hline
        O módulo OBDH deve possuir uma memória Flash para armazenar pacotes (preferencialmente com ECC) & Para armazenar dados e pacotes recebidos & Alta\tabularnewline 

        \hline
        O módulo OBDH deve possuir um WDT para reiniciar o microcontrolador em caso de falha de \textit{software} & Reinicia automaticamente o microcontrolador caso haja a falha  & Alta \tabularnewline

        \hline
        O módulo OBDH deve possuir sensores de medição de tensão e corrente em suas tensões & Para monitoramento de potência consumida & Alta\tabularnewline

        \hline
        O módulo OBDH deve possuir proteção de sobrecorrente (20\% acima do valor nominal) & Para proteção contra \textit{latch-up}  & Alta \tabularnewline

        \hline
        O módulo OBDH deve possuir um giroscópio para medição de velocidade angular & Para permitir controle ativo do satélite  & Alta \tabularnewline 

       \hline
        O módulo OBDH deve possuir um magnetômetro & Para permitir controle ativo do satélite  & Alta \tabularnewline

        \hline
        O módulo OBDH deve possuir uma interface RS-422 para transmissão de mensagens de \textit{debug/log} e receber parâmetros de configuração & Comunicação de longa distância com maior imunidade ao ruído e maior taxa de dados (comparando com UART)  & Alta \tabularnewline

       \hline
        O módulo OBDH deve possuir uma interface CAN para receber e transmitir comandos e dados & Comunicação robusta e com suporte a múltiplos sistemas do CubeSat  & Alta\tabularnewline

        \hline
        O módulo OBDH deve possuir uma interface acessível externamente para programação do microcontrolador & Para o módulo ser facilmente programado pelo time  & Alta \tabularnewline

        \hline
        O módulo OBDH deve possuir uma interface para uma \textit{daughter board} & Para prover suporte a outras interfaces e periféricos  & Baixa \tabularnewline

        \hline
        O módulo OBDH deve possuir um sensor de temperatura com precisão menor ou igual a 1°C & Para prevenir danos de temperaturas extremas & Baixa\tabularnewline

        \hline
        O módulo OBDH deve possuir uma interface RS-485 para receber e transmitir comandos e dados  & Para transmissão robusta de dados com módulos externos & Baixa \tabularnewline
       \hline
	\centering{\fonte{Elaboração própria.}}
\end{longtable}
\begin{flushleft}
\begin{thebibliography}{00} %59
\bibitem{b1} AAC Clyde Space. \textbf{Datasheet: Kryten-M3}. Disponível em $<$https://www.aac-clyde.space/wp-content/uploads/2021/10/AAC\_DataSheet\_Kryten.pdf$>$.  Acesso em: 07 jun. 2024.

\bibitem{b36} \textbf{A3G4250D Gyroscope Datasheet}. Disponível em: $<$\url{https://www.st.com/content/ccc/resource/technical/document/datasheet/5c/f1/a4/70/1b/fa/40/d2/DM00047823.pdf/files/DM00047823.pdf/jcr:content/translations/en.DM00047823.pdf}$>$. Acesso em: 27 out. 2024. 

\bibitem{b21} \textbf{AN-600: Understanding Latch-Up in Advanced CMOS Logic}. [s.l: s.n.]. Disponível em: $<$https://large.stanford.edu/courses/2015/ph241/clark2/docs/AN-600.pdf$>$. Acesso em: 26 out. 2024.

\bibitem{b48} BARLES, A. et al. \textbf{Mission ORCA: Orbit refinement for collision avoidance}. Advances in Astronautics Science and Technology, v. 5, n. 2, p. 149–165, 2022.

%\bibitem{b57} BOING, M. \textbf{Técnicas de mitigação de efeitos da radiação e sua aplicação no projeto de uma arquitetura de hardware para uso em satélites}. Florianópolis, SC: UFSC, 19 de julho de 2024.

\bibitem{b56} BOUKHOBZA, J.; OLIVIER, P. \textbf{Emerging Non-volatile Memories}. Em: Flash Memory Integration. [s.l.] Elsevier, 2017. p. 203–224.

\bibitem{b40} CADENCE PCB SOLUTIONS. \textbf{Zener diode applications: Circuit protection}. Disponível em: <https://resources.pcb.cadence.com/blog/2023-zener-diode-applications-circuit-protection>. Acesso em: 28 out. 2024

\bibitem{b13} CARMO, T. A.; MOREIRA , J. Q.; MANEA, S. \textbf{Análise de blindagem à radiação “TID” e “SEU” em memória do tipo SRAM em orbita LEO (Low Earth Orbit)}. 12° Workshop em Engenharia e Tecnologia Espaciais, 6 nov. 2021.

\bibitem{b49} CAMPS, A. et al. \textbf{Fsscat, the 2017 Copernicus masters’ “Esa sentinel small satellite challenge” winner: A federated polar and soil moisture tandem mission based on 6U cubesats}. IGARSS 2018 - 2018 IEEE International Geoscience and Remote Sensing Symposium. Anais...IEEE, 2018.

\bibitem{b2} CAPPELLETTI, C.; BATTISTINI, S.; MALPHRUS, B. \textbf{Cubesat Handbook: From Mission Design to Operations}. Editora Elsevier, 2021.

\bibitem{b58} \textit{CubeSat101: Basic Concepts and Processes for First-Time CubeSat Developers}. [S.l.: s.n.], 2017. Disponível em $<$https://www.nasa.gov/wp-content/uploads/2017/03/nasa\_csli\_cubesat\_101\_508.pdf?emrc=05d3e2$>$. Acesso em 28 out. 2024.

\bibitem{b15} \textbf{CUBESAT Design Specification}. [S.l.: s.n.], 2022. Disponível em $<$https://www.cubesat.org/cubesatinfo$>$. Acesso em: 25 out. 2024.

\bibitem{b22} \textbf{CY15B104QN FRAM Datasheet}. Disponível em: $<$\url{https://www.infineon.com/dgdl/Infineon-CY15B104QN\_CY15V104QN\_Excelon(TM)\_LP\_4-Mbit\_(512K\_X\_8)\_Serial\_(SPI)\_F-RAM-DataSheet-v12\_00-EN.pdf?fileId=8ac78c8c7d0d8da4017d0ee7709b704a\&utm\_source=cypress\&utm\_medium=referral\&utm\_campaign=202110\_globe\_en\_all\_integr}$>$. Acesso em: 27 out. 2024.

\bibitem{b47} \textbf{DS191 - Zynq-7000 SoC (Z-7030, Z-7035, Z-7045, and Z-7100): DC and AC Switching Characteristics}. 2018. Disponível em $<$https://docs.amd.com/v/u/en-US/ds191-XC7Z030-XC7Z045-data-sheet$>$. Acesso em: 29 out. 2024.

\bibitem{b14} ECSS. \textbf{ECSS-Q-ST-60C Rev.3 – Electrical, electronic and electromechanical (EEE) components (12 May 2022)}. Disponível em: $<$https://ecss.nl/standard/ecss-q-st-60c-rev-3-electrical-electronic-and-electromechanical-eee-components-2-may-2022/$>$. Acesso em: 6 out. 2024.

\bibitem{b57} ECSS. \textbf{ECSS-E-ST-10-04C - Space Environment}. The Netherlands: [s.n.], 2008.

\bibitem{b20} GERARDIN, S.; PACCAGNELLA, A. \textbf{Present and future non-volatile memories for space}. IEEE transactions on nuclear science, 2010.

\bibitem{b3} GEORGE, A. D.; WILSON, C. M. \textbf{Onboard processing with hybrid and reconfigurable computing on small satellites}. Proceedings of the IEEE. Institute of Electrical and Electronics Engineers, 2018.

\bibitem{b4} \textbf{Gomspace NanoMind A3200 Datasheet}. Disponível em $<$https://gomspace.com/UserFiles/Subsystems/datasheet/gs-ds-nanomind-a3200\_1006901-117.pdf$>$. Acesso em: 07 jun. 2024.

\bibitem{b5} \textbf{Gomspace NanoMind HP MK3 Datasheet}. Disponível em $<$https://gomspace.com/UserFiles/Subsystems/datasheet/gs-ds-NanoMind\_HP\_MK3.pdf$>$. Acesso em: 07 jun. 2024.

\bibitem{b51} \textbf{Gomspace Nanomind Z7000 Datasheet}. 2019. Disponível em: $<$https://gomspace.com/UserFiles/Subsystems/datasheet/gs-ds-nanomind-z7000-15.pdf$>$. Acesso em: 30 oct. 2024.

\bibitem{b34} \textbf{INA180A2IDBVR Current Sense Datasheet}. Disponível em: $<$\url{https://www.ti.com/lit/ds/symlink/ina180.pdf?HQS=dis-dk-null-digikeymode-dsf-pf-null-wwe\&ts=1727202734954\&ref\_url=https\%253A\%252F\%252Fwww.ti.com\%252Fgeneral\%252Fdocs\%252Fsuppproductinfo.tsp\%253FdistId\%253D10\%2526gotoUrl\%253Dhttps\%253A\%252F\%252Fwww.ti.com\%252Flit\%252Fgpn\%252Fina180}$>$. Acesso em: 27 out. 2024. 

\bibitem{b6} \textbf{ISIS Space On Board Computer}. Disponível em $<$https://www.isispace.nl/product/on-board-computer/$>$. Acesso em: 07 jun. 2024.

\bibitem{b19} JEDEC. \textbf{Double Data Rate (DDR) SDRAM Standard}. 2008. Disponível em: $<$https://www.jedec.org/standards-documents/docs/jesd-79f$>$. Acesso em: 26 out. 2024.

\bibitem{b10} JUNQUEIRA, B. C.; MANEA, S.\textbf{Utilização de COTS em nano satélites}. Brazilian Journal of Development, v. 6, n. 1, p. 1476-1490, 2020.

\bibitem{b16} KADI, M. A. et al. \textbf{Dynamic and partial reconfiguration of Zynq 7000 under Linux}. 2013 International Conference on Reconfigurable Computing and FPGAs (ReConFig). Anais...IEEE, 2013.

\bibitem{b18} KLEHN, B.; BROX, M. A. \textbf{Comparison of current SDRAM types: SDR, DDR, and RDRAM}. Advances in radio science, v. 1, p. 265–271, 2003.

\bibitem{b59} LABEL, K. A. \textbf{Radiation Effects on Electronics 101: Simple Concepts and New Challenges}. NEPP Webex Presentation, 2004.

\bibitem{b54} LOFFLER, T. et al. \textbf{Research and Observation in Medium Earth Orbit (ROMEO) with a cost-effective microsatellite platform}. 72nd International Astronautical Congress (IAC), Dubai, United Arab Emirates. 2021. 

\bibitem{b12} \textbf{Low earth orbit}. Disponível em: $<$https://www.esa.int/ESA\_Multimedia/Images/2020/03/Low\_Earth\_orbit$>$. Acesso em: 6 out. 2024.

\bibitem{b27} \textbf{LTC2991IMS\#TRPBF Voltage, Current and Temperature Monitor Datasheet}. Disponível em: $<$https://www.analog.com/media/en/technical-documentation/data-sheets/2991ff.pdf$>$. Acesso em: 27 out. 2024. 

\bibitem{b37} \textbf{LTC4361 Overcurrent Protection Controller Datasheet}. 2018. Disponível em: $<$https://www.analog.com/media/en/technical-documentation/data-sheets/LTC4361-1-4361-2.pdf$>$. Acesso em: 27 out. 2024. 

\bibitem{b11} MAYANBARI, Masood; KASESAZ, Yaser. \textbf{Design and analyse space radiation shielding for a nanosatellite in Low Earth Orbit (LEO)}. In: Proceedings of 5th International Conference on Recent Advances in Space Technologies-RAST2011. IEEE, 2011. p. 489-493.

\bibitem{b42} MAK, B. \textbf{Basics of Load Switches}. 2018. Disponível em: <https://www.ti.com/lit/an/slva652a/slva652a.pdf?ts=1730079105016>. Acesso em: 28 out. 2024.

\bibitem{b8} MARCELINO, G. M. et al. \textbf{A critical embedded system challenge: The FloripaSat-1 mission}. IEEE Latin America Transactions, 2020.

\bibitem{b9} MARCELINO, G. M. et al. \textbf{FloripaSat-2: An Open-Source Platform for CubeSats}. IEEE embedded systems letters, 2024.

\bibitem{b35} \textbf{MMC5983MA Magnetometer Datasheet}. Disponível em: $<$\url{https://mm.digikey.com/Volume0/opasdata/d220001/medias/docus/333/MMC5983MA\_RevA\_4-3-19.pdf}$>$. Acesso em: 27 out. 2024. 

\bibitem{b25} \textbf{MT25QL128ABB1ESE-0AUT Flash NOR Datasheet}. Disponível em: $<$https://www.micron.com/content/dam/micron/global/secure/products/data-sheet/nor-flash/serial-nor/mt25q/die-rev-b/mt25q-qlht-l-128-abb-xxt.pdf$>$. Acesso em: 27 out. 2024. 

\bibitem{b24} \textbf{MT29F1G01ABAFDSF-AAT:F Flash NAND Datasheet}. Disponível em: $<$https://www.micron.com/content/dam/micron/global/secure/products/data-sheet/nand-flash/70-series/m78a-1gb-spi-auto.pdf$>$. Acesso em: 27 out. 2024. 

\bibitem{b23} \textbf{MT41K256M8DA-125:K DDR3L Datasheet}. Disponível em: $<$https://www.micron.com/content/dam/micron/global/secure/products/data-sheet/dram/ddr3/2gb-1-35v-ddr3l.pdf$>$. Acesso em: 27 out. 2024. 

\bibitem{b7} Nano Avionics. \textbf{CubeSat On-Board Computer – Main Bus Unit SatBus 3C2}. Disponível em $<$https://nanoavionics.com/cubesat-components/cubesat-on-board-computer-main-bus-unit-satbus-3c2/$>$. Acesso em: 07 jun. 2024.

\bibitem{b13} \textbf{NASA Parts Selection List (NPSL)}. Disponível em: $<$https://nepp.nasa.gov/npsl/$>$. Acesso em: 6 out. 2024.

\bibitem{b55} \textbf{NASA Product Verification}. Disponível em: $<$https://www.nasa.gov/reference/5-3-product-verification/$>$. Acesso em: 30 out. 2024.

\bibitem{b53} PUTRA, A. C. A. Y.; WIJANTO, H.; EDWAR. \textbf{Design and implementation RTOS (real time operating system) as a nano satellite control for responding to space environmental conditions}. 2021 IEEE Asia Pacific Conference on Wireless and Mobile (APWiMob). Anais...IEEE, 2021.

\bibitem{b41} SARJEANT, W. \textbf{Capacitors}. IEEE transactions on electrical insulation, v. 25, n. 5, p. 861–922, 1990.

\bibitem{b39} SOH, W.-S. et al. \textbf{Filter design for suppression of noise coupling from PCB to DC power supply}. 2010 Asia-Pacific International Symposium on Electromagnetic Compatibility. Anais...IEEE, 2010.

\bibitem{b28} \textbf{TCA4311ADR I2C Buffer Datasheet}. Disponível em: $<$\url{https://www.ti.com/general/docs/suppproductinfo.tsp?distId=10\&gotoUrl=https\%3A\%2F\%2Fwww.ti.com\%2Flit\%2Fgpn\%2Ftca4311a}$>$. Acesso em: 27 out. 2024. 

\bibitem{b33} \textbf{TCAN330D CAN Transceiver Datasheet}. Disponível em: $<$\url{https://www.ti.com/lit/ds/symlink/tcan330.pdf?ts=1729084231140\&ref\_url=https\%253A\%252F\%252Fbr.mouser.com\%252F}$>$. Acesso em: 27 out. 2024. 

\bibitem{b29} \textbf{THVD1451DR RS-485 Transceiver Datasheet}. Disponível em: $<$\url{https://www.ti.com/lit/ds/symlink/thvd1410.pdf?HQS=dis-dk-null-digikeymode-dsf-pf-null-wwe\&ts=1722094226249\&ref\_url=https\%253A\%252F\%252Fwww.ti.com\%252Fgeneral\%252Fdocs\%252Fsuppproductinfo.tsp\%253FdistId\%253D10\%2526gotoUrl\%253Dhttps\%253A\%252F\%252Fwww.ti.com\%252Flit\%252Fgpn\%252Fthvd1410}$>$. Acesso em: 27 out. 2024. 

\bibitem{b30} \textbf{TPS22920YZPR Load Switch Datasheet}. 2016. Disponível em: $<$\url{https://www.ti.com/general/docs/suppproductinfo.tsp?distId=10\&gotoUrl=https\%3A\%2F\%2Fwww.ti.com\%2Flit\%2Fgpn\%2Ftps22920}$>$. Acesso em: 27 out. 2024. 

\bibitem{b26} \textbf{TPS3823-33QDBVRQ1 Watchdog Timer Datasheet}. Disponível em: $<$\url{https://www.ti.com/general/docs/suppproductinfo.tsp?distId=10\&gotoUrl=https\%3A\%2F\%2Fwww.ti.com\%2Flit\%2Fgpn\%2Ftps3828-q1}$>$. Acesso em: 27 out. 2024. 

\bibitem{b31} \textbf{TPS51200DRCR DC-DC Voltage Regulator Datasheet}. Disponível em: $<$\url{https://www.ti.com/general/docs/suppproductinfo.tsp?distId=10\&gotoUrl=https\%3A\%2F\%2Fwww.ti.com\%2Flit\%2Fgpn\%2Ftps51200}$>$. Acesso em: 27 out. 2024. 

\bibitem{b32} \textbf{TPS82085SILR DC-DC Voltage Regulator Datasheet}. 2019. Disponível em: $<$\url{https://www.ti.com/general/docs/suppproductinfo.tsp?distId=10\&gotoUrl=https\%3A\%2F\%2Fwww.ti.com\%2Flit\%2Fgpn\%2Ftps82085}$>$. Acesso em: 27 out. 2024. 

\bibitem{b44} \textbf{UG470 - 7 Series FPGAs Configuration User Guide}. 2023. Disponível em $<$https://docs.amd.com/v/u/en-US/ug470\_7Series\_Config$>$. Acesso em: 29 out. 2024.

\bibitem{b45} \textbf{UG480 - 7 Series FPGAs and Zynq-7000 SoC XADC Dual 12-Bit 1 MSPS Analog-to-Digital Converter User Guide}. 2022. Disponível em $<$https://docs.amd.com/r/en-US/ug480\_7Series\_XADC$>$. Acesso em: 29 out. 2024.

\bibitem{b38} \textbf{UG585 - Zynq 7000 SoC Technical Reference Manual - AMD technical information portal}. 2023. Disponível em: $<$https://docs.amd.com/r/en-US/ug585-zynq-7000-SoC-TRM/Zynq-7000-SoC-Technical-Reference-Manual$>$. Acesso em: 25 out. 2024.

\bibitem{b17} \textbf{UG865 - Zynq-7000 SoC Packaging and Pinout Product Specification - AMD technical information portal}. 2021. Disponível em: $<$https://docs.amd.com/v/u/en-US/ug865-Zynq-7000-Pkg-Pinout$>$. Acesso em: 25 out. 2024.

\bibitem{b46} \textbf{UG933 - Zynq-7000 SoC PCB Design Guide}. 2019. Disponível em $<$https://docs.amd.com/v/u/en-US/ug933-Zynq-7000-PCB$>$. Acesso em: 29 out. 2024.

\bibitem{b50} WEIDMANN, D. et al. \textbf{Cubesats for monitoring atmospheric processes (CubeMAP): a constellation mission to study the middle atmosphere}. Sensors, Systems, and Next-Generation Satellites XXIV. Anais...SPIE, 2020.

\bibitem{b43} \textbf{XPE - Power Estimator}. 2019. Disponível em: $<$https://www.amd.com/en/products/adaptive-socs-and-fpgas/technologies/power-efficiency/power-estimator.html$>$. Acesso em: 28 out. 2024.

\bibitem{b52} ZHOU, Q. et al. \textbf{Design of a compact and reconfigurable onboard data handling system}. 2018 IEEE Intl Conf on Parallel \& Distributed Processing with Applications, Ubiquitous Computing \& Communications, Big Data \& Cloud Computing, Social Computing \& Networking, Sustainable Computing \& Communications (ISPA/IUCC/BDCloud/SocialCom/SustainCom). Anais...IEEE, 2018.

\end{thebibliography}
\end{flushleft}


\postextual
%\begin{flushleft}
\begin{thebibliography}{00} %59
\bibitem{b1} AAC Clyde Space. \textbf{Datasheet: Kryten-M3}. Disponível em $<$https://www.aac-clyde.space/wp-content/uploads/2021/10/AAC\_DataSheet\_Kryten.pdf$>$.  Acesso em: 07 jun. 2024.

\bibitem{b36} \textbf{A3G4250D Gyroscope Datasheet}. Disponível em: $<$\url{https://www.st.com/content/ccc/resource/technical/document/datasheet/5c/f1/a4/70/1b/fa/40/d2/DM00047823.pdf/files/DM00047823.pdf/jcr:content/translations/en.DM00047823.pdf}$>$. Acesso em: 27 out. 2024. 

\bibitem{b21} \textbf{AN-600: Understanding Latch-Up in Advanced CMOS Logic}. [s.l: s.n.]. Disponível em: $<$https://large.stanford.edu/courses/2015/ph241/clark2/docs/AN-600.pdf$>$. Acesso em: 26 out. 2024.

\bibitem{b48} BARLES, A. et al. \textbf{Mission ORCA: Orbit refinement for collision avoidance}. Advances in Astronautics Science and Technology, v. 5, n. 2, p. 149–165, 2022.

%\bibitem{b57} BOING, M. \textbf{Técnicas de mitigação de efeitos da radiação e sua aplicação no projeto de uma arquitetura de hardware para uso em satélites}. Florianópolis, SC: UFSC, 19 de julho de 2024.

\bibitem{b56} BOUKHOBZA, J.; OLIVIER, P. \textbf{Emerging Non-volatile Memories}. Em: Flash Memory Integration. [s.l.] Elsevier, 2017. p. 203–224.

\bibitem{b40} CADENCE PCB SOLUTIONS. \textbf{Zener diode applications: Circuit protection}. Disponível em: <https://resources.pcb.cadence.com/blog/2023-zener-diode-applications-circuit-protection>. Acesso em: 28 out. 2024

\bibitem{b13} CARMO, T. A.; MOREIRA , J. Q.; MANEA, S. \textbf{Análise de blindagem à radiação “TID” e “SEU” em memória do tipo SRAM em orbita LEO (Low Earth Orbit)}. 12° Workshop em Engenharia e Tecnologia Espaciais, 6 nov. 2021.

\bibitem{b49} CAMPS, A. et al. \textbf{Fsscat, the 2017 Copernicus masters’ “Esa sentinel small satellite challenge” winner: A federated polar and soil moisture tandem mission based on 6U cubesats}. IGARSS 2018 - 2018 IEEE International Geoscience and Remote Sensing Symposium. Anais...IEEE, 2018.

\bibitem{b2} CAPPELLETTI, C.; BATTISTINI, S.; MALPHRUS, B. \textbf{Cubesat Handbook: From Mission Design to Operations}. Editora Elsevier, 2021.

\bibitem{b58} \textit{CubeSat101: Basic Concepts and Processes for First-Time CubeSat Developers}. [S.l.: s.n.], 2017. Disponível em $<$https://www.nasa.gov/wp-content/uploads/2017/03/nasa\_csli\_cubesat\_101\_508.pdf?emrc=05d3e2$>$. Acesso em 28 out. 2024.

\bibitem{b15} \textbf{CUBESAT Design Specification}. [S.l.: s.n.], 2022. Disponível em $<$https://www.cubesat.org/cubesatinfo$>$. Acesso em: 25 out. 2024.

\bibitem{b22} \textbf{CY15B104QN FRAM Datasheet}. Disponível em: $<$\url{https://www.infineon.com/dgdl/Infineon-CY15B104QN\_CY15V104QN\_Excelon(TM)\_LP\_4-Mbit\_(512K\_X\_8)\_Serial\_(SPI)\_F-RAM-DataSheet-v12\_00-EN.pdf?fileId=8ac78c8c7d0d8da4017d0ee7709b704a\&utm\_source=cypress\&utm\_medium=referral\&utm\_campaign=202110\_globe\_en\_all\_integr}$>$. Acesso em: 27 out. 2024.

\bibitem{b47} \textbf{DS191 - Zynq-7000 SoC (Z-7030, Z-7035, Z-7045, and Z-7100): DC and AC Switching Characteristics}. 2018. Disponível em $<$https://docs.amd.com/v/u/en-US/ds191-XC7Z030-XC7Z045-data-sheet$>$. Acesso em: 29 out. 2024.

\bibitem{b14} ECSS. \textbf{ECSS-Q-ST-60C Rev.3 – Electrical, electronic and electromechanical (EEE) components (12 May 2022)}. Disponível em: $<$https://ecss.nl/standard/ecss-q-st-60c-rev-3-electrical-electronic-and-electromechanical-eee-components-2-may-2022/$>$. Acesso em: 6 out. 2024.

\bibitem{b57} ECSS. \textbf{ECSS-E-ST-10-04C - Space Environment}. The Netherlands: [s.n.], 2008.

\bibitem{b20} GERARDIN, S.; PACCAGNELLA, A. \textbf{Present and future non-volatile memories for space}. IEEE transactions on nuclear science, 2010.

\bibitem{b3} GEORGE, A. D.; WILSON, C. M. \textbf{Onboard processing with hybrid and reconfigurable computing on small satellites}. Proceedings of the IEEE. Institute of Electrical and Electronics Engineers, 2018.

\bibitem{b4} \textbf{Gomspace NanoMind A3200 Datasheet}. Disponível em $<$https://gomspace.com/UserFiles/Subsystems/datasheet/gs-ds-nanomind-a3200\_1006901-117.pdf$>$. Acesso em: 07 jun. 2024.

\bibitem{b5} \textbf{Gomspace NanoMind HP MK3 Datasheet}. Disponível em $<$https://gomspace.com/UserFiles/Subsystems/datasheet/gs-ds-NanoMind\_HP\_MK3.pdf$>$. Acesso em: 07 jun. 2024.

\bibitem{b51} \textbf{Gomspace Nanomind Z7000 Datasheet}. 2019. Disponível em: $<$https://gomspace.com/UserFiles/Subsystems/datasheet/gs-ds-nanomind-z7000-15.pdf$>$. Acesso em: 30 oct. 2024.

\bibitem{b34} \textbf{INA180A2IDBVR Current Sense Datasheet}. Disponível em: $<$\url{https://www.ti.com/lit/ds/symlink/ina180.pdf?HQS=dis-dk-null-digikeymode-dsf-pf-null-wwe\&ts=1727202734954\&ref\_url=https\%253A\%252F\%252Fwww.ti.com\%252Fgeneral\%252Fdocs\%252Fsuppproductinfo.tsp\%253FdistId\%253D10\%2526gotoUrl\%253Dhttps\%253A\%252F\%252Fwww.ti.com\%252Flit\%252Fgpn\%252Fina180}$>$. Acesso em: 27 out. 2024. 

\bibitem{b6} \textbf{ISIS Space On Board Computer}. Disponível em $<$https://www.isispace.nl/product/on-board-computer/$>$. Acesso em: 07 jun. 2024.

\bibitem{b19} JEDEC. \textbf{Double Data Rate (DDR) SDRAM Standard}. 2008. Disponível em: $<$https://www.jedec.org/standards-documents/docs/jesd-79f$>$. Acesso em: 26 out. 2024.

\bibitem{b10} JUNQUEIRA, B. C.; MANEA, S.\textbf{Utilização de COTS em nano satélites}. Brazilian Journal of Development, v. 6, n. 1, p. 1476-1490, 2020.

\bibitem{b16} KADI, M. A. et al. \textbf{Dynamic and partial reconfiguration of Zynq 7000 under Linux}. 2013 International Conference on Reconfigurable Computing and FPGAs (ReConFig). Anais...IEEE, 2013.

\bibitem{b18} KLEHN, B.; BROX, M. A. \textbf{Comparison of current SDRAM types: SDR, DDR, and RDRAM}. Advances in radio science, v. 1, p. 265–271, 2003.

\bibitem{b59} LABEL, K. A. \textbf{Radiation Effects on Electronics 101: Simple Concepts and New Challenges}. NEPP Webex Presentation, 2004.

\bibitem{b54} LOFFLER, T. et al. \textbf{Research and Observation in Medium Earth Orbit (ROMEO) with a cost-effective microsatellite platform}. 72nd International Astronautical Congress (IAC), Dubai, United Arab Emirates. 2021. 

\bibitem{b12} \textbf{Low earth orbit}. Disponível em: $<$https://www.esa.int/ESA\_Multimedia/Images/2020/03/Low\_Earth\_orbit$>$. Acesso em: 6 out. 2024.

\bibitem{b27} \textbf{LTC2991IMS\#TRPBF Voltage, Current and Temperature Monitor Datasheet}. Disponível em: $<$https://www.analog.com/media/en/technical-documentation/data-sheets/2991ff.pdf$>$. Acesso em: 27 out. 2024. 

\bibitem{b37} \textbf{LTC4361 Overcurrent Protection Controller Datasheet}. 2018. Disponível em: $<$https://www.analog.com/media/en/technical-documentation/data-sheets/LTC4361-1-4361-2.pdf$>$. Acesso em: 27 out. 2024. 

\bibitem{b11} MAYANBARI, Masood; KASESAZ, Yaser. \textbf{Design and analyse space radiation shielding for a nanosatellite in Low Earth Orbit (LEO)}. In: Proceedings of 5th International Conference on Recent Advances in Space Technologies-RAST2011. IEEE, 2011. p. 489-493.

\bibitem{b42} MAK, B. \textbf{Basics of Load Switches}. 2018. Disponível em: <https://www.ti.com/lit/an/slva652a/slva652a.pdf?ts=1730079105016>. Acesso em: 28 out. 2024.

\bibitem{b8} MARCELINO, G. M. et al. \textbf{A critical embedded system challenge: The FloripaSat-1 mission}. IEEE Latin America Transactions, 2020.

\bibitem{b9} MARCELINO, G. M. et al. \textbf{FloripaSat-2: An Open-Source Platform for CubeSats}. IEEE embedded systems letters, 2024.

\bibitem{b35} \textbf{MMC5983MA Magnetometer Datasheet}. Disponível em: $<$\url{https://mm.digikey.com/Volume0/opasdata/d220001/medias/docus/333/MMC5983MA\_RevA\_4-3-19.pdf}$>$. Acesso em: 27 out. 2024. 

\bibitem{b25} \textbf{MT25QL128ABB1ESE-0AUT Flash NOR Datasheet}. Disponível em: $<$https://www.micron.com/content/dam/micron/global/secure/products/data-sheet/nor-flash/serial-nor/mt25q/die-rev-b/mt25q-qlht-l-128-abb-xxt.pdf$>$. Acesso em: 27 out. 2024. 

\bibitem{b24} \textbf{MT29F1G01ABAFDSF-AAT:F Flash NAND Datasheet}. Disponível em: $<$https://www.micron.com/content/dam/micron/global/secure/products/data-sheet/nand-flash/70-series/m78a-1gb-spi-auto.pdf$>$. Acesso em: 27 out. 2024. 

\bibitem{b23} \textbf{MT41K256M8DA-125:K DDR3L Datasheet}. Disponível em: $<$https://www.micron.com/content/dam/micron/global/secure/products/data-sheet/dram/ddr3/2gb-1-35v-ddr3l.pdf$>$. Acesso em: 27 out. 2024. 

\bibitem{b7} Nano Avionics. \textbf{CubeSat On-Board Computer – Main Bus Unit SatBus 3C2}. Disponível em $<$https://nanoavionics.com/cubesat-components/cubesat-on-board-computer-main-bus-unit-satbus-3c2/$>$. Acesso em: 07 jun. 2024.

\bibitem{b13} \textbf{NASA Parts Selection List (NPSL)}. Disponível em: $<$https://nepp.nasa.gov/npsl/$>$. Acesso em: 6 out. 2024.

\bibitem{b55} \textbf{NASA Product Verification}. Disponível em: $<$https://www.nasa.gov/reference/5-3-product-verification/$>$. Acesso em: 30 out. 2024.

\bibitem{b53} PUTRA, A. C. A. Y.; WIJANTO, H.; EDWAR. \textbf{Design and implementation RTOS (real time operating system) as a nano satellite control for responding to space environmental conditions}. 2021 IEEE Asia Pacific Conference on Wireless and Mobile (APWiMob). Anais...IEEE, 2021.

\bibitem{b41} SARJEANT, W. \textbf{Capacitors}. IEEE transactions on electrical insulation, v. 25, n. 5, p. 861–922, 1990.

\bibitem{b39} SOH, W.-S. et al. \textbf{Filter design for suppression of noise coupling from PCB to DC power supply}. 2010 Asia-Pacific International Symposium on Electromagnetic Compatibility. Anais...IEEE, 2010.

\bibitem{b28} \textbf{TCA4311ADR I2C Buffer Datasheet}. Disponível em: $<$\url{https://www.ti.com/general/docs/suppproductinfo.tsp?distId=10\&gotoUrl=https\%3A\%2F\%2Fwww.ti.com\%2Flit\%2Fgpn\%2Ftca4311a}$>$. Acesso em: 27 out. 2024. 

\bibitem{b33} \textbf{TCAN330D CAN Transceiver Datasheet}. Disponível em: $<$\url{https://www.ti.com/lit/ds/symlink/tcan330.pdf?ts=1729084231140\&ref\_url=https\%253A\%252F\%252Fbr.mouser.com\%252F}$>$. Acesso em: 27 out. 2024. 

\bibitem{b29} \textbf{THVD1451DR RS-485 Transceiver Datasheet}. Disponível em: $<$\url{https://www.ti.com/lit/ds/symlink/thvd1410.pdf?HQS=dis-dk-null-digikeymode-dsf-pf-null-wwe\&ts=1722094226249\&ref\_url=https\%253A\%252F\%252Fwww.ti.com\%252Fgeneral\%252Fdocs\%252Fsuppproductinfo.tsp\%253FdistId\%253D10\%2526gotoUrl\%253Dhttps\%253A\%252F\%252Fwww.ti.com\%252Flit\%252Fgpn\%252Fthvd1410}$>$. Acesso em: 27 out. 2024. 

\bibitem{b30} \textbf{TPS22920YZPR Load Switch Datasheet}. 2016. Disponível em: $<$\url{https://www.ti.com/general/docs/suppproductinfo.tsp?distId=10\&gotoUrl=https\%3A\%2F\%2Fwww.ti.com\%2Flit\%2Fgpn\%2Ftps22920}$>$. Acesso em: 27 out. 2024. 

\bibitem{b26} \textbf{TPS3823-33QDBVRQ1 Watchdog Timer Datasheet}. Disponível em: $<$\url{https://www.ti.com/general/docs/suppproductinfo.tsp?distId=10\&gotoUrl=https\%3A\%2F\%2Fwww.ti.com\%2Flit\%2Fgpn\%2Ftps3828-q1}$>$. Acesso em: 27 out. 2024. 

\bibitem{b31} \textbf{TPS51200DRCR DC-DC Voltage Regulator Datasheet}. Disponível em: $<$\url{https://www.ti.com/general/docs/suppproductinfo.tsp?distId=10\&gotoUrl=https\%3A\%2F\%2Fwww.ti.com\%2Flit\%2Fgpn\%2Ftps51200}$>$. Acesso em: 27 out. 2024. 

\bibitem{b32} \textbf{TPS82085SILR DC-DC Voltage Regulator Datasheet}. 2019. Disponível em: $<$\url{https://www.ti.com/general/docs/suppproductinfo.tsp?distId=10\&gotoUrl=https\%3A\%2F\%2Fwww.ti.com\%2Flit\%2Fgpn\%2Ftps82085}$>$. Acesso em: 27 out. 2024. 

\bibitem{b44} \textbf{UG470 - 7 Series FPGAs Configuration User Guide}. 2023. Disponível em $<$https://docs.amd.com/v/u/en-US/ug470\_7Series\_Config$>$. Acesso em: 29 out. 2024.

\bibitem{b45} \textbf{UG480 - 7 Series FPGAs and Zynq-7000 SoC XADC Dual 12-Bit 1 MSPS Analog-to-Digital Converter User Guide}. 2022. Disponível em $<$https://docs.amd.com/r/en-US/ug480\_7Series\_XADC$>$. Acesso em: 29 out. 2024.

\bibitem{b38} \textbf{UG585 - Zynq 7000 SoC Technical Reference Manual - AMD technical information portal}. 2023. Disponível em: $<$https://docs.amd.com/r/en-US/ug585-zynq-7000-SoC-TRM/Zynq-7000-SoC-Technical-Reference-Manual$>$. Acesso em: 25 out. 2024.

\bibitem{b17} \textbf{UG865 - Zynq-7000 SoC Packaging and Pinout Product Specification - AMD technical information portal}. 2021. Disponível em: $<$https://docs.amd.com/v/u/en-US/ug865-Zynq-7000-Pkg-Pinout$>$. Acesso em: 25 out. 2024.

\bibitem{b46} \textbf{UG933 - Zynq-7000 SoC PCB Design Guide}. 2019. Disponível em $<$https://docs.amd.com/v/u/en-US/ug933-Zynq-7000-PCB$>$. Acesso em: 29 out. 2024.

\bibitem{b50} WEIDMANN, D. et al. \textbf{Cubesats for monitoring atmospheric processes (CubeMAP): a constellation mission to study the middle atmosphere}. Sensors, Systems, and Next-Generation Satellites XXIV. Anais...SPIE, 2020.

\bibitem{b43} \textbf{XPE - Power Estimator}. 2019. Disponível em: $<$https://www.amd.com/en/products/adaptive-socs-and-fpgas/technologies/power-efficiency/power-estimator.html$>$. Acesso em: 28 out. 2024.

\bibitem{b52} ZHOU, Q. et al. \textbf{Design of a compact and reconfigurable onboard data handling system}. 2018 IEEE Intl Conf on Parallel \& Distributed Processing with Applications, Ubiquitous Computing \& Communications, Big Data \& Cloud Computing, Social Computing \& Networking, Sustainable Computing \& Communications (ISPA/IUCC/BDCloud/SocialCom/SustainCom). Anais...IEEE, 2018.

\end{thebibliography}
\end{flushleft}


%\include{ref.bib}

\end{document}