% ----------------------------------------------------------
\chapter{Arquitetura}
% ----------------------------------------------------------

Após estudar os desdobramentos dos efeitos de órbita baixa e entender o que é necessário para se realizar um projeto confiável de computador de bordo de um nanossatélite, foi necessária a compreensão dos pré-requisitos de projeto. Com isso, foram escolhidos os componentes principais da placa, propondo-se uma arquitetura para o sistema, propondo um hardware confiável, robusto e versátil.  

\section{Pré-Requisitos de Projeto}

Como dito, foi preciso entender os pré-requisitos impostos para o OBDH da terceira geração do SpaceLab. Abaixo, na Tabela \ref{tab:Tab_Req}, se encontram os requisitos gerais do projeto, em conjunto com o pretexto e com o nível de prioridade.

\begin{longtable}{@{}p{5cm}p{5cm}p{3.5cm}@{}}
    \centering
	\ABNTEXfontereduzida
	\label{tab:Tab_Req}\tabularnewline
	\caption{Requisitos do projeto.}\tabularnewline
	%\begin{tabular}{@{}p{2cm}p{2cm}p{2cm}p{2cm}p{2cm}p{2cm}p{3cm}@{}}
	\hline
	\textbf{\centering{Descrição}} & \textbf{\centering{Pretexto}} & \textbf{\centering{Prioridade}} \tabularnewline
        \hline
        O módulo OBDH deve ser compatível com o padrão CubeSat & Assegura compatibilidade com outros satélites desenvolvidos no SpaceLab & Alta \tabularnewline
        
       \hline
        O módulo OBDH deve operar corretamente entre -40°C e 85°C & Para operar com segurança em um ambiente LEO & Alta \tabularnewline

       \hline
        O módulo OBDH deve possuir um microcontrolador capaz de usar um sistema Linux & Para gerenciar e coordenar operações dentro e fora do módulo, sendo capaz de realizar tarefas complexas  & Alta \tabularnewline

       \hline
        O módulo OBDH deve possuir uma memória DDR com capacidade de 512Mb (preferencialmente com ECC)  & Memória suficiente para operações do OBDH e armazenamento de dados  & Alta\tabularnewline

        \hline
        O módulo OBDH deve possuir uma memória FRAM para armazenar parâmetros de configuração & Provê memória não-volátil e duradoura, menos sucetível à radiação & Alta \tabularnewline 

        \hline
        O módulo OBDH deve possuir uma memória Flash para armazenar pacotes (preferencialmente com ECC) & Para armazenar dados e pacotes recebidos & Alta\tabularnewline 

        \hline
        O módulo OBDH deve possuir um WDT para reiniciar o microcontrolador em caso de falha de \textit{software} & Reinicia automaticamente o microcontrolador caso haja a falha  & Alta \tabularnewline

        \hline
        O módulo OBDH deve possuir sensores de medição de tensão e corrente em suas tensões & Para monitoramento de potência consumida & Alta\tabularnewline

        \hline
        O módulo OBDH deve possuir proteção de sobrecorrente (20\% acima do valor nominal) & Para proteção contra \textit{latch-up}  & Alta \tabularnewline

        \hline
        O módulo OBDH deve possuir um giroscópio para medição de velocidade angular & Para permitir controle ativo do satélite  & Alta \tabularnewline 

       \hline
        O módulo OBDH deve possuir um magnetômetro & Para permitir controle ativo do satélite  & Alta \tabularnewline

        \hline
        O módulo OBDH deve possuir uma interface RS-422 para transmissão de mensagens de \textit{debug/log} e receber parâmetros de configuração & Comunicação de longa distância com maior imunidade ao ruído e maior taxa de dados (comparando com UART)  & Alta \tabularnewline

       \hline
        O módulo OBDH deve possuir uma interface CAN para receber e transmitir comandos e dados & Comunicação robusta e com suporte a múltiplos sistemas do CubeSat  & Alta\tabularnewline

        \hline
        O módulo OBDH deve possuir uma interface acessível externamente para programação do microcontrolador & Para o módulo ser facilmente programado pelo time  & Alta \tabularnewline

        \hline
        O módulo OBDH deve possuir uma interface para uma \textit{daughter board} & Para prover suporte a outras interfaces e periféricos  & Baixa \tabularnewline

        \hline
        O módulo OBDH deve possuir um sensor de temperatura com precisão menor ou igual a 1°C & Para prevenir danos de temperaturas extremas & Baixa\tabularnewline

        \hline
        O módulo OBDH deve possuir uma interface RS-485 para receber e transmitir comandos e dados  & Para transmissão robusta de dados com módulos externos & Baixa \tabularnewline
       \hline
	\centering{\fonte{Elaboração própria.}}
\end{longtable}

Com as definições apresentadas na Tabela \ref{tab:Tab_Req}, foi então necessária a definição da arquitetura do hardware, ou seja, os componentes e sua interconexões, bem como as interfaces de comunicação e saídas necessárias.

\section{Arquitetura}

A partir dos requisitos, o primeiro passo foi definir de forma geral como seria o funcionamento do \textit{hardware} do projeto. Na Figura \ref{fig:arq_geral}, pode-se verificar um esquema inicial de proposta de arquitetura, usando os pontos descritos anteriormente.

\begin{figure}[htp]
    \centering
    \includegraphics[scale=0.7]{images/arquitetura geral.png}
    \caption{Arquitetura proposta.}
    \label{fig:arq_geral}
\end{figure}

Como podemos verificar, o microprocessador será crucial e deverá ter pinos suficientes para interface com todas as memórias, sensores e para se comunicar com os outros módulos do CubeSat. Além disso, a parte dedicada às tensões usadas deverá ser cuidadosamente feita, para suportar a potência dissipada por todos os componentes da placa de circuito impresso. A escolha de cada componente será descrita nas seções a seguir, respeitando sempre os seguintes critérios:

\begin{itemize}
	\item O componente deve funcionar corretamente nas temperaturas entre -40°C e 85°C;
	\item Circuitos integrados devem possuir herança de voo sempre que possível;
	\item Caso o circuito integrado necessite de um circuito específico, o mesmo deve conter itens preferencialmente dispostos na ECSS-Q-ST-60C, na NPSL ou similar aos mesmos, especialmente componentes discretos (capacitores, resistores, indutores, diodos, transistores, entre outros);
\end{itemize}

\subsection{Microcontrolador}

Como visto na Tabela \ref{tab:Tab_Missoes}, a fabricante com maior herança de voo estudada é a Xilinx, em especial os chips da família Zynq 7000, que são SoCs (\textit{System on a Chip}). Após um estudo próprio, o SoC Zynq 7030 se mostrou mais adequado pelas seguintes características:

\begin{itemize}
	\item Foi usado em missões extensivas em pequenos satélites (Gomspace, 2024), ou seja, possui herança de voo em missões similares em LEO e em CubeSats;
	\item Possui um envelopamento com 484 pinos, suficiente para prover as conexões necessárias para todas as interfaces requeridas (UG865, 2021);
	\item Capaz de rodar um sistema Linux (KADI et al.,2013);
	\item Por ser um SoC, possui alta adaptabilidade e flexibilidade, disponibilizando no mesmo chip uma FPGA (\textit{Field-Programmable Gate Array}) e um microprocessador, denominados respectivamente de PL e PS;
\end{itemize}

\subsection{Memórias}

As memórias serão necessárias para realizar operações, armazenar dados externos e internos e armazenar parâmetros de configuração do OBDH e de outros subsistemas do CubeSat. Para cada uma dessas funções uma memória diferente é necessária, seguindo suas características principais, sendo elas: tempo de acesso, tamanho do armazenamento e volatilidade.

\subsubsection{Memórias voláteis}

Partindo do princípio que a robustez e versatilidade estão alinhadas com a velocidade da memória, bem como sua capacidade de armazenamento máximo, a principal opção se tornou as memórias do tipo DDR (\textit{Double Data Rate}), que utilizam ambas a borda de subida e de descida para transferência de dados, atingindo o dobro de largura de banda de uma memória com SDR (\textit{Single Data Rate}) para uma mesma frequência de relógio (JEDEC, 2008). Essa relação pode ser ilustrada pela Figura \ref{fig:sdrvsddr}, onde pode-se verificar a transferência de dados do sinal DQ em relação ao sinal de relógio (bCLK e CLK) para SDR e DDR.

\begin{figure}[htp]
    \centering
    \includegraphics[scale=1]{images/ddrsdr.png}
    \caption{Comparação entre DDR e SDR (KLEHN E BROX, 2003).}
    \label{fig:sdrvsddr}
\end{figure}
 
Por essa razão, foi escolhida uma memória do tipo DDR3, com capacidade de 2Gb e frequência de operação de 800 MHz.

\subsubsection{Memórias não voláteis}

No caso das memórias não voláteis, é necessária uma atenção especial ao tipo de dado que será armazenado em cada uma delas. Para o caso de dados críticos, é preciso de uma memória que possua alta resistência aos efeitos da radiação, mantendo-se um compromisso com os tempos de escrita e leitura. Por sua vez, para dados de inicialização são mais críticos os tempos de leitura, enquanto para uma memória de dados mais gerais, o importante é o armazenamento total. Por meio desses critérios, foi possível avaliar, por meio da Tabela \ref{tab:memnvol}, o tipo de memória ideal para cada caso.

\begin{table}[H]
	\ABNTEXfontereduzida
	\caption{\label{tab:memnvol}Tabela comparativa de memórias não voláteis.}
	%\begin{tabular}{@{}p{2cm}p{2cm}p{2cm}p{2cm}p{2cm}p{2cm}p{3cm}@{}}
    \centering
    \begin{tabular}{@{} >{\centering}p{2cm} >{\centering}p{3cm} >{\centering}p{3cm} >{\centering}p{3cm}>{\centering}p{3cm} @{}}
    
		\toprule
		\textbf{Memória} & \textbf{Tempo de leitura} & \textbf{Tempo de escrita} & \textbf{Tolerância à radiação} & \textbf{Armazenamento máximo} \tabularnewline 
        \midrule
        Flash NOR & Rápido & Lento & Ruim & Regular\tabularnewline
        
        \midrule
        Flash NAND & Rápido & Lento & Ruim & Bom \tabularnewline 

        \midrule
        FRAM & Rápido & Rápido & Bom & Ruim \tabularnewline 
        
        \bottomrule
	\end{tabular}
	\fonte{Adaptado de GERARDIN E PACCAGNELLA, 2010.}
\end{table}

Com isso, foi então escolhida uma FRAM (\textit{Ferroelectric Random-Access Memory}) para armazenar dados críticos, uma Flash NAND para armazenamento de dados gerais e uma Flash NOR para armazenar o boot do sistema Linux no SoC.

\subsection{Conversores DC-DC}
Nos sistemas CubeSat do SpaceLab da UFSC, o módulo responsável pelo fornecimento de potência é o chamado EPS (MARCELINO, 2024). A partir disso, partindo do pressuposto que haverá uma tensão fornecida de 3,3 V, pode-se inferir a cascata de potência a partir do mesmo. Para o caso do Zynq e da memória DDR3, circuitos integrados são necessários para gerar as seguintes tensões: 

\begin{itemize}
	\item Zynq: 1 V e 1,8 V; 
	\item DDR3: 1,35 V e 0,675 V.
\end{itemize}

Todos os demais periféricos devem aceitar uma tensão de alimentação de 3,3 V. Outro ponto importante são os circuitos de proteção contra \textit{latch-up}, um efeito similar a um curto-circuito na trilha de alimentação de circuitos CMOS (AN-600, 1989). 

\subsection{Sensores e Periféricos}

\subsection{Conectores}

\section{Interfaces de comunicação}

\subsection{I2C}

\subsection{SPI}

\subsection{CAN}

\subsection{UART}

\section{Visualização da Arquitetura Proposta}