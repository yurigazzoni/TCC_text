% ----------------------------------------------------------
\chapter{Conclusão}
% ----------------------------------------------------------

Conclui-se que o projeto realizado cumpre os objetivos propostos, apresentando uma solução robusta, segura e adaptável para a operação em CubeSats, respeitando os requisitos. A flexibilidade conferida pela arquitetura do SoC (com FPGA integrada), associada à robustez das interfaces de comunicação e ao uso de memórias para fuções específicas permite que o sistema seja facilmente adaptável a diferentes missões. A partir das diretrizes iniciais, os resultados planejados foram atingidos, validando a arquitetura desenvolvida e sua viabilidade para diferentes aplicações em missões usando o padrão CubeSat. 

Outro ponto importante é que o OBDH desenvolvido apresenta um potencial significativo para aplicações futuras tanto no SpaceLab quanto em outras instituições. O módulo ter sido projetado como adaptável o permite ser utilizado em missões variadas, com diferentes requisitos e subsistemas. Além disso, sendo um projeto aberto, outras instituições poderão adotar essa solução conforme suas necessidades específicas, o que amplia o escopo de uso e promove um ambiente colaborativo de desenvolvimento de tecnologia espacial. No entanto, esse trabalho é apenas o início, com aprimoramentos e implementações a serem realizados em trabalhos futuros.

\section{Trabalhos Futuros}

Para dar continuidade ao desenvolvimento deste OBDH, ainda restam etapas importantes. A primeira delas é o desenvolvimento do layout da placa de circuito impresso, seguindo as diretrizes da ESA (ECSS, 2014). Depois disso, devem ser conduzidos os testes dos circuitos, visando validar o funcionamento e observar possíveis pontos de melhoria. Outra expectativa é o desenvolvimento do \textit{firmware} do sistema, para que o SoC atue no controle das operações e na comunicação com periféricos, sensores e subsistemas do CubeSat, usando um sistema Linux. Além disso, deve-se desenvolver a parte mecânica requerida, com a conexão entre os submódulos e a estrutura metálica.

Por fim, a proteção contra TID também é uma continuação importante e interessante. Um estudo sobre o uso de \textit{shields} de proteção, a fim de atenuar esse efeito sobre os componentes e sobre a placa como um todo seria de grande valia, visto que isso melhoraria a durabilidade do OBDH e aumentaria a resistência do sistema aos efeitos da radiação em LEO.
